\documentclass[10pt,letter]{article}
	% basic article document class
	% use percent signs to make comments to yourself -- they will not show up.

\usepackage{amsmath}
\usepackage{amssymb}
	% packages that allow mathematical formatting

\usepackage{graphicx}
\renewcommand{\arraystretch}{1.5}
	% package that allows you to include graphics
\usepackage{multirow}
\usepackage{setspace}
	% package that allows you to change spacing


    \usepackage[breakable]{tcolorbox}
    \usepackage{parskip} % Stop auto-indenting (to mimic markdown behaviour)

    \usepackage{iftex}
    \ifPDFTeX
    	\usepackage[T1]{fontenc}
    	\usepackage{mathpazo}
    \else
    	\usepackage{fontspec}
    \fi

    % Basic figure setup, for now with no caption control since it's done
    % automatically by Pandoc (which extracts ![](path) syntax from Markdown).
    \usepackage{graphicx}
    % Maintain compatibility with old templates. Remove in nbconvert 6.0
    \let\Oldincludegraphics\includegraphics
    % Ensure that by default, figures have no caption (until we provide a
    % proper Figure object with a Caption API and a way to capture that
    % in the conversion process - todo).
    \usepackage{caption}
    \DeclareCaptionFormat{nocaption}{}
    \captionsetup{format=nocaption,aboveskip=0pt,belowskip=0pt}

    \usepackage[Export]{adjustbox} % Used to constrain images to a maximum size
    \adjustboxset{max size={0.9\linewidth}{0.9\paperheight}}
    \usepackage{float}
    \floatplacement{figure}{H} % forces figures to be placed at the correct location
    \usepackage{xcolor} % Allow colors to be defined
    \usepackage{enumerate} % Needed for markdown enumerations to work
    \usepackage{geometry} % Used to adjust the document margins
    \usepackage{amsmath} % Equations
    \usepackage{amssymb} % Equations
    \usepackage{textcomp} % defines textquotesingle
    % Hack from http://tex.stackexchange.com/a/47451/13684:
    \AtBeginDocument{%
        \def\PYZsq{\textquotesingle}% Upright quotes in Pygmentized code
    }
    \usepackage{upquote} % Upright quotes for verbatim code
    \usepackage{eurosym} % defines \euro
    \usepackage[mathletters]{ucs} % Extended unicode (utf-8) support
    \usepackage{fancyvrb} % verbatim replacement that allows latex
    \usepackage{grffile} % extends the file name processing of package graphics
                         % to support a larger range
    \makeatletter % fix for grffile with XeLaTeX
    \def\Gread@@xetex#1{%
      \IfFileExists{"\Gin@base".bb}%
      {\Gread@eps{\Gin@base.bb}}%
      {\Gread@@xetex@aux#1}%
    }
    \makeatother

    % The hyperref package gives us a pdf with properly built
    % internal navigation ('pdf bookmarks' for the table of contents,
    % internal cross-reference links, web links for URLs, etc.)
    \usepackage{hyperref}
    % The default LaTeX title has an obnoxious amount of whitespace. By default,
    % titling removes some of it. It also provides customization options.
    \usepackage{titling}
    \usepackage{longtable} % longtable support required by pandoc >1.10
    \usepackage{booktabs}  % table support for pandoc > 1.12.2
    \usepackage[inline]{enumitem} % IRkernel/repr support (it uses the enumerate* environment)
    \usepackage[normalem]{ulem} % ulem is needed to support strikethroughs (\sout)
                                % normalem makes italics be italics, not underlines
    \usepackage{mathrsfs}



    % Colors for the hyperref package
    \definecolor{urlcolor}{rgb}{0,.145,.698}
    \definecolor{linkcolor}{rgb}{.71,0.21,0.01}
    \definecolor{citecolor}{rgb}{.12,.54,.11}

    % ANSI colors
    \definecolor{ansi-black}{HTML}{3E424D}
    \definecolor{ansi-black-intense}{HTML}{282C36}
    \definecolor{ansi-red}{HTML}{E75C58}
    \definecolor{ansi-red-intense}{HTML}{B22B31}
    \definecolor{ansi-green}{HTML}{00A250}
    \definecolor{ansi-green-intense}{HTML}{007427}
    \definecolor{ansi-yellow}{HTML}{DDB62B}
    \definecolor{ansi-yellow-intense}{HTML}{B27D12}
    \definecolor{ansi-blue}{HTML}{208FFB}
    \definecolor{ansi-blue-intense}{HTML}{0065CA}
    \definecolor{ansi-magenta}{HTML}{D160C4}
    \definecolor{ansi-magenta-intense}{HTML}{A03196}
    \definecolor{ansi-cyan}{HTML}{60C6C8}
    \definecolor{ansi-cyan-intense}{HTML}{258F8F}
    \definecolor{ansi-white}{HTML}{C5C1B4}
    \definecolor{ansi-white-intense}{HTML}{A1A6B2}
    \definecolor{ansi-default-inverse-fg}{HTML}{FFFFFF}
    \definecolor{ansi-default-inverse-bg}{HTML}{000000}

    % commands and environments needed by pandoc snippets
    % extracted from the output of `pandoc -s`
    \providecommand{\tightlist}{%
      \setlength{\itemsep}{0pt}\setlength{\parskip}{0pt}}
    \DefineVerbatimEnvironment{Highlighting}{Verbatim}{commandchars=\\\{\}}
    % Add ',fontsize=\small' for more characters per line
    \newenvironment{Shaded}{}{}
    \newcommand{\KeywordTok}[1]{\textcolor[rgb]{0.00,0.44,0.13}{\textbf{{#1}}}}
    \newcommand{\DataTypeTok}[1]{\textcolor[rgb]{0.56,0.13,0.00}{{#1}}}
    \newcommand{\DecValTok}[1]{\textcolor[rgb]{0.25,0.63,0.44}{{#1}}}
    \newcommand{\BaseNTok}[1]{\textcolor[rgb]{0.25,0.63,0.44}{{#1}}}
    \newcommand{\FloatTok}[1]{\textcolor[rgb]{0.25,0.63,0.44}{{#1}}}
    \newcommand{\CharTok}[1]{\textcolor[rgb]{0.25,0.44,0.63}{{#1}}}
    \newcommand{\StringTok}[1]{\textcolor[rgb]{0.25,0.44,0.63}{{#1}}}
    \newcommand{\CommentTok}[1]{\textcolor[rgb]{0.38,0.63,0.69}{\textit{{#1}}}}
    \newcommand{\OtherTok}[1]{\textcolor[rgb]{0.00,0.44,0.13}{{#1}}}
    \newcommand{\AlertTok}[1]{\textcolor[rgb]{1.00,0.00,0.00}{\textbf{{#1}}}}
    \newcommand{\FunctionTok}[1]{\textcolor[rgb]{0.02,0.16,0.49}{{#1}}}
    \newcommand{\RegionMarkerTok}[1]{{#1}}
    \newcommand{\ErrorTok}[1]{\textcolor[rgb]{1.00,0.00,0.00}{\textbf{{#1}}}}
    \newcommand{\NormalTok}[1]{{#1}}

    % Additional commands for more recent versions of Pandoc
    \newcommand{\ConstantTok}[1]{\textcolor[rgb]{0.53,0.00,0.00}{{#1}}}
    \newcommand{\SpecialCharTok}[1]{\textcolor[rgb]{0.25,0.44,0.63}{{#1}}}
    \newcommand{\VerbatimStringTok}[1]{\textcolor[rgb]{0.25,0.44,0.63}{{#1}}}
    \newcommand{\SpecialStringTok}[1]{\textcolor[rgb]{0.73,0.40,0.53}{{#1}}}
    \newcommand{\ImportTok}[1]{{#1}}
    \newcommand{\DocumentationTok}[1]{\textcolor[rgb]{0.73,0.13,0.13}{\textit{{#1}}}}
    \newcommand{\AnnotationTok}[1]{\textcolor[rgb]{0.38,0.63,0.69}{\textbf{\textit{{#1}}}}}
    \newcommand{\CommentVarTok}[1]{\textcolor[rgb]{0.38,0.63,0.69}{\textbf{\textit{{#1}}}}}
    \newcommand{\VariableTok}[1]{\textcolor[rgb]{0.10,0.09,0.49}{{#1}}}
    \newcommand{\ControlFlowTok}[1]{\textcolor[rgb]{0.00,0.44,0.13}{\textbf{{#1}}}}
    \newcommand{\OperatorTok}[1]{\textcolor[rgb]{0.40,0.40,0.40}{{#1}}}
    \newcommand{\BuiltInTok}[1]{{#1}}
    \newcommand{\ExtensionTok}[1]{{#1}}
    \newcommand{\PreprocessorTok}[1]{\textcolor[rgb]{0.74,0.48,0.00}{{#1}}}
    \newcommand{\AttributeTok}[1]{\textcolor[rgb]{0.49,0.56,0.16}{{#1}}}
    \newcommand{\InformationTok}[1]{\textcolor[rgb]{0.38,0.63,0.69}{\textbf{\textit{{#1}}}}}
    \newcommand{\WarningTok}[1]{\textcolor[rgb]{0.38,0.63,0.69}{\textbf{\textit{{#1}}}}}


    % Define a nice break command that doesn't care if a line doesn't already
    % exist.
    \def\br{\hspace*{\fill} \\* }
    % Math Jax compatibility definitions
    \def\gt{>}
    \def\lt{<}
    \let\Oldtex\TeX
    \let\Oldlatex\LaTeX
    \renewcommand{\TeX}{\textrm{\Oldtex}}
    \renewcommand{\LaTeX}{\textrm{\Oldlatex}}
    % Document parameters
    % Document title
    \title{MergerHomework}





% Pygments definitions
\makeatletter
\def\PY@reset{\let\PY@it=\relax \let\PY@bf=\relax%
    \let\PY@ul=\relax \let\PY@tc=\relax%
    \let\PY@bc=\relax \let\PY@ff=\relax}
\def\PY@tok#1{\csname PY@tok@#1\endcsname}
\def\PY@toks#1+{\ifx\relax#1\empty\else%
    \PY@tok{#1}\expandafter\PY@toks\fi}
\def\PY@do#1{\PY@bc{\PY@tc{\PY@ul{%
    \PY@it{\PY@bf{\PY@ff{#1}}}}}}}
\def\PY#1#2{\PY@reset\PY@toks#1+\relax+\PY@do{#2}}

\expandafter\def\csname PY@tok@w\endcsname{\def\PY@tc##1{\textcolor[rgb]{0.73,0.73,0.73}{##1}}}
\expandafter\def\csname PY@tok@c\endcsname{\let\PY@it=\textit\def\PY@tc##1{\textcolor[rgb]{0.25,0.50,0.50}{##1}}}
\expandafter\def\csname PY@tok@cp\endcsname{\def\PY@tc##1{\textcolor[rgb]{0.74,0.48,0.00}{##1}}}
\expandafter\def\csname PY@tok@k\endcsname{\let\PY@bf=\textbf\def\PY@tc##1{\textcolor[rgb]{0.00,0.50,0.00}{##1}}}
\expandafter\def\csname PY@tok@kp\endcsname{\def\PY@tc##1{\textcolor[rgb]{0.00,0.50,0.00}{##1}}}
\expandafter\def\csname PY@tok@kt\endcsname{\def\PY@tc##1{\textcolor[rgb]{0.69,0.00,0.25}{##1}}}
\expandafter\def\csname PY@tok@o\endcsname{\def\PY@tc##1{\textcolor[rgb]{0.40,0.40,0.40}{##1}}}
\expandafter\def\csname PY@tok@ow\endcsname{\let\PY@bf=\textbf\def\PY@tc##1{\textcolor[rgb]{0.67,0.13,1.00}{##1}}}
\expandafter\def\csname PY@tok@nb\endcsname{\def\PY@tc##1{\textcolor[rgb]{0.00,0.50,0.00}{##1}}}
\expandafter\def\csname PY@tok@nf\endcsname{\def\PY@tc##1{\textcolor[rgb]{0.00,0.00,1.00}{##1}}}
\expandafter\def\csname PY@tok@nc\endcsname{\let\PY@bf=\textbf\def\PY@tc##1{\textcolor[rgb]{0.00,0.00,1.00}{##1}}}
\expandafter\def\csname PY@tok@nn\endcsname{\let\PY@bf=\textbf\def\PY@tc##1{\textcolor[rgb]{0.00,0.00,1.00}{##1}}}
\expandafter\def\csname PY@tok@ne\endcsname{\let\PY@bf=\textbf\def\PY@tc##1{\textcolor[rgb]{0.82,0.25,0.23}{##1}}}
\expandafter\def\csname PY@tok@nv\endcsname{\def\PY@tc##1{\textcolor[rgb]{0.10,0.09,0.49}{##1}}}
\expandafter\def\csname PY@tok@no\endcsname{\def\PY@tc##1{\textcolor[rgb]{0.53,0.00,0.00}{##1}}}
\expandafter\def\csname PY@tok@nl\endcsname{\def\PY@tc##1{\textcolor[rgb]{0.63,0.63,0.00}{##1}}}
\expandafter\def\csname PY@tok@ni\endcsname{\let\PY@bf=\textbf\def\PY@tc##1{\textcolor[rgb]{0.60,0.60,0.60}{##1}}}
\expandafter\def\csname PY@tok@na\endcsname{\def\PY@tc##1{\textcolor[rgb]{0.49,0.56,0.16}{##1}}}
\expandafter\def\csname PY@tok@nt\endcsname{\let\PY@bf=\textbf\def\PY@tc##1{\textcolor[rgb]{0.00,0.50,0.00}{##1}}}
\expandafter\def\csname PY@tok@nd\endcsname{\def\PY@tc##1{\textcolor[rgb]{0.67,0.13,1.00}{##1}}}
\expandafter\def\csname PY@tok@s\endcsname{\def\PY@tc##1{\textcolor[rgb]{0.73,0.13,0.13}{##1}}}
\expandafter\def\csname PY@tok@sd\endcsname{\let\PY@it=\textit\def\PY@tc##1{\textcolor[rgb]{0.73,0.13,0.13}{##1}}}
\expandafter\def\csname PY@tok@si\endcsname{\let\PY@bf=\textbf\def\PY@tc##1{\textcolor[rgb]{0.73,0.40,0.53}{##1}}}
\expandafter\def\csname PY@tok@se\endcsname{\let\PY@bf=\textbf\def\PY@tc##1{\textcolor[rgb]{0.73,0.40,0.13}{##1}}}
\expandafter\def\csname PY@tok@sr\endcsname{\def\PY@tc##1{\textcolor[rgb]{0.73,0.40,0.53}{##1}}}
\expandafter\def\csname PY@tok@ss\endcsname{\def\PY@tc##1{\textcolor[rgb]{0.10,0.09,0.49}{##1}}}
\expandafter\def\csname PY@tok@sx\endcsname{\def\PY@tc##1{\textcolor[rgb]{0.00,0.50,0.00}{##1}}}
\expandafter\def\csname PY@tok@m\endcsname{\def\PY@tc##1{\textcolor[rgb]{0.40,0.40,0.40}{##1}}}
\expandafter\def\csname PY@tok@gh\endcsname{\let\PY@bf=\textbf\def\PY@tc##1{\textcolor[rgb]{0.00,0.00,0.50}{##1}}}
\expandafter\def\csname PY@tok@gu\endcsname{\let\PY@bf=\textbf\def\PY@tc##1{\textcolor[rgb]{0.50,0.00,0.50}{##1}}}
\expandafter\def\csname PY@tok@gd\endcsname{\def\PY@tc##1{\textcolor[rgb]{0.63,0.00,0.00}{##1}}}
\expandafter\def\csname PY@tok@gi\endcsname{\def\PY@tc##1{\textcolor[rgb]{0.00,0.63,0.00}{##1}}}
\expandafter\def\csname PY@tok@gr\endcsname{\def\PY@tc##1{\textcolor[rgb]{1.00,0.00,0.00}{##1}}}
\expandafter\def\csname PY@tok@ge\endcsname{\let\PY@it=\textit}
\expandafter\def\csname PY@tok@gs\endcsname{\let\PY@bf=\textbf}
\expandafter\def\csname PY@tok@gp\endcsname{\let\PY@bf=\textbf\def\PY@tc##1{\textcolor[rgb]{0.00,0.00,0.50}{##1}}}
\expandafter\def\csname PY@tok@go\endcsname{\def\PY@tc##1{\textcolor[rgb]{0.53,0.53,0.53}{##1}}}
\expandafter\def\csname PY@tok@gt\endcsname{\def\PY@tc##1{\textcolor[rgb]{0.00,0.27,0.87}{##1}}}
\expandafter\def\csname PY@tok@err\endcsname{\def\PY@bc##1{\setlength{\fboxsep}{0pt}\fcolorbox[rgb]{1.00,0.00,0.00}{1,1,1}{\strut ##1}}}
\expandafter\def\csname PY@tok@kc\endcsname{\let\PY@bf=\textbf\def\PY@tc##1{\textcolor[rgb]{0.00,0.50,0.00}{##1}}}
\expandafter\def\csname PY@tok@kd\endcsname{\let\PY@bf=\textbf\def\PY@tc##1{\textcolor[rgb]{0.00,0.50,0.00}{##1}}}
\expandafter\def\csname PY@tok@kn\endcsname{\let\PY@bf=\textbf\def\PY@tc##1{\textcolor[rgb]{0.00,0.50,0.00}{##1}}}
\expandafter\def\csname PY@tok@kr\endcsname{\let\PY@bf=\textbf\def\PY@tc##1{\textcolor[rgb]{0.00,0.50,0.00}{##1}}}
\expandafter\def\csname PY@tok@bp\endcsname{\def\PY@tc##1{\textcolor[rgb]{0.00,0.50,0.00}{##1}}}
\expandafter\def\csname PY@tok@fm\endcsname{\def\PY@tc##1{\textcolor[rgb]{0.00,0.00,1.00}{##1}}}
\expandafter\def\csname PY@tok@vc\endcsname{\def\PY@tc##1{\textcolor[rgb]{0.10,0.09,0.49}{##1}}}
\expandafter\def\csname PY@tok@vg\endcsname{\def\PY@tc##1{\textcolor[rgb]{0.10,0.09,0.49}{##1}}}
\expandafter\def\csname PY@tok@vi\endcsname{\def\PY@tc##1{\textcolor[rgb]{0.10,0.09,0.49}{##1}}}
\expandafter\def\csname PY@tok@vm\endcsname{\def\PY@tc##1{\textcolor[rgb]{0.10,0.09,0.49}{##1}}}
\expandafter\def\csname PY@tok@sa\endcsname{\def\PY@tc##1{\textcolor[rgb]{0.73,0.13,0.13}{##1}}}
\expandafter\def\csname PY@tok@sb\endcsname{\def\PY@tc##1{\textcolor[rgb]{0.73,0.13,0.13}{##1}}}
\expandafter\def\csname PY@tok@sc\endcsname{\def\PY@tc##1{\textcolor[rgb]{0.73,0.13,0.13}{##1}}}
\expandafter\def\csname PY@tok@dl\endcsname{\def\PY@tc##1{\textcolor[rgb]{0.73,0.13,0.13}{##1}}}
\expandafter\def\csname PY@tok@s2\endcsname{\def\PY@tc##1{\textcolor[rgb]{0.73,0.13,0.13}{##1}}}
\expandafter\def\csname PY@tok@sh\endcsname{\def\PY@tc##1{\textcolor[rgb]{0.73,0.13,0.13}{##1}}}
\expandafter\def\csname PY@tok@s1\endcsname{\def\PY@tc##1{\textcolor[rgb]{0.73,0.13,0.13}{##1}}}
\expandafter\def\csname PY@tok@mb\endcsname{\def\PY@tc##1{\textcolor[rgb]{0.40,0.40,0.40}{##1}}}
\expandafter\def\csname PY@tok@mf\endcsname{\def\PY@tc##1{\textcolor[rgb]{0.40,0.40,0.40}{##1}}}
\expandafter\def\csname PY@tok@mh\endcsname{\def\PY@tc##1{\textcolor[rgb]{0.40,0.40,0.40}{##1}}}
\expandafter\def\csname PY@tok@mi\endcsname{\def\PY@tc##1{\textcolor[rgb]{0.40,0.40,0.40}{##1}}}
\expandafter\def\csname PY@tok@il\endcsname{\def\PY@tc##1{\textcolor[rgb]{0.40,0.40,0.40}{##1}}}
\expandafter\def\csname PY@tok@mo\endcsname{\def\PY@tc##1{\textcolor[rgb]{0.40,0.40,0.40}{##1}}}
\expandafter\def\csname PY@tok@ch\endcsname{\let\PY@it=\textit\def\PY@tc##1{\textcolor[rgb]{0.25,0.50,0.50}{##1}}}
\expandafter\def\csname PY@tok@cm\endcsname{\let\PY@it=\textit\def\PY@tc##1{\textcolor[rgb]{0.25,0.50,0.50}{##1}}}
\expandafter\def\csname PY@tok@cpf\endcsname{\let\PY@it=\textit\def\PY@tc##1{\textcolor[rgb]{0.25,0.50,0.50}{##1}}}
\expandafter\def\csname PY@tok@c1\endcsname{\let\PY@it=\textit\def\PY@tc##1{\textcolor[rgb]{0.25,0.50,0.50}{##1}}}
\expandafter\def\csname PY@tok@cs\endcsname{\let\PY@it=\textit\def\PY@tc##1{\textcolor[rgb]{0.25,0.50,0.50}{##1}}}

\def\PYZbs{\char`\\}
\def\PYZus{\char`\_}
\def\PYZob{\char`\{}
\def\PYZcb{\char`\}}
\def\PYZca{\char`\^}
\def\PYZam{\char`\&}
\def\PYZlt{\char`\<}
\def\PYZgt{\char`\>}
\def\PYZsh{\char`\#}
\def\PYZpc{\char`\%}
\def\PYZdl{\char`\$}
\def\PYZhy{\char`\-}
\def\PYZsq{\char`\'}
\def\PYZdq{\char`\"}
\def\PYZti{\char`\~}
% for compatibility with earlier versions
\def\PYZat{@}
\def\PYZlb{[}
\def\PYZrb{]}
\makeatother


    % For linebreaks inside Verbatim environment from package fancyvrb.
    \makeatletter
        \newbox\Wrappedcontinuationbox
        \newbox\Wrappedvisiblespacebox
        \newcommand*\Wrappedvisiblespace {\textcolor{red}{\textvisiblespace}}
        \newcommand*\Wrappedcontinuationsymbol {\textcolor{red}{\llap{\tiny$\m@th\hookrightarrow$}}}
        \newcommand*\Wrappedcontinuationindent {3ex }
        \newcommand*\Wrappedafterbreak {\kern\Wrappedcontinuationindent\copy\Wrappedcontinuationbox}
        % Take advantage of the already applied Pygments mark-up to insert
        % potential linebreaks for TeX processing.
        %        {, <, #, %, $, ' and ": go to next line.
        %        _, }, ^, &, >, - and ~: stay at end of broken line.
        % Use of \textquotesingle for straight quote.
        \newcommand*\Wrappedbreaksatspecials {%
            \def\PYGZus{\discretionary{\char`\_}{\Wrappedafterbreak}{\char`\_}}%
            \def\PYGZob{\discretionary{}{\Wrappedafterbreak\char`\{}{\char`\{}}%
            \def\PYGZcb{\discretionary{\char`\}}{\Wrappedafterbreak}{\char`\}}}%
            \def\PYGZca{\discretionary{\char`\^}{\Wrappedafterbreak}{\char`\^}}%
            \def\PYGZam{\discretionary{\char`\&}{\Wrappedafterbreak}{\char`\&}}%
            \def\PYGZlt{\discretionary{}{\Wrappedafterbreak\char`\<}{\char`\<}}%
            \def\PYGZgt{\discretionary{\char`\>}{\Wrappedafterbreak}{\char`\>}}%
            \def\PYGZsh{\discretionary{}{\Wrappedafterbreak\char`\#}{\char`\#}}%
            \def\PYGZpc{\discretionary{}{\Wrappedafterbreak\char`\%}{\char`\%}}%
            \def\PYGZdl{\discretionary{}{\Wrappedafterbreak\char`\$}{\char`\$}}%
            \def\PYGZhy{\discretionary{\char`\-}{\Wrappedafterbreak}{\char`\-}}%
            \def\PYGZsq{\discretionary{}{\Wrappedafterbreak\textquotesingle}{\textquotesingle}}%
            \def\PYGZdq{\discretionary{}{\Wrappedafterbreak\char`\"}{\char`\"}}%
            \def\PYGZti{\discretionary{\char`\~}{\Wrappedafterbreak}{\char`\~}}%
        }
        % Some characters . , ; ? ! / are not pygmentized.
        % This macro makes them "active" and they will insert potential linebreaks
        \newcommand*\Wrappedbreaksatpunct {%
            \lccode`\~`\.\lowercase{\def~}{\discretionary{\hbox{\char`\.}}{\Wrappedafterbreak}{\hbox{\char`\.}}}%
            \lccode`\~`\,\lowercase{\def~}{\discretionary{\hbox{\char`\,}}{\Wrappedafterbreak}{\hbox{\char`\,}}}%
            \lccode`\~`\;\lowercase{\def~}{\discretionary{\hbox{\char`\;}}{\Wrappedafterbreak}{\hbox{\char`\;}}}%
            \lccode`\~`\:\lowercase{\def~}{\discretionary{\hbox{\char`\:}}{\Wrappedafterbreak}{\hbox{\char`\:}}}%
            \lccode`\~`\?\lowercase{\def~}{\discretionary{\hbox{\char`\?}}{\Wrappedafterbreak}{\hbox{\char`\?}}}%
            \lccode`\~`\!\lowercase{\def~}{\discretionary{\hbox{\char`\!}}{\Wrappedafterbreak}{\hbox{\char`\!}}}%
            \lccode`\~`\/\lowercase{\def~}{\discretionary{\hbox{\char`\/}}{\Wrappedafterbreak}{\hbox{\char`\/}}}%
            \catcode`\.\active
            \catcode`\,\active
            \catcode`\;\active
            \catcode`\:\active
            \catcode`\?\active
            \catcode`\!\active
            \catcode`\/\active
            \lccode`\~`\~
        }
    \makeatother

    \let\OriginalVerbatim=\Verbatim
    \makeatletter
    \renewcommand{\Verbatim}[1][1]{%
        %\parskip\z@skip
        \sbox\Wrappedcontinuationbox {\Wrappedcontinuationsymbol}%
        \sbox\Wrappedvisiblespacebox {\FV@SetupFont\Wrappedvisiblespace}%
        \def\FancyVerbFormatLine ##1{\hsize\linewidth
            \vtop{\raggedright\hyphenpenalty\z@\exhyphenpenalty\z@
                \doublehyphendemerits\z@\finalhyphendemerits\z@
                \strut ##1\strut}%
        }%
        % If the linebreak is at a space, the latter will be displayed as visible
        % space at end of first line, and a continuation symbol starts next line.
        % Stretch/shrink are however usually zero for typewriter font.
        \def\FV@Space {%
            \nobreak\hskip\z@ plus\fontdimen3\font minus\fontdimen4\font
            \discretionary{\copy\Wrappedvisiblespacebox}{\Wrappedafterbreak}
            {\kern\fontdimen2\font}%
        }%

        % Allow breaks at special characters using \PYG... macros.
        \Wrappedbreaksatspecials
        % Breaks at punctuation characters . , ; ? ! and / need catcode=\active
        \OriginalVerbatim[#1,codes*=\Wrappedbreaksatpunct]%
    }
    \makeatother

    % Exact colors from NB
    \definecolor{incolor}{HTML}{303F9F}
    \definecolor{outcolor}{HTML}{D84315}
    \definecolor{cellborder}{HTML}{CFCFCF}
    \definecolor{cellbackground}{HTML}{F7F7F7}

    % prompt
    \makeatletter
    \newcommand{\boxspacing}{\kern\kvtcb@left@rule\kern\kvtcb@boxsep}
    \makeatother
    \newcommand{\prompt}[4]{
        \ttfamily\llap{{\color{#2}[#3]:\hspace{3pt}#4}}\vspace{-\baselineskip}
    }



    % Prevent overflowing lines due to hard-to-break entities
    \sloppy
    % Setup hyperref package
    \hypersetup{
      breaklinks=true,  % so long urls are correctly broken across lines
      colorlinks=true,
      urlcolor=urlcolor,
      linkcolor=linkcolor,
      citecolor=citecolor,
      }
    % Slightly bigger margins than the latex defaults

    \geometry{verbose,tmargin=1in,bmargin=1in,lmargin=1in,rmargin=1in}


\onehalfspacing
	% text become 1.5 spaced

\usepackage{fullpage}
	% package that specifies normal margins

\usepackage{listings}
\usepackage{color}
\definecolor{dkgreen}{rgb}{0,0.6,0}
\definecolor{gray}{rgb}{0.5,0.5,0.5}
\definecolor{mauve}{rgb}{0.58,0,0.82}
\lstset{frame=tb,
  language=Python,
  aboveskip=3mm,
  belowskip=3mm,
  showstringspaces=false,
  columns=flexible,
  basicstyle={\small\ttfamily},
  numbers=none,
  numberstyle=\tiny\color{gray},
  keywordstyle=\color{blue},
  commentstyle=\color{dkgreen},
  stringstyle=\color{mauve},
  breaklines=true,
  breakatwhitespace=true,
  tabsize=3 }
 % package for writing code

\usepackage{enumitem}

\begin{document}
	% line of code telling latex that your document is beginning


\title{ECON 600: Merger Homework}

\author{Nicholas Wu}

\date{Fall 2021}
	% Note: when you omit this command, the current dateis automatically included

\maketitle
	% tells latex to follow your header (e.g., title, author) commands.

All code is in Python.

\section*{3: Generating Data}
\paragraph{(1)} See code.
\paragraph{(2)}
\begin{enumerate}[label=(\alph*)]
\item \begin{enumerate}[label=(\roman*)]
\item We first note that in the parameter specification,

\[ \overline{\beta^{(2)}} = 4 \]
\[ \overline{\beta^{(3)}} = 4 \]

Hence, defining, $\sigma^{(2)} = \sigma^{(3)} = 1$, we have that
\[ \beta_{it}^{(2)} = \overline{\beta^{(2)}} + \sigma^{(2)} \nu_{it}^{(2)}  \]
\[ \beta_{it}^{(3)} = \overline{\beta^{(3)}} + \sigma^{(3)} \nu_{it}^{(3)}  \]
where $\nu^{(2)}_i$ and $\nu^{(3)}_i$ are i.i.d standard normal.

Define
\[ \delta_{jt} = x_{jt} + \overline{\beta^{(2)}}satellite_j +\overline{\beta^{(3)}}wired_j + \alpha p_{jt} + \xi_{jt} \]
\[ \mu_{ijt} = \sigma^{(2)}satellite_j \nu_{it}^{(2)}  + \sigma^{(3)}wired_j \nu_{it}^{(3)}  \]

The multinomial logit choice probabilities are, conditional on all realized coefficients,
\[ s_{0t} = \int \frac{1}{Z} \ d\Phi(\nu) \]
\[ s_{jt} = \int\frac{\exp(\delta_{jt} + \mu_{ijt})}{Z} \ d\Phi(\nu)  \]
for $j > 0$. Where
\[ Z = 1 + \sum_{j=1}^J \exp(\delta_{jt} + \mu_{ijt}) \]
Then the derivatives are
\[ \frac{\partial s_{jt}}{\partial p_j} = \int\frac{\alpha \exp(\delta_{jt} + \mu_{ijt}) Z - \exp(\delta_{jt} + \mu_{ijt})\left(\alpha\exp(\delta_{jt} + \mu_{ijt})\right)}{Z^2} \ d\Phi(\nu)  \]
\[ \frac{\partial s_{jt}}{\partial p_k} = \int-\frac{\exp(\delta_{jt} + \mu_{ijt})\left(\alpha\exp(\delta_{kt} + \mu_{ikt})\right)}{Z^2} \ d\Phi(\nu)  \]
\item See code.
\item See code. We were happy with the precision provided by using $3000$ draws.
\end{enumerate}
\item Ok.
\item See code.
\end{enumerate}
\paragraph{(3)} See code.
\paragraph{(4)} I am pretty happy with the variation provided by these simulated values.
\section*{4: Misspecified Models}
\paragraph{(5)} See code.
\paragraph{(6)} See code.
\paragraph{(7)}
\paragraph{(8)} We analytically compute the derivatives in the nested logit model. Let
\[ \delta_{jt} = \beta^{(1)} x_{jt} + \overline{\beta^{(2)}}satellite_j +\overline{\beta^{(3)}}wired_j + \alpha p_{jt} \]
Then
\[ s_{j/ g}(\delta_{jt}, \sigma_g) = \frac{\exp(\delta_{jt} / (1 - \sigma_g))}{\sum_{i \in \mathcal{J}_g} \exp(\delta_{it} / (1 - \sigma_g))} \]
The own-price derivative is:
\[ \frac{\partial}{\partial p_j}s_{j/ g}(\delta_{jt}, \sigma_g) = \frac{\left(\sum_{i \in \mathcal{J}_g} \exp(\delta_{it} / (1 - \sigma_g)) \right)\frac{\alpha}{1 - \sigma_g}\exp(\delta_{jt} / (1 - \sigma_g)) - \frac{\alpha}{1-\sigma_g}\left( \exp(\delta_{jt} / (1 - \sigma_g))\right)^2}{\left(\sum_{i \in \mathcal{J}_g} \exp(\delta_{it} / (1 - \sigma_g)) \right)^2} \]
\[ = \frac{\alpha}{1 - \sigma_g}s_{j/ g}(\delta_{jt}, \sigma_g)  \frac{\left(\sum_{i \in \mathcal{J}_g} \exp(\delta_{it} / (1 - \sigma_g)) \right) -\left( \exp(\delta_{jt} / (1 - \sigma_g))\right)}{\left(\sum_{i \in \mathcal{J}_g} \exp(\delta_{it} / (1 - \sigma_g)) \right)}  = \frac{\alpha}{1 - \sigma_g}s_{j/ g}(\delta_{jt}, \sigma_g) (1 - s_{j/ g}(\delta_{jt}, \sigma_g)) \]
The within-group price derivative is:
\[ \frac{\partial}{\partial p_k}s_{j/ g}(\delta_{jt}, \sigma_g) = -\frac{\alpha}{1-\sigma_g}\frac{\exp(\delta_{jt} / (1 - \sigma_g))\exp(\delta_{kt} / (1 - \sigma_g))}{\left( \sum_{i \in \mathcal{J}_g} \exp(\delta_{it} / (1 - \sigma_g))\right)^2}\]
\[ =-\frac{\alpha}{1-\sigma_g} s_{j/ g}(\delta_{t}, \sigma_g)s_{k / g}(\delta_{t}, \sigma_g) \]
The outside-group price derivative of the within-group share is 0. Let $\delta_t$ denote the vector of $\delta_jt$ for all $j$, and let $\sigma$ denote the vector of $\sigma_g$ for all $g$. The group shares are given by
\[ s_{g}(\delta_t, \sigma) = \frac{\left(\sum_{i \in \mathcal{J}_g} \exp(\delta_{it} / (1 - \sigma_g))\right)^{1-\sigma_g}}{1 + \sum_{g'}\left(\sum_{i \in \mathcal{J}_{g'}} \exp(\delta_{it} / (1 - \sigma_{g'}))\right)^{1 - \sigma_{g'}}} \]
The within-group price derivative is given by:
\[ \frac{\partial}{\partial p_j}s_{g}(\delta_t, \sigma) = \frac{(1-\sigma_g)\left(\sum_{i \in \mathcal{J}_g} \exp(\delta_{it} / (1 - \sigma_g))\right)^{-\sigma_g} \frac{\alpha}{1-\sigma_g}\exp(\delta_{jt} / (1 - \sigma_g))}{1 + \sum_{g'}\left(\sum_{i \in \mathcal{J}_{g'}} \exp(\delta_{it} / (1 - \sigma_{g'}))\right)^{1 - \sigma_{g'}}} \]
\[ - \frac{\left(\sum_{i \in \mathcal{J}_g} \exp(\delta_{it} / (1 - \sigma_g))\right)^{1-\sigma_g}(1-\sigma_g)\left(\sum_{i \in \mathcal{J}_g} \exp(\delta_{it} / (1 - \sigma_g))\right)^{-\sigma_g}\frac{\alpha}{1-\sigma_g}\exp(\delta_{jt} / (1 - \sigma_g))}{\left(1 + \sum_{g'}\left(\sum_{i \in \mathcal{J}_{g'}} \exp(\delta_{it} / (1 - \sigma_{g'}))\right)^{1 - \sigma_{g'}}\right)^2}  \]
\[ = \frac{\alpha \left(\sum_{i \in \mathcal{J}_g} \exp(\delta_{it} / (1 - \sigma_g))\right)^{-\sigma_g} \exp(\delta_{jt} / (1 - \sigma_g))}{1 + \sum_{g'}\left(\sum_{i \in \mathcal{J}_{g'}} \exp(\delta_{it} / (1 - \sigma_{g'}))\right)^{1 - \sigma_{g'}}}(1-s_g(\delta_t, \sigma)) \]
\[ = \frac{\alpha s_g(\delta_t, \sigma) \exp(\delta_{jt} / (1 - \sigma_g))}{\sum_{i \in \mathcal{J}_g} \exp(\delta_{it} / (1 - \sigma_g))}(1-s_g(\delta_t, \sigma)) \]
\[ = \alpha s_g(\delta_t, \sigma) s_{j/ g}(\delta_{jt}, \sigma_g)(1-s_g(\delta_t, \sigma)) \]
\[ = \alpha s_j(\delta_t, \sigma)(1-s_g(\delta_t, \sigma)) \]
The outside-group price derivative
\[ \frac{\partial}{\partial p_k}s_{g}(\delta_t, \sigma) =  - s_{g}(\delta_t, \sigma)\frac{\alpha \left(\sum_{i \in \mathcal{J}_{g_k}} \exp(\delta_{it} / (1 - \sigma_{g_k}))\right)^{-\sigma_{g_k}} \exp(\delta_{kt} / (1 - \sigma_{g_k}))}{1 + \sum_{g'}\left(\sum_{i \in \mathcal{J}_{g'}} \exp(\delta_{it} / (1 - \sigma_{g'}))\right)^{1 - \sigma_{g'}}} \]
\[=- s_{g}(\delta_t, \sigma)\frac{\alpha s_{g_k}(\delta_t, \sigma) \exp(\delta_{kt} / (1 - \sigma_{g_k}))}{\sum_{i \in \mathcal{J}_{g_k}} \exp(\delta_{it} / (1 - \sigma_{g_k}))} \]
\[=- \alpha s_{g}(\delta_t, \sigma)s_{g_k}(\delta_t, \sigma) s_{k/ g}(\delta_{jt}, \sigma_g) \]
\[=- \alpha s_{g}(\delta_t, \sigma)s_{k}(\delta_t, \sigma) \]
The market share function is then given by
\[ s_j(\delta_t, \sigma) = s_{g}(\delta_t, \sigma)s_{j/ g}(\delta_{jt}, \sigma_g) \]
\[ \frac{\partial }{\partial p}s_j(\delta_t, \sigma) = \frac{\partial }{\partial p}s_{g}(\delta_t, \sigma)s_{j/ g}(\delta_{jt}, \sigma_g)  + s_{g}(\delta_t, \sigma)\frac{\partial }{\partial p}s_{j/ g}(\delta_{jt}, \sigma_g)\]
The own-price derivative is then:
\[ \frac{\partial }{\partial p_j}s_j(\delta_t, \sigma) = \alpha s_j(\delta_t, \sigma)(1-s_g(\delta_t, \sigma)) s_{j/ g}(\delta_{jt}, \sigma_g)+ s_{g}(\delta_t, \sigma)\frac{\alpha}{1 - \sigma_g}s_{j/ g}(\delta_{jt}, \sigma_g) (1 - s_{j/ g}(\delta_{jt}, \sigma_g)) \]
\[= \alpha s_j(\delta_t, \sigma)(1-s_g(\delta_t, \sigma)) s_{j/ g}(\delta_{jt}, \sigma_g)+ s_{j}(\delta_t, \sigma)\frac{\alpha}{1 - \sigma_g}(1 - s_{j/ g}(\delta_{jt}, \sigma_g)) \]
\[= \alpha s_j(\delta_t, \sigma)\left( (1-s_g(\delta_t, \sigma)) s_{j/ g}(\delta_{jt}, \sigma_g)+ \frac{1}{1 - \sigma_g}(1 - s_{j/ g}(\delta_{jt}, \sigma_g))\right) \]
\[= \frac{\alpha s_j(\delta_t, \sigma)}{1-\sigma_g}\left( (1-\sigma_g)s_{j/ g}(\delta_{jt}, \sigma_g)-(1-\sigma_g)s_j(\delta_t, \sigma) + 1 - s_{j/ g}(\delta_{jt}, \sigma_g)\right) \]
\[= \frac{\alpha s_j(\delta_t, \sigma)}{1-\sigma_g}\left( 1 -\sigma_g s_{j/ g}(\delta_{jt}, \sigma_g)-(1-\sigma_g)s_j(\delta_t, \sigma)  \right) \]
The within-group price derivative is
\[ \frac{\partial }{\partial p_k}s_j(\delta_t, \sigma) = \alpha s_k(\delta_t, \sigma)(1-s_g(\delta_t, \sigma))s_{j/ g}(\delta_{jt}, \sigma_g)  - s_{g}(\delta_t, \sigma)\frac{\alpha}{1-\sigma_g} s_{j/ g}(\delta_{t}, \sigma_g)s_{k / g}(\delta_{t}, \sigma_g)\]
\[= \frac{\alpha}{1-\sigma_g} s_k(\delta_t, \sigma)\left( (1-\sigma_g)s_{j/ g}(\delta_{jt}, \sigma_g)-(1-\sigma_g)s_j(\delta_t, \sigma)  -  s_{j/ g}(\delta_{t}, \sigma_g)\right)\]
\[= - \frac{\alpha}{1-\sigma_g} s_k(\delta_t, \sigma)\left( \sigma_g s_{j/ g}(\delta_{jt}, \sigma_g)+ (1-\sigma_g)s_j(\delta_t, \sigma)  \right)\]
The outside-group price derivative is
\[\frac{\partial }{\partial p_k}s_j(\delta_t, \sigma) =- \alpha s_{j}(\delta_t, \sigma)s_{k}(\delta_t, \sigma) \]

And the outside-option derivative is
\[ \frac{\partial}{\partial p_j} s_0(\delta_t, \sigma) = - \alpha s_0(\delta_t, \sigma) s_j(\delta_t, \sigma)\]

\section*{5: Estimating the Correctly Specified Model}
(Note: some weird behavior exists because the parameters for $satellite$ and $wired$ are collinear).

\paragraph{(9)} See Tables 1 and 2. We prefer the full model estimation (due to the better estimates).

\paragraph{(10)} Let $\varepsilon_i$ denote the own-price elasticity of good $i$, and let\[ \mathcal{D}_{jk}  = - \frac{\partial s_{k} / \partial p_j}{\partial s_j / \partial p_j} \] The true and estimated matrix of own-price elasticities is in Table 3, and the true and estimated average diversion ratios are in Table 4 and Table 5, respectively.


\begin{table}
\centering
\begin{tabular}{ |c|c|c|c|c|c| }
 \hline
 \multicolumn{6}{|c|}{Parameter Estimates, Demand-side Estimation Only} \\
 \hline
 $\alpha$ & $\beta^{(1)}$ & $\overline{\beta^{(2)}}$ & $\overline{\beta^{(3)}}$ & $\sigma_2$ & $\sigma_3$ \\
 \hline
 -1.852408 & 0.9872258 & 3.615042 & 3.622520 & 1.0000 & 1.0000\\
 (0.01867589) & (0.04741592) & (0.04524844) & (0.04691815) & (0.3071605) & (0.3172754) \\
 \hline
\end{tabular}
\label{table5_1}
\caption{Parameter Estimates, Demand-side Estimation Only}
\end{table}



\begin{table}
\centering
\begin{tabular}{ |c|c|c|c|c|c|c|c| }
 \hline
 \multicolumn{8}{|c|}{Parameter Estimates, Full Model Estimation} \\
 \hline
 $\alpha$ & $\beta^{(1)}$ & $\overline{\beta^{(2)}}$ & $\overline{\beta^{(3)}}$ & $\sigma_2$ & $\sigma_3$ & $\gamma_0$ & $\gamma_1$ \\
 \hline
 -2.0347 & 1.0568 & 4.0361 & 4.0444 & 1.1782 & 1.1932 & 0.49112 & 0.25381 \\
(0.0858) & (0.0454) & (0.2111) & (0.2131) & (0.2196) & (0.2108) & (0.01772) & (0.00912)\\
 \hline
\end{tabular}
\label{table5_2}
\caption{Parameter Estimates, Full Model Estimation}
\end{table}



\begin{table}
\centering
\begin{tabular}{ |c|c|c|c|c| }
\hline
 & $\varepsilon_1$ &   $\varepsilon_2$ &  $\varepsilon_3$ &  $\varepsilon_4$ \\
 \hline
 True Values & -4.06535006 & -4.16553436 & -4.17726162 & -4.18978309  \\
 \hline
Estimated Values & -4.0525488 & -4.15853012 & -4.16252984 & -4.17736646 \\
 \hline
\end{tabular}
\label{table5_3}
\caption{Average Own-Price Elasticities, Full Model Estimation}
\end{table}



\begin{table}
\centering
\begin{tabular}{ |cccc| }
 \hline
\multicolumn{4}{|c|}{$\mathcal{D}$: diagonal entries $\mathcal{D}_{jj}$ replaced with $\mathcal{D}_{j0}$}
 \\
 \hline
 0.33115087 & 0.30335128 & 0.18522023 & 0.18027762   \\
0.32317153 & 0.32122579 & 0.18063565 & 0.17496703  \\
0.19329289 & 0.17575241 & 0.32765373 & 0.30330097  \\
0.19192008 & 0.17341037 & 0.31037504 & 0.32429451  \\
 \hline
\end{tabular}
\label{table5_4}
\caption{True Average Diversion Ratio Matrix}
\end{table}



\begin{table}
\centering
\begin{tabular}{ |cccc| }
 \hline
\multicolumn{4}{|c|}{$\mathcal{D}$: diagonal entries $\mathcal{D}_{jj}$ replaced with $\mathcal{D}_{j0}$}
 \\
 \hline
 0.32908096 & 0.32674409 & 0.1743611 & 0.16981386   \\
0.34688774 & 0.31879102 & 0.16981928 & 0.16450196  \\
0.18220138 & 0.16573254 & 0.32424023 & 0.32782585  \\
0.18083009 & 0.16332965 & 0.33517888 & 0.32066138  \\
 \hline
\end{tabular}
\label{table5_4}
\caption{Estimated Average Diversion Ratio Matrix}
\end{table}


\section*{6: Merger Simulation}
\paragraph{(11)} When two of the firms merge, prices will generically increase for the merged firm's goods. The firms all increase prices because the merged firm can price its own goods closer to monopoly pricing.
\paragraph{(12)} See code.
\paragraph{(13)} See Table 6.1. Intuitively, it makes sense that the merger of 1 and 2 results in larger price increases than 1 and 3; this is because merging 1 and 2 means the merged firm produces the only satellite products, and hence has a stronger incentive to raise prices of the satellite TV services.
\paragraph{(14)} A reduction in marginal cost means that prices may not necessarily increase as a result of the merger, and hence can potentially improve efficiency; the merged firm can earn more profits to outweigh any consumer welfare decrease.
\paragraph{(15)} See Table 6.1 for the post-merger prices with cost reduction. The net consumer welfare actually decreases as a result of the merger by $6.8384$. However, the firm manages to earn significantly more profits: specifically, the firm earns $69.3230$ more in profits. Hence the overall predicted welfare change is $62.4846$. We need to assume the markets have uniform measure of consumers here because previously all the computations were performed using in-market shares, which has no reliance on the size of the market. For net consumer welfare and profits, we have to aggregate across markets, and hence we need assumptions on the measure of consumers in each market.

\begin{table}
\centering
\begin{tabular}{ |c|c|c|c|c|}
 \hline
\multicolumn{5}{|c|}{Table 6.1: Average Prices across Markets, Merger Analysis}
 \\
 \hline
 & $p_1$ & $p_2$ & $p_3$ & $p_4$ \\
 \hline
 Pre-Merger & 2.7327 & 2.7165  & 2.7608 & 2.7391 \\
 \hline
 Merging 1 and 2 & 2.9808 & 2.9949 & 2.7712 & 2.7488 \\
 \hline
 Merging 1 and 3 & 2.8464 & 2.7285 & 2.8826 & 2.7514\\
 \hline
 Merging 1 and 2, with cost decrease & 2.7833 & 2.7954 & 2.7612 & 2.7391\\
 \hline
\end{tabular}
\label{table6_1}
\caption{Average Prices across Markets, Merger Analysis}
\end{table}


\section*{Appendix: Code}


    \begin{tcolorbox}[breakable, size=fbox, boxrule=1pt, pad at break*=1mm,colback=cellbackground, colframe=cellborder]
\prompt{In}{incolor}{1}{\boxspacing}
\begin{Verbatim}[commandchars=\\\{\}]
\PY{k+kn}{import} \PY{n+nn}{numpy} \PY{k}{as} \PY{n+nn}{np}
\PY{k+kn}{from} \PY{n+nn}{scipy}\PY{n+nn}{.}\PY{n+nn}{optimize} \PY{k}{import} \PY{n}{fsolve}\PY{p}{,} \PY{n}{fixed\PYZus{}point}
\PY{k+kn}{from} \PY{n+nn}{matplotlib} \PY{k}{import} \PY{n}{pyplot} \PY{k}{as} \PY{n}{plt}
\PY{k+kn}{import} \PY{n+nn}{pyblp}
\PY{k+kn}{from} \PY{n+nn}{tqdm}\PY{n+nn}{.}\PY{n+nn}{notebook} \PY{k}{import} \PY{n}{trange}
\PY{k+kn}{import} \PY{n+nn}{statsmodels}\PY{n+nn}{.}\PY{n+nn}{api} \PY{k}{as} \PY{n+nn}{sm}
\PY{k+kn}{import} \PY{n+nn}{statsmodels}\PY{n+nn}{.}\PY{n+nn}{formula}\PY{n+nn}{.}\PY{n+nn}{api} \PY{k}{as} \PY{n+nn}{smf}
\PY{k+kn}{from} \PY{n+nn}{linearmodels}\PY{n+nn}{.}\PY{n+nn}{iv} \PY{k}{import} \PY{n}{IV2SLS}
\PY{k+kn}{import} \PY{n+nn}{pandas} \PY{k}{as} \PY{n+nn}{pd}
\end{Verbatim}
\end{tcolorbox}

    \begin{tcolorbox}[breakable, size=fbox, boxrule=1pt, pad at break*=1mm,colback=cellbackground, colframe=cellborder]
\prompt{In}{incolor}{2}{\boxspacing}
\begin{Verbatim}[commandchars=\\\{\}]
\PY{n}{RNG\PYZus{}SEED} \PY{o}{=} \PY{l+m+mi}{8476263}

\PY{n}{rng} \PY{o}{=} \PY{n}{np}\PY{o}{.}\PY{n}{random}\PY{o}{.}\PY{n}{default\PYZus{}rng}\PY{p}{(}\PY{n}{RNG\PYZus{}SEED}\PY{p}{)} \PY{c+c1}{\PYZsh{} this random seeding is for reproducibility}
\end{Verbatim}
\end{tcolorbox}

    \begin{tcolorbox}[breakable, size=fbox, boxrule=1pt, pad at break*=1mm,colback=cellbackground, colframe=cellborder]
\prompt{In}{incolor}{3}{\boxspacing}
\begin{Verbatim}[commandchars=\\\{\}]
\PY{c+c1}{\PYZsh{} I am horrified that we have to overrun the default collinearity checks}
\PY{c+c1}{\PYZsh{} however, wired and satellite dummy variables are collinear}
\PY{c+c1}{\PYZsh{} so to prevent PyBLP from throwing a fit, we must do this.}
\PY{n}{pyblp}\PY{o}{.}\PY{n}{options}\PY{o}{.}\PY{n}{collinear\PYZus{}rtol} \PY{o}{=} \PY{l+m+mi}{0}
\PY{n}{pyblp}\PY{o}{.}\PY{n}{options}\PY{o}{.}\PY{n}{collinear\PYZus{}atol} \PY{o}{=} \PY{l+m+mi}{0}
\end{Verbatim}
\end{tcolorbox}

    \begin{tcolorbox}[breakable, size=fbox, boxrule=1pt, pad at break*=1mm,colback=cellbackground, colframe=cellborder]
\prompt{In}{incolor}{4}{\boxspacing}
\begin{Verbatim}[commandchars=\\\{\}]
\PY{c+c1}{\PYZsh{} fixed parameter definitions}

\PY{n}{beta1} \PY{o}{=} \PY{l+m+mi}{1}
\PY{n}{alpha} \PY{o}{=} \PY{o}{\PYZhy{}}\PY{l+m+mi}{2}
\PY{n}{gamma0} \PY{o}{=} \PY{l+m+mi}{1}\PY{o}{/}\PY{l+m+mi}{2}
\PY{n}{gamma1} \PY{o}{=} \PY{l+m+mi}{1}\PY{o}{/}\PY{l+m+mi}{4}
\PY{n}{beta2\PYZus{}bar} \PY{o}{=} \PY{l+m+mi}{4}
\PY{n}{beta3\PYZus{}bar} \PY{o}{=} \PY{l+m+mi}{4}
\PY{n}{sigma2} \PY{o}{=} \PY{l+m+mi}{1}
\PY{n}{sigma3} \PY{o}{=} \PY{l+m+mi}{1}

\PY{c+c1}{\PYZsh{} markets and goods}
\PY{n}{T} \PY{o}{=} \PY{l+m+mi}{600}
\PY{n}{J} \PY{o}{=} \PY{l+m+mi}{4}
\end{Verbatim}
\end{tcolorbox}

    \begin{tcolorbox}[breakable, size=fbox, boxrule=1pt, pad at break*=1mm,colback=cellbackground, colframe=cellborder]
\prompt{In}{incolor}{5}{\boxspacing}
\begin{Verbatim}[commandchars=\\\{\}]
\PY{c+c1}{\PYZsh{} 3.1}

\PY{c+c1}{\PYZsh{} x\PYZus{}jt, w\PYZus{}jt are absolute value of iid standard normal variables}
\PY{n}{x} \PY{o}{=} \PY{n}{np}\PY{o}{.}\PY{n}{absolute}\PY{p}{(}\PY{n}{rng}\PY{o}{.}\PY{n}{standard\PYZus{}normal}\PY{p}{(}\PY{n}{size}\PY{o}{=}\PY{p}{(}\PY{n}{J}\PY{p}{,}\PY{n}{T}\PY{p}{)}\PY{p}{)}\PY{p}{)}
\PY{n}{w} \PY{o}{=} \PY{n}{np}\PY{o}{.}\PY{n}{absolute}\PY{p}{(}\PY{n}{rng}\PY{o}{.}\PY{n}{standard\PYZus{}normal}\PY{p}{(}\PY{n}{size}\PY{o}{=}\PY{p}{(}\PY{n}{J}\PY{p}{,}\PY{n}{T}\PY{p}{)}\PY{p}{)}\PY{p}{)}

\PY{n}{unobservable\PYZus{}mean} \PY{o}{=} \PY{p}{[}\PY{l+m+mi}{0}\PY{p}{,}\PY{l+m+mi}{0}\PY{p}{]}
\PY{n}{unobservable\PYZus{}cov} \PY{o}{=} \PY{p}{[}\PY{p}{[}\PY{l+m+mi}{1}\PY{p}{,}\PY{l+m+mf}{0.25}\PY{p}{]}\PY{p}{,}\PY{p}{[}\PY{l+m+mf}{0.25}\PY{p}{,}\PY{l+m+mi}{1}\PY{p}{]}\PY{p}{]}
\PY{n}{unobservables} \PY{o}{=} \PY{n}{rng}\PY{o}{.}\PY{n}{multivariate\PYZus{}normal}\PY{p}{(}\PY{n}{unobservable\PYZus{}mean}\PY{p}{,} \PY{n}{unobservable\PYZus{}cov}\PY{p}{,} \PY{n}{size}\PY{o}{=}\PY{p}{(}\PY{n}{J}\PY{p}{,}\PY{n}{T}\PY{p}{)}\PY{p}{)}
\PY{n}{xi} \PY{o}{=} \PY{n}{unobservables}\PY{p}{[}\PY{p}{:}\PY{p}{,}\PY{p}{:}\PY{p}{,}\PY{l+m+mi}{0}\PY{p}{]}
\PY{n}{omega} \PY{o}{=} \PY{n}{unobservables}\PY{p}{[}\PY{p}{:}\PY{p}{,}\PY{p}{:}\PY{p}{,}\PY{l+m+mi}{1}\PY{p}{]}
\end{Verbatim}
\end{tcolorbox}

    \begin{tcolorbox}[breakable, size=fbox, boxrule=1pt, pad at break*=1mm,colback=cellbackground, colframe=cellborder]
\prompt{In}{incolor}{6}{\boxspacing}
\begin{Verbatim}[commandchars=\\\{\}]
\PY{c+c1}{\PYZsh{} 3.2a}
\PY{c+c1}{\PYZsh{} defining the market share}

\PY{k}{def} \PY{n+nf}{own\PYZus{}mkt\PYZus{}share\PYZus{}derivative}\PY{p}{(}\PY{n}{t}\PY{p}{,} \PY{n}{p}\PY{p}{,} \PY{n}{beta2}\PY{p}{,} \PY{n}{beta3}\PY{p}{)}\PY{p}{:}
    \PY{c+c1}{\PYZsh{} p should be a length J vector}
    \PY{c+c1}{\PYZsh{} betas should be num\PYZus{}sims}

    \PY{n}{u\PYZus{}t} \PY{o}{=} \PY{n}{np}\PY{o}{.}\PY{n}{tile}\PY{p}{(}\PY{n}{x}\PY{p}{[}\PY{p}{:}\PY{p}{,}\PY{n}{t}\PY{p}{]} \PY{o}{+} \PY{n}{xi}\PY{p}{[}\PY{p}{:}\PY{p}{,}\PY{n}{t}\PY{p}{]} \PY{o}{+} \PY{n}{alpha}\PY{o}{*}\PY{n}{p}\PY{p}{,} \PY{p}{(}\PY{n+nb}{len}\PY{p}{(}\PY{n}{beta2}\PY{p}{)}\PY{p}{,} \PY{l+m+mi}{1}\PY{p}{)}\PY{p}{)} \PY{c+c1}{\PYZsh{} num\PYZus{}sims x J }
    \PY{k}{for} \PY{n}{j} \PY{o+ow}{in} \PY{n+nb}{range}\PY{p}{(}\PY{n}{J}\PY{p}{)}\PY{p}{:}
        \PY{k}{if} \PY{n}{j} \PY{o}{\PYZlt{}} \PY{l+m+mi}{2}\PY{p}{:}
            \PY{n}{u\PYZus{}t}\PY{p}{[}\PY{p}{:}\PY{p}{,}\PY{n}{j}\PY{p}{]} \PY{o}{=} \PY{n}{u\PYZus{}t}\PY{p}{[}\PY{p}{:}\PY{p}{,}\PY{n}{j}\PY{p}{]} \PY{o}{+} \PY{n}{beta2}
        \PY{k}{else}\PY{p}{:}
            \PY{n}{u\PYZus{}t}\PY{p}{[}\PY{p}{:}\PY{p}{,}\PY{n}{j}\PY{p}{]} \PY{o}{=} \PY{n}{u\PYZus{}t}\PY{p}{[}\PY{p}{:}\PY{p}{,}\PY{n}{j}\PY{p}{]} \PY{o}{+} \PY{n}{beta3}

    \PY{n}{Z} \PY{o}{=} \PY{n}{np}\PY{o}{.}\PY{n}{tile}\PY{p}{(} \PY{l+m+mi}{1} \PY{o}{+} \PY{n}{np}\PY{o}{.}\PY{n}{sum}\PY{p}{(}\PY{n}{np}\PY{o}{.}\PY{n}{exp}\PY{p}{(}\PY{n}{u\PYZus{}t}\PY{p}{)}\PY{p}{,}\PY{n}{axis}\PY{o}{=}\PY{o}{\PYZhy{}}\PY{l+m+mi}{1}\PY{p}{)}\PY{p}{,} \PY{p}{(}\PY{n}{J}\PY{p}{,}\PY{l+m+mi}{1}\PY{p}{)}\PY{p}{)}\PY{o}{.}\PY{n}{T}
    \PY{n}{numerator} \PY{o}{=} \PY{n}{alpha}\PY{o}{*}\PY{n}{np}\PY{o}{.}\PY{n}{exp}\PY{p}{(}\PY{n}{u\PYZus{}t}\PY{p}{)}\PY{o}{*}\PY{n}{Z} \PY{o}{\PYZhy{}} \PY{n}{alpha}\PY{o}{*}\PY{n}{np}\PY{o}{.}\PY{n}{square}\PY{p}{(}\PY{n}{np}\PY{o}{.}\PY{n}{exp}\PY{p}{(}\PY{n}{u\PYZus{}t}\PY{p}{)}\PY{p}{)} \PY{c+c1}{\PYZsh{} num\PYZus{}sims x J}
    \PY{n}{denominator} \PY{o}{=} \PY{n}{np}\PY{o}{.}\PY{n}{square}\PY{p}{(}\PY{n}{Z}\PY{p}{)}

    \PY{k}{return} \PY{n}{np}\PY{o}{.}\PY{n}{mean}\PY{p}{(}\PY{n}{numerator} \PY{o}{/} \PY{n}{denominator}\PY{p}{,} \PY{n}{axis}\PY{o}{=}\PY{l+m+mi}{0}\PY{p}{)}

\PY{k}{def} \PY{n+nf}{outside\PYZus{}mkt\PYZus{}share\PYZus{}derivative}\PY{p}{(}\PY{n}{t}\PY{p}{,} \PY{n}{p}\PY{p}{,} \PY{n}{beta2}\PY{p}{,} \PY{n}{beta3}\PY{p}{)}\PY{p}{:}
    \PY{c+c1}{\PYZsh{} p should be a length J vector}
    \PY{c+c1}{\PYZsh{} betas should be num\PYZus{}sims}

    \PY{n}{u\PYZus{}t} \PY{o}{=} \PY{n}{np}\PY{o}{.}\PY{n}{tile}\PY{p}{(}\PY{n}{x}\PY{p}{[}\PY{p}{:}\PY{p}{,}\PY{n}{t}\PY{p}{]} \PY{o}{+} \PY{n}{xi}\PY{p}{[}\PY{p}{:}\PY{p}{,}\PY{n}{t}\PY{p}{]} \PY{o}{+} \PY{n}{alpha}\PY{o}{*}\PY{n}{p}\PY{p}{,} \PY{p}{(}\PY{n+nb}{len}\PY{p}{(}\PY{n}{beta2}\PY{p}{)}\PY{p}{,} \PY{l+m+mi}{1}\PY{p}{)}\PY{p}{)} \PY{c+c1}{\PYZsh{} num\PYZus{}sims x J }
    \PY{k}{for} \PY{n}{j} \PY{o+ow}{in} \PY{n+nb}{range}\PY{p}{(}\PY{n}{J}\PY{p}{)}\PY{p}{:}
        \PY{k}{if} \PY{n}{j} \PY{o}{\PYZlt{}} \PY{l+m+mi}{2}\PY{p}{:}
            \PY{n}{u\PYZus{}t}\PY{p}{[}\PY{p}{:}\PY{p}{,}\PY{n}{j}\PY{p}{]} \PY{o}{=} \PY{n}{u\PYZus{}t}\PY{p}{[}\PY{p}{:}\PY{p}{,}\PY{n}{j}\PY{p}{]} \PY{o}{+} \PY{n}{beta2}
        \PY{k}{else}\PY{p}{:}
            \PY{n}{u\PYZus{}t}\PY{p}{[}\PY{p}{:}\PY{p}{,}\PY{n}{j}\PY{p}{]} \PY{o}{=} \PY{n}{u\PYZus{}t}\PY{p}{[}\PY{p}{:}\PY{p}{,}\PY{n}{j}\PY{p}{]} \PY{o}{+} \PY{n}{beta3}

    \PY{n}{Z} \PY{o}{=} \PY{n}{np}\PY{o}{.}\PY{n}{tile}\PY{p}{(} \PY{l+m+mi}{1} \PY{o}{+} \PY{n}{np}\PY{o}{.}\PY{n}{sum}\PY{p}{(}\PY{n}{np}\PY{o}{.}\PY{n}{exp}\PY{p}{(}\PY{n}{u\PYZus{}t}\PY{p}{)}\PY{p}{,}\PY{n}{axis}\PY{o}{=}\PY{o}{\PYZhy{}}\PY{l+m+mi}{1}\PY{p}{)}\PY{p}{,} \PY{p}{(}\PY{n}{J}\PY{p}{,}\PY{l+m+mi}{1}\PY{p}{)}\PY{p}{)}\PY{o}{.}\PY{n}{T}
    \PY{n}{numerator} \PY{o}{=} \PY{o}{\PYZhy{}}\PY{l+m+mi}{1}\PY{o}{*}\PY{n}{alpha}\PY{o}{*}\PY{n}{np}\PY{o}{.}\PY{n}{exp}\PY{p}{(}\PY{n}{u\PYZus{}t}\PY{p}{)} \PY{c+c1}{\PYZsh{} num\PYZus{}sims x J}
    \PY{n}{denominator} \PY{o}{=} \PY{n}{np}\PY{o}{.}\PY{n}{square}\PY{p}{(}\PY{n}{Z}\PY{p}{)}

    \PY{k}{return} \PY{n}{np}\PY{o}{.}\PY{n}{mean}\PY{p}{(}\PY{n}{numerator} \PY{o}{/} \PY{n}{denominator}\PY{p}{,} \PY{n}{axis}\PY{o}{=}\PY{l+m+mi}{0}\PY{p}{)}

\PY{k}{def} \PY{n+nf}{full\PYZus{}mkt\PYZus{}share\PYZus{}derivative}\PY{p}{(}\PY{n}{t}\PY{p}{,} \PY{n}{p}\PY{p}{,} \PY{n}{beta2}\PY{p}{,} \PY{n}{beta3}\PY{p}{)}\PY{p}{:}
    \PY{c+c1}{\PYZsh{} p should be a length J vector}
    \PY{c+c1}{\PYZsh{} betas should be num\PYZus{}sims}

    \PY{n}{u\PYZus{}t} \PY{o}{=} \PY{n}{np}\PY{o}{.}\PY{n}{tile}\PY{p}{(}\PY{n}{x}\PY{p}{[}\PY{p}{:}\PY{p}{,}\PY{n}{t}\PY{p}{]} \PY{o}{+} \PY{n}{xi}\PY{p}{[}\PY{p}{:}\PY{p}{,}\PY{n}{t}\PY{p}{]} \PY{o}{+} \PY{n}{alpha}\PY{o}{*}\PY{n}{p}\PY{p}{,} \PY{p}{(}\PY{n+nb}{len}\PY{p}{(}\PY{n}{beta2}\PY{p}{)}\PY{p}{,} \PY{l+m+mi}{1}\PY{p}{)}\PY{p}{)} \PY{c+c1}{\PYZsh{} num\PYZus{}sims x J }
    \PY{k}{for} \PY{n}{j} \PY{o+ow}{in} \PY{n+nb}{range}\PY{p}{(}\PY{n}{J}\PY{p}{)}\PY{p}{:}
        \PY{k}{if} \PY{n}{j} \PY{o}{\PYZlt{}} \PY{l+m+mi}{2}\PY{p}{:}
            \PY{n}{u\PYZus{}t}\PY{p}{[}\PY{p}{:}\PY{p}{,}\PY{n}{j}\PY{p}{]} \PY{o}{=} \PY{n}{u\PYZus{}t}\PY{p}{[}\PY{p}{:}\PY{p}{,}\PY{n}{j}\PY{p}{]} \PY{o}{+} \PY{n}{beta2}
        \PY{k}{else}\PY{p}{:}
            \PY{n}{u\PYZus{}t}\PY{p}{[}\PY{p}{:}\PY{p}{,}\PY{n}{j}\PY{p}{]} \PY{o}{=} \PY{n}{u\PYZus{}t}\PY{p}{[}\PY{p}{:}\PY{p}{,}\PY{n}{j}\PY{p}{]} \PY{o}{+} \PY{n}{beta3}

    \PY{n}{Z} \PY{o}{=} \PY{n}{np}\PY{o}{.}\PY{n}{tile}\PY{p}{(} \PY{l+m+mi}{1} \PY{o}{+} \PY{n}{np}\PY{o}{.}\PY{n}{sum}\PY{p}{(}\PY{n}{np}\PY{o}{.}\PY{n}{exp}\PY{p}{(}\PY{n}{u\PYZus{}t}\PY{p}{)}\PY{p}{,}\PY{n}{axis}\PY{o}{=}\PY{o}{\PYZhy{}}\PY{l+m+mi}{1}\PY{p}{)}\PY{p}{,} \PY{p}{(}\PY{n}{J}\PY{p}{,}\PY{l+m+mi}{1}\PY{p}{)}\PY{p}{)}\PY{o}{.}\PY{n}{T} \PY{c+c1}{\PYZsh{} num\PYZus{}sims x J}

    \PY{n}{derivatives} \PY{o}{=} \PY{n}{np}\PY{o}{.}\PY{n}{zeros}\PY{p}{(}\PY{p}{(}\PY{n}{J}\PY{p}{,}\PY{n}{J}\PY{p}{)}\PY{p}{)}

    \PY{n}{own\PYZus{}numerator} \PY{o}{=} \PY{n}{alpha}\PY{o}{*}\PY{n}{np}\PY{o}{.}\PY{n}{exp}\PY{p}{(}\PY{n}{u\PYZus{}t}\PY{p}{)}\PY{o}{*}\PY{n}{Z} \PY{o}{\PYZhy{}} \PY{n}{alpha}\PY{o}{*}\PY{n}{np}\PY{o}{.}\PY{n}{square}\PY{p}{(}\PY{n}{np}\PY{o}{.}\PY{n}{exp}\PY{p}{(}\PY{n}{u\PYZus{}t}\PY{p}{)}\PY{p}{)} \PY{c+c1}{\PYZsh{} num\PYZus{}sims x J}
    \PY{n}{denominator} \PY{o}{=} \PY{n}{np}\PY{o}{.}\PY{n}{square}\PY{p}{(}\PY{n}{Z}\PY{p}{)}

    \PY{k}{for} \PY{n}{j} \PY{o+ow}{in} \PY{n+nb}{range}\PY{p}{(}\PY{n}{J}\PY{p}{)}\PY{p}{:}
        \PY{n}{derivatives}\PY{p}{[}\PY{n}{j}\PY{p}{,}\PY{n}{j}\PY{p}{]} \PY{o}{=} \PY{n}{np}\PY{o}{.}\PY{n}{mean}\PY{p}{(}\PY{n}{own\PYZus{}numerator} \PY{o}{/} \PY{n}{denominator}\PY{p}{,} \PY{n}{axis}\PY{o}{=}\PY{l+m+mi}{0}\PY{p}{)}\PY{p}{[}\PY{n}{j}\PY{p}{]}

    \PY{k}{for} \PY{n}{j} \PY{o+ow}{in} \PY{n+nb}{range}\PY{p}{(}\PY{n}{J}\PY{p}{)}\PY{p}{:}
        \PY{k}{for} \PY{n}{k} \PY{o+ow}{in} \PY{n+nb}{range}\PY{p}{(}\PY{n}{J}\PY{p}{)}\PY{p}{:}
            \PY{k}{if} \PY{o+ow}{not} \PY{p}{(}\PY{n}{j} \PY{o}{==} \PY{n}{k}\PY{p}{)}\PY{p}{:}
                \PY{n}{derivatives}\PY{p}{[}\PY{n}{j}\PY{p}{,}\PY{n}{k}\PY{p}{]} \PY{o}{=} \PY{n}{np}\PY{o}{.}\PY{n}{mean}\PY{p}{(}\PY{o}{\PYZhy{}}\PY{l+m+mi}{1}\PY{o}{*}\PY{n}{alpha}\PY{o}{*}\PY{n}{np}\PY{o}{.}\PY{n}{exp}\PY{p}{(}\PY{n}{u\PYZus{}t}\PY{p}{)}\PY{p}{[}\PY{p}{:}\PY{p}{,}\PY{n}{k}\PY{p}{]}\PY{o}{*}\PY{n}{np}\PY{o}{.}\PY{n}{exp}\PY{p}{(}\PY{n}{u\PYZus{}t}\PY{p}{)}\PY{p}{[}\PY{p}{:}\PY{p}{,}\PY{n}{j}\PY{p}{]} \PY{o}{/} \PY{n}{np}\PY{o}{.}\PY{n}{square}\PY{p}{(}\PY{l+m+mi}{1} \PY{o}{+} \PY{n}{np}\PY{o}{.}\PY{n}{sum}\PY{p}{(}\PY{n}{np}\PY{o}{.}\PY{n}{exp}\PY{p}{(}\PY{n}{u\PYZus{}t}\PY{p}{)}\PY{p}{,}\PY{n}{axis}\PY{o}{=}\PY{o}{\PYZhy{}}\PY{l+m+mi}{1}\PY{p}{)}\PY{p}{)}\PY{p}{)}

    \PY{k}{return} \PY{n}{derivatives}
\end{Verbatim}
\end{tcolorbox}

    \begin{tcolorbox}[breakable, size=fbox, boxrule=1pt, pad at break*=1mm,colback=cellbackground, colframe=cellborder]
\prompt{In}{incolor}{7}{\boxspacing}
\begin{Verbatim}[commandchars=\\\{\}]
\PY{c+c1}{\PYZsh{} s\PYZus{}jt(p) }
\PY{k}{def} \PY{n+nf}{mkt\PYZus{}share}\PY{p}{(}\PY{n}{t}\PY{p}{,} \PY{n}{p}\PY{p}{,} \PY{n}{beta2}\PY{p}{,} \PY{n}{beta3}\PY{p}{)}\PY{p}{:}
    \PY{c+c1}{\PYZsh{} p should be a length J vector}
    \PY{c+c1}{\PYZsh{} betas should be num\PYZus{}sims}

    \PY{n}{u\PYZus{}t} \PY{o}{=} \PY{n}{np}\PY{o}{.}\PY{n}{tile}\PY{p}{(}\PY{n}{x}\PY{p}{[}\PY{p}{:}\PY{p}{,}\PY{n}{t}\PY{p}{]} \PY{o}{+} \PY{n}{xi}\PY{p}{[}\PY{p}{:}\PY{p}{,}\PY{n}{t}\PY{p}{]} \PY{o}{+} \PY{n}{alpha}\PY{o}{*}\PY{n}{p}\PY{p}{,} \PY{p}{(}\PY{n+nb}{len}\PY{p}{(}\PY{n}{beta2}\PY{p}{)}\PY{p}{,} \PY{l+m+mi}{1}\PY{p}{)}\PY{p}{)} \PY{c+c1}{\PYZsh{} num\PYZus{}sims x J }
    \PY{k}{for} \PY{n}{j} \PY{o+ow}{in} \PY{n+nb}{range}\PY{p}{(}\PY{n}{J}\PY{p}{)}\PY{p}{:}
        \PY{k}{if} \PY{n}{j} \PY{o}{\PYZlt{}} \PY{l+m+mi}{2}\PY{p}{:}
            \PY{n}{u\PYZus{}t}\PY{p}{[}\PY{p}{:}\PY{p}{,}\PY{n}{j}\PY{p}{]} \PY{o}{=} \PY{n}{u\PYZus{}t}\PY{p}{[}\PY{p}{:}\PY{p}{,}\PY{n}{j}\PY{p}{]} \PY{o}{+} \PY{n}{beta2}
        \PY{k}{else}\PY{p}{:}
            \PY{n}{u\PYZus{}t}\PY{p}{[}\PY{p}{:}\PY{p}{,}\PY{n}{j}\PY{p}{]} \PY{o}{=} \PY{n}{u\PYZus{}t}\PY{p}{[}\PY{p}{:}\PY{p}{,}\PY{n}{j}\PY{p}{]} \PY{o}{+} \PY{n}{beta3}

    \PY{n}{numerator} \PY{o}{=} \PY{n}{np}\PY{o}{.}\PY{n}{exp}\PY{p}{(}\PY{n}{u\PYZus{}t}\PY{p}{)}
    \PY{n}{denominator} \PY{o}{=} \PY{l+m+mi}{1} \PY{o}{+} \PY{n}{np}\PY{o}{.}\PY{n}{sum}\PY{p}{(}\PY{n}{np}\PY{o}{.}\PY{n}{exp}\PY{p}{(}\PY{n}{u\PYZus{}t}\PY{p}{)}\PY{p}{,}\PY{n}{axis}\PY{o}{=}\PY{o}{\PYZhy{}}\PY{l+m+mi}{1}\PY{p}{)} \PY{c+c1}{\PYZsh{} num\PYZus{}sims}

    \PY{k}{return} \PY{n}{np}\PY{o}{.}\PY{n}{mean}\PY{p}{(}\PY{n}{numerator} \PY{o}{/} \PY{p}{(}\PY{n}{np}\PY{o}{.}\PY{n}{tile}\PY{p}{(}\PY{n}{denominator}\PY{p}{,} \PY{p}{(}\PY{n}{J}\PY{p}{,} \PY{l+m+mi}{1}\PY{p}{)}\PY{p}{)}\PY{o}{.}\PY{n}{T}\PY{p}{)}\PY{p}{,} \PY{n}{axis}\PY{o}{=}\PY{l+m+mi}{0}\PY{p}{)}
\end{Verbatim}
\end{tcolorbox}

    \begin{tcolorbox}[breakable, size=fbox, boxrule=1pt, pad at break*=1mm,colback=cellbackground, colframe=cellborder]
\prompt{In}{incolor}{8}{\boxspacing}
\begin{Verbatim}[commandchars=\\\{\}]
\PY{c+c1}{\PYZsh{} 3.2a(iv)}
\PY{c+c1}{\PYZsh{} draw beta coefficients for N individuals S times, observe variation in market share derivatives}

\PY{n}{S} \PY{o}{=} \PY{l+m+mi}{100}

\PY{n}{all\PYZus{}derivatives} \PY{o}{=} \PY{n}{np}\PY{o}{.}\PY{n}{zeros}\PY{p}{(}\PY{p}{(}\PY{n}{J}\PY{p}{,}\PY{n}{J}\PY{p}{,}\PY{n}{S}\PY{p}{)}\PY{p}{)}
\PY{n}{all\PYZus{}shares} \PY{o}{=} \PY{n}{np}\PY{o}{.}\PY{n}{zeros}\PY{p}{(}\PY{p}{(}\PY{n}{J}\PY{p}{,}\PY{n}{S}\PY{p}{)}\PY{p}{)}

\PY{n}{N} \PY{o}{=} \PY{l+m+mi}{3000}

\PY{n}{price} \PY{o}{=} \PY{n}{np}\PY{o}{.}\PY{n}{array}\PY{p}{(}\PY{p}{[}\PY{l+m+mi}{1}\PY{p}{,}\PY{l+m+mi}{1}\PY{p}{,}\PY{l+m+mi}{1}\PY{p}{,}\PY{l+m+mi}{1}\PY{p}{]}\PY{p}{)}

\PY{k}{for} \PY{n}{s} \PY{o+ow}{in} \PY{n}{trange}\PY{p}{(}\PY{n}{S}\PY{p}{)}\PY{p}{:}
    \PY{n}{beta2} \PY{o}{=} \PY{n}{rng}\PY{o}{.}\PY{n}{normal}\PY{p}{(}\PY{n}{beta2\PYZus{}bar}\PY{p}{,} \PY{n}{sigma2}\PY{p}{,} \PY{n}{N}\PY{p}{)}
    \PY{n}{beta3} \PY{o}{=} \PY{n}{rng}\PY{o}{.}\PY{n}{normal}\PY{p}{(}\PY{n}{beta3\PYZus{}bar}\PY{p}{,} \PY{n}{sigma3}\PY{p}{,} \PY{n}{N}\PY{p}{)}
    \PY{n}{all\PYZus{}derivatives}\PY{p}{[}\PY{p}{:}\PY{p}{,}\PY{p}{:}\PY{p}{,}\PY{n}{s}\PY{p}{]} \PY{o}{=} \PY{n}{full\PYZus{}mkt\PYZus{}share\PYZus{}derivative}\PY{p}{(}\PY{l+m+mi}{0}\PY{p}{,} \PY{n}{price}\PY{p}{,} \PY{n}{beta2}\PY{p}{,} \PY{n}{beta3}\PY{p}{)}
    \PY{n}{all\PYZus{}shares}\PY{p}{[}\PY{p}{:}\PY{p}{,}\PY{n}{s}\PY{p}{]} \PY{o}{=} \PY{n}{mkt\PYZus{}share}\PY{p}{(}\PY{l+m+mi}{1}\PY{p}{,} \PY{n}{price}\PY{p}{,} \PY{n}{beta2}\PY{p}{,} \PY{n}{beta3}\PY{p}{)}
\PY{p}{(}\PY{n}{np}\PY{o}{.}\PY{n}{mean}\PY{p}{(}\PY{n}{all\PYZus{}shares}\PY{p}{,}\PY{n}{axis}\PY{o}{=}\PY{l+m+mi}{1}\PY{p}{)}\PY{p}{,} \PY{n}{np}\PY{o}{.}\PY{n}{std}\PY{p}{(}\PY{n}{all\PYZus{}shares}\PY{p}{,}\PY{n}{axis}\PY{o}{=}\PY{l+m+mi}{1}\PY{p}{)}\PY{p}{,} \PY{n}{np}\PY{o}{.}\PY{n}{mean}\PY{p}{(}\PY{n}{all\PYZus{}derivatives}\PY{p}{,} \PY{n}{axis}\PY{o}{=}\PY{l+m+mi}{2}\PY{p}{)}\PY{p}{,} \PY{n}{np}\PY{o}{.}\PY{n}{std}\PY{p}{(}\PY{n}{all\PYZus{}derivatives}\PY{p}{,} \PY{n}{axis}\PY{o}{=}\PY{l+m+mi}{2}\PY{p}{)}\PY{p}{)}
\end{Verbatim}
\end{tcolorbox}


    \begin{verbatim}
HBox(children=(IntProgress(value=0), HTML(value='')))
    \end{verbatim}


    \begin{Verbatim}[commandchars=\\\{\}]

    \end{Verbatim}

            \begin{tcolorbox}[breakable, size=fbox, boxrule=.5pt, pad at break*=1mm, opacityfill=0]
\prompt{Out}{outcolor}{8}{\boxspacing}
\begin{Verbatim}[commandchars=\\\{\}]
(array([0.04784253, 0.15288083, 0.44046486, 0.35277411]),
 array([0.0008991 , 0.00287306, 0.00211771, 0.0016961 ]),
 array([[-0.29478105,  0.06650959,  0.2168944 ,  0.00709914],
        [ 0.06650959, -0.16315672,  0.09183023,  0.00300568],
        [ 0.2168944 ,  0.09183023, -0.35012373,  0.03100105],
        [ 0.00709914,  0.00300568,  0.03100105, -0.04144621]]),
 array([[3.08401066e-03, 1.64034900e-03, 1.89106999e-03, 6.18963701e-05],
        [1.64034900e-03, 2.14278030e-03, 8.00654110e-04, 2.62061074e-05],
        [1.89106999e-03, 8.00654110e-04, 2.45783477e-03, 3.51894072e-04],
        [6.18963701e-05, 2.62061074e-05, 3.51894072e-04, 2.89493485e-04]]))
\end{Verbatim}
\end{tcolorbox}

    \begin{tcolorbox}[breakable, size=fbox, boxrule=1pt, pad at break*=1mm,colback=cellbackground, colframe=cellborder]
\prompt{In}{incolor}{9}{\boxspacing}
\begin{Verbatim}[commandchars=\\\{\}]
\PY{n}{mc} \PY{o}{=} \PY{n}{np}\PY{o}{.}\PY{n}{exp}\PY{p}{(} \PY{n}{gamma0} \PY{o}{+} \PY{n}{gamma1}\PY{o}{*}\PY{n}{w} \PY{o}{+} \PY{n}{omega}\PY{o}{/}\PY{l+m+mi}{8}\PY{p}{)}
\end{Verbatim}
\end{tcolorbox}

    \begin{tcolorbox}[breakable, size=fbox, boxrule=1pt, pad at break*=1mm,colback=cellbackground, colframe=cellborder]
\prompt{In}{incolor}{10}{\boxspacing}
\begin{Verbatim}[commandchars=\\\{\}]
\PY{c+c1}{\PYZsh{} define function to solve}

\PY{k}{def} \PY{n+nf}{get\PYZus{}function\PYZus{}to\PYZus{}solve}\PY{p}{(}\PY{n}{t}\PY{p}{,} \PY{n}{beta2}\PY{p}{,} \PY{n}{beta3}\PY{p}{)}\PY{p}{:}
    \PY{k}{def} \PY{n+nf}{F}\PY{p}{(}\PY{n}{p}\PY{p}{)}\PY{p}{:}
        \PY{c+c1}{\PYZsh{} p is a }
        \PY{n}{ds\PYZus{}dp} \PY{o}{=} \PY{n}{own\PYZus{}mkt\PYZus{}share\PYZus{}derivative}\PY{p}{(}\PY{n}{t}\PY{p}{,} \PY{n}{p}\PY{p}{,} \PY{n}{beta2}\PY{p}{,} \PY{n}{beta3}\PY{p}{)}
        \PY{n}{shares} \PY{o}{=} \PY{n}{mkt\PYZus{}share}\PY{p}{(}\PY{n}{t}\PY{p}{,} \PY{n}{p}\PY{p}{,} \PY{n}{beta2}\PY{p}{,} \PY{n}{beta3}\PY{p}{)}
        \PY{k}{return} \PY{n}{p} \PY{o}{\PYZhy{}} \PY{n}{mc}\PY{p}{[}\PY{p}{:}\PY{p}{,}\PY{n}{t}\PY{p}{]} \PY{o}{+} \PY{n}{np}\PY{o}{.}\PY{n}{reciprocal}\PY{p}{(}\PY{n}{ds\PYZus{}dp}\PY{p}{)}\PY{o}{*}\PY{n}{shares}

    \PY{k}{return} \PY{n}{F}
\end{Verbatim}
\end{tcolorbox}

    \begin{tcolorbox}[breakable, size=fbox, boxrule=1pt, pad at break*=1mm,colback=cellbackground, colframe=cellborder]
\prompt{In}{incolor}{11}{\boxspacing}
\begin{Verbatim}[commandchars=\\\{\}]
\PY{c+c1}{\PYZsh{} draw betas, now compute equilibrium prices and shares}

\PY{n}{beta2} \PY{o}{=} \PY{n}{rng}\PY{o}{.}\PY{n}{normal}\PY{p}{(}\PY{n}{beta2\PYZus{}bar}\PY{p}{,} \PY{n}{sigma2}\PY{p}{,} \PY{p}{(}\PY{n}{N}\PY{p}{,}\PY{n}{T}\PY{p}{)}\PY{p}{)}
\PY{n}{beta3} \PY{o}{=} \PY{n}{rng}\PY{o}{.}\PY{n}{normal}\PY{p}{(}\PY{n}{beta3\PYZus{}bar}\PY{p}{,} \PY{n}{sigma3}\PY{p}{,} \PY{p}{(}\PY{n}{N}\PY{p}{,}\PY{n}{T}\PY{p}{)}\PY{p}{)}
\end{Verbatim}
\end{tcolorbox}

    \begin{tcolorbox}[breakable, size=fbox, boxrule=1pt, pad at break*=1mm,colback=cellbackground, colframe=cellborder]
\prompt{In}{incolor}{12}{\boxspacing}
\begin{Verbatim}[commandchars=\\\{\}]
\PY{c+c1}{\PYZsh{} 3.2 and 3.3: compute equilibrium shares, prices}

\PY{c+c1}{\PYZsh{} these two variables are the prices and shares}
\PY{n}{eq\PYZus{}prices} \PY{o}{=} \PY{n}{np}\PY{o}{.}\PY{n}{zeros}\PY{p}{(}\PY{p}{(}\PY{n}{J}\PY{p}{,} \PY{n}{T}\PY{p}{)}\PY{p}{)}
\PY{n}{eq\PYZus{}shares} \PY{o}{=} \PY{n}{np}\PY{o}{.}\PY{n}{zeros}\PY{p}{(}\PY{p}{(}\PY{n}{J}\PY{p}{,} \PY{n}{T}\PY{p}{)}\PY{p}{)}

\PY{n}{flag\PYZus{}total} \PY{o}{=} \PY{l+m+mi}{0}

\PY{k}{for} \PY{n}{t} \PY{o+ow}{in} \PY{n}{trange}\PY{p}{(}\PY{n}{T}\PY{p}{)}\PY{p}{:}
    \PY{n}{fn} \PY{o}{=} \PY{n}{get\PYZus{}function\PYZus{}to\PYZus{}solve}\PY{p}{(}\PY{n}{t}\PY{p}{,} \PY{n}{beta2}\PY{p}{[}\PY{p}{:}\PY{p}{,}\PY{n}{t}\PY{p}{]}\PY{p}{,} \PY{n}{beta3}\PY{p}{[}\PY{p}{:}\PY{p}{,}\PY{n}{t}\PY{p}{]}\PY{p}{)}
    \PY{n}{mkt\PYZus{}eq\PYZus{}prices}\PY{p}{,} \PY{n}{\PYZus{}} \PY{p}{,} \PY{n}{flag}\PY{p}{,} \PY{n}{\PYZus{}} \PY{o}{=} \PY{n}{fsolve}\PY{p}{(}\PY{n}{fn}\PY{p}{,} \PY{n}{np}\PY{o}{.}\PY{n}{array}\PY{p}{(}\PY{p}{[}\PY{l+m+mi}{1}\PY{p}{,}\PY{l+m+mi}{1}\PY{p}{,}\PY{l+m+mi}{1}\PY{p}{,}\PY{l+m+mi}{1}\PY{p}{]}\PY{p}{)}\PY{p}{,} \PY{n}{full\PYZus{}output}\PY{o}{=}\PY{k+kc}{True}\PY{p}{)}
    \PY{n}{flag\PYZus{}total} \PY{o}{+}\PY{o}{=} \PY{n}{flag}
    \PY{n}{eq\PYZus{}prices}\PY{p}{[}\PY{p}{:}\PY{p}{,}\PY{n}{t}\PY{p}{]} \PY{o}{=} \PY{n}{mkt\PYZus{}eq\PYZus{}prices}
    \PY{n}{eq\PYZus{}shares}\PY{p}{[}\PY{p}{:}\PY{p}{,} \PY{n}{t}\PY{p}{]} \PY{o}{=} \PY{n}{mkt\PYZus{}share}\PY{p}{(}\PY{n}{t}\PY{p}{,} \PY{n}{mkt\PYZus{}eq\PYZus{}prices}\PY{p}{,} \PY{n}{beta2}\PY{p}{[}\PY{p}{:}\PY{p}{,}\PY{n}{t}\PY{p}{]}\PY{p}{,} \PY{n}{beta3}\PY{p}{[}\PY{p}{:}\PY{p}{,}\PY{n}{t}\PY{p}{]}\PY{p}{)}

\PY{c+c1}{\PYZsh{} this should be True iff all of the fsolves converge}
\PY{n}{flag\PYZus{}total} \PY{o}{==} \PY{n}{T}
\end{Verbatim}
\end{tcolorbox}


    \begin{verbatim}
HBox(children=(IntProgress(value=0, max=600), HTML(value='')))
    \end{verbatim}


    \begin{Verbatim}[commandchars=\\\{\}]

    \end{Verbatim}

            \begin{tcolorbox}[breakable, size=fbox, boxrule=.5pt, pad at break*=1mm, opacityfill=0]
\prompt{Out}{outcolor}{12}{\boxspacing}
\begin{Verbatim}[commandchars=\\\{\}]
True
\end{Verbatim}
\end{tcolorbox}

    \begin{tcolorbox}[breakable, size=fbox, boxrule=1pt, pad at break*=1mm,colback=cellbackground, colframe=cellborder]
\prompt{In}{incolor}{13}{\boxspacing}
\begin{Verbatim}[commandchars=\\\{\}]
\PY{c+c1}{\PYZsh{} check that at the equilibrium prices, the estimates for market shares and market share derivatives are precise}
\PY{c+c1}{\PYZsh{} repeating the exercise of simulation with equilibrium prices, trying to get equilibrium shares}

\PY{n}{S} \PY{o}{=} \PY{l+m+mi}{100}

\PY{n}{all\PYZus{}derivatives} \PY{o}{=} \PY{n}{np}\PY{o}{.}\PY{n}{zeros}\PY{p}{(}\PY{p}{(}\PY{n}{J}\PY{p}{,}\PY{n}{J}\PY{p}{,}\PY{n}{S}\PY{p}{)}\PY{p}{)}
\PY{n}{all\PYZus{}shares} \PY{o}{=} \PY{n}{np}\PY{o}{.}\PY{n}{zeros}\PY{p}{(}\PY{p}{(}\PY{n}{J}\PY{p}{,}\PY{n}{S}\PY{p}{)}\PY{p}{)}

\PY{n}{N} \PY{o}{=} \PY{l+m+mi}{100}

\PY{k}{for} \PY{n}{t} \PY{o+ow}{in} \PY{n}{trange}\PY{p}{(}\PY{n}{T}\PY{p}{)}\PY{p}{:}
    \PY{n}{price} \PY{o}{=} \PY{n}{np}\PY{o}{.}\PY{n}{array}\PY{p}{(}\PY{n}{eq\PYZus{}prices}\PY{p}{[}\PY{p}{:}\PY{p}{,}\PY{n}{t}\PY{p}{]}\PY{p}{)}
    \PY{k}{for} \PY{n}{s} \PY{o+ow}{in} \PY{n+nb}{range}\PY{p}{(}\PY{n}{S}\PY{p}{)}\PY{p}{:}
        \PY{n}{beta2\PYZus{}s} \PY{o}{=} \PY{n}{np}\PY{o}{.}\PY{n}{random}\PY{o}{.}\PY{n}{normal}\PY{p}{(}\PY{n}{beta2\PYZus{}bar}\PY{p}{,} \PY{n}{sigma2}\PY{p}{,} \PY{n}{N}\PY{p}{)}
        \PY{n}{beta3\PYZus{}s} \PY{o}{=} \PY{n}{np}\PY{o}{.}\PY{n}{random}\PY{o}{.}\PY{n}{normal}\PY{p}{(}\PY{n}{beta3\PYZus{}bar}\PY{p}{,} \PY{n}{sigma3}\PY{p}{,} \PY{n}{N}\PY{p}{)}
        \PY{n}{all\PYZus{}derivatives}\PY{p}{[}\PY{p}{:}\PY{p}{,}\PY{p}{:}\PY{p}{,}\PY{n}{s}\PY{p}{]} \PY{o}{=} \PY{n}{full\PYZus{}mkt\PYZus{}share\PYZus{}derivative}\PY{p}{(}\PY{l+m+mi}{0}\PY{p}{,} \PY{n}{price}\PY{p}{,} \PY{n}{beta2\PYZus{}s}\PY{p}{,} \PY{n}{beta3\PYZus{}s}\PY{p}{)}
        \PY{n}{all\PYZus{}shares}\PY{p}{[}\PY{p}{:}\PY{p}{,}\PY{n}{s}\PY{p}{]} \PY{o}{=} \PY{n}{mkt\PYZus{}share}\PY{p}{(}\PY{n}{t}\PY{p}{,} \PY{n}{price}\PY{p}{,} \PY{n}{beta2\PYZus{}s}\PY{p}{,} \PY{n}{beta3\PYZus{}s}\PY{p}{)}

\PY{p}{(}\PY{n}{np}\PY{o}{.}\PY{n}{mean}\PY{p}{(}\PY{n}{all\PYZus{}shares}\PY{p}{,}\PY{n}{axis}\PY{o}{=}\PY{l+m+mi}{1}\PY{p}{)}\PY{p}{,} \PY{n}{np}\PY{o}{.}\PY{n}{std}\PY{p}{(}\PY{n}{all\PYZus{}shares}\PY{p}{,}\PY{n}{axis}\PY{o}{=}\PY{l+m+mi}{1}\PY{p}{)}\PY{p}{,} \PY{n}{np}\PY{o}{.}\PY{n}{mean}\PY{p}{(}\PY{n}{all\PYZus{}derivatives}\PY{p}{,} \PY{n}{axis}\PY{o}{=}\PY{l+m+mi}{2}\PY{p}{)}\PY{p}{,} \PY{n}{np}\PY{o}{.}\PY{n}{std}\PY{p}{(}\PY{n}{all\PYZus{}derivatives}\PY{p}{,} \PY{n}{axis}\PY{o}{=}\PY{l+m+mi}{2}\PY{p}{)}\PY{p}{)}
\end{Verbatim}
\end{tcolorbox}


    \begin{verbatim}
HBox(children=(IntProgress(value=0, max=600), HTML(value='')))
    \end{verbatim}


    \begin{Verbatim}[commandchars=\\\{\}]

    \end{Verbatim}

            \begin{tcolorbox}[breakable, size=fbox, boxrule=.5pt, pad at break*=1mm, opacityfill=0]
\prompt{Out}{outcolor}{13}{\boxspacing}
\begin{Verbatim}[commandchars=\\\{\}]
(array([0.25025446, 0.1106518 , 0.40782337, 0.08575528]),
 array([0.01739545, 0.00769152, 0.02222747, 0.00467389]),
 array([[-0.3235269 ,  0.04609141,  0.19011712,  0.00800107],
        [ 0.04609141, -0.1104657 ,  0.04411352,  0.00185652],
        [ 0.19011712,  0.04411352, -0.39876114,  0.02550838],
        [ 0.00800107,  0.00185652,  0.02550838, -0.0412167 ]]),
 array([[1.66922701e-02, 5.70831379e-03, 9.55360361e-03, 4.02062959e-04],
        [5.70831379e-03, 7.73063888e-03, 2.21675482e-03, 9.32920224e-05],
        [9.55360361e-03, 2.21675482e-03, 1.02488422e-02, 2.18776846e-03],
        [4.02062959e-04, 9.32920224e-05, 2.18776846e-03, 2.04859466e-03]]))
\end{Verbatim}
\end{tcolorbox}

    \begin{tcolorbox}[breakable, size=fbox, boxrule=1pt, pad at break*=1mm,colback=cellbackground, colframe=cellborder]
\prompt{In}{incolor}{14}{\boxspacing}
\begin{Verbatim}[commandchars=\\\{\}]
\PY{c+c1}{\PYZsh{} Morrow and Skerlos (2011) Method: (see equation 27 in Conlon + Gortmaker)}

\PY{k}{def} \PY{n+nf}{get\PYZus{}matrices}\PY{p}{(}\PY{n}{t}\PY{p}{,} \PY{n}{p}\PY{p}{,} \PY{n}{beta2}\PY{p}{,} \PY{n}{beta3}\PY{p}{)}\PY{p}{:}
    \PY{c+c1}{\PYZsh{} p should be a length J vector}
    \PY{c+c1}{\PYZsh{} betas should be num\PYZus{}sims}

    \PY{n}{u\PYZus{}t} \PY{o}{=} \PY{n}{np}\PY{o}{.}\PY{n}{tile}\PY{p}{(}\PY{n}{x}\PY{p}{[}\PY{p}{:}\PY{p}{,}\PY{n}{t}\PY{p}{]} \PY{o}{+} \PY{n}{xi}\PY{p}{[}\PY{p}{:}\PY{p}{,}\PY{n}{t}\PY{p}{]} \PY{o}{+} \PY{n}{alpha}\PY{o}{*}\PY{n}{p}\PY{p}{,} \PY{p}{(}\PY{n+nb}{len}\PY{p}{(}\PY{n}{beta2}\PY{p}{)}\PY{p}{,} \PY{l+m+mi}{1}\PY{p}{)}\PY{p}{)} \PY{c+c1}{\PYZsh{} num\PYZus{}sims x J }
    \PY{k}{for} \PY{n}{j} \PY{o+ow}{in} \PY{n+nb}{range}\PY{p}{(}\PY{n}{J}\PY{p}{)}\PY{p}{:}
        \PY{k}{if} \PY{n}{j} \PY{o}{\PYZlt{}} \PY{l+m+mi}{2}\PY{p}{:}
            \PY{n}{u\PYZus{}t}\PY{p}{[}\PY{p}{:}\PY{p}{,}\PY{n}{j}\PY{p}{]} \PY{o}{=} \PY{n}{u\PYZus{}t}\PY{p}{[}\PY{p}{:}\PY{p}{,}\PY{n}{j}\PY{p}{]} \PY{o}{+} \PY{n}{beta2}
        \PY{k}{else}\PY{p}{:}
            \PY{n}{u\PYZus{}t}\PY{p}{[}\PY{p}{:}\PY{p}{,}\PY{n}{j}\PY{p}{]} \PY{o}{=} \PY{n}{u\PYZus{}t}\PY{p}{[}\PY{p}{:}\PY{p}{,}\PY{n}{j}\PY{p}{]} \PY{o}{+} \PY{n}{beta3}

    \PY{n}{Z} \PY{o}{=} \PY{n}{np}\PY{o}{.}\PY{n}{tile}\PY{p}{(} \PY{l+m+mi}{1} \PY{o}{+} \PY{n}{np}\PY{o}{.}\PY{n}{sum}\PY{p}{(}\PY{n}{np}\PY{o}{.}\PY{n}{exp}\PY{p}{(}\PY{n}{u\PYZus{}t}\PY{p}{)}\PY{p}{,}\PY{n}{axis}\PY{o}{=}\PY{o}{\PYZhy{}}\PY{l+m+mi}{1}\PY{p}{)}\PY{p}{,} \PY{p}{(}\PY{n}{J}\PY{p}{,}\PY{l+m+mi}{1}\PY{p}{)}\PY{p}{)}\PY{o}{.}\PY{n}{T} \PY{c+c1}{\PYZsh{} num\PYZus{}sims x J}

    \PY{n}{Lambda\PYZus{}inv} \PY{o}{=} \PY{n}{np}\PY{o}{.}\PY{n}{zeros}\PY{p}{(}\PY{p}{(}\PY{n}{J}\PY{p}{,}\PY{n}{J}\PY{p}{)}\PY{p}{)}
    \PY{n}{Gamma} \PY{o}{=} \PY{n}{np}\PY{o}{.}\PY{n}{zeros}\PY{p}{(}\PY{p}{(}\PY{n}{J}\PY{p}{,}\PY{n}{J}\PY{p}{)}\PY{p}{)}

    \PY{n}{own\PYZus{}numerator} \PY{o}{=} \PY{n}{alpha}\PY{o}{*}\PY{n}{np}\PY{o}{.}\PY{n}{exp}\PY{p}{(}\PY{n}{u\PYZus{}t}\PY{p}{)}  \PY{c+c1}{\PYZsh{} num\PYZus{}sims x J}
    \PY{n}{denominator} \PY{o}{=} \PY{n}{Z}

    \PY{k}{for} \PY{n}{j} \PY{o+ow}{in} \PY{n+nb}{range}\PY{p}{(}\PY{n}{J}\PY{p}{)}\PY{p}{:}
        \PY{n}{Lambda\PYZus{}inv}\PY{p}{[}\PY{n}{j}\PY{p}{,}\PY{n}{j}\PY{p}{]} \PY{o}{=} \PY{l+m+mi}{1} \PY{o}{/} \PY{p}{(}\PY{n}{np}\PY{o}{.}\PY{n}{mean}\PY{p}{(}\PY{n}{own\PYZus{}numerator} \PY{o}{/} \PY{n}{denominator}\PY{p}{,} \PY{n}{axis}\PY{o}{=}\PY{l+m+mi}{0}\PY{p}{)}\PY{p}{[}\PY{n}{j}\PY{p}{]}\PY{p}{)}

    \PY{k}{for} \PY{n}{j} \PY{o+ow}{in} \PY{n+nb}{range}\PY{p}{(}\PY{n}{J}\PY{p}{)}\PY{p}{:}
        \PY{k}{for} \PY{n}{k} \PY{o+ow}{in} \PY{n+nb}{range}\PY{p}{(}\PY{n}{J}\PY{p}{)}\PY{p}{:}
            \PY{n}{Gamma}\PY{p}{[}\PY{n}{j}\PY{p}{,}\PY{n}{k}\PY{p}{]} \PY{o}{=} \PY{n}{np}\PY{o}{.}\PY{n}{mean}\PY{p}{(}\PY{n}{alpha}\PY{o}{*}\PY{n}{np}\PY{o}{.}\PY{n}{exp}\PY{p}{(}\PY{n}{u\PYZus{}t}\PY{p}{)}\PY{p}{[}\PY{p}{:}\PY{p}{,}\PY{n}{k}\PY{p}{]}\PY{o}{*}\PY{n}{np}\PY{o}{.}\PY{n}{exp}\PY{p}{(}\PY{n}{u\PYZus{}t}\PY{p}{)}\PY{p}{[}\PY{p}{:}\PY{p}{,}\PY{n}{j}\PY{p}{]} \PY{o}{/} \PY{n}{np}\PY{o}{.}\PY{n}{square}\PY{p}{(}\PY{l+m+mi}{1} \PY{o}{+} \PY{n}{np}\PY{o}{.}\PY{n}{sum}\PY{p}{(}\PY{n}{np}\PY{o}{.}\PY{n}{exp}\PY{p}{(}\PY{n}{u\PYZus{}t}\PY{p}{)}\PY{p}{,}\PY{n}{axis}\PY{o}{=}\PY{o}{\PYZhy{}}\PY{l+m+mi}{1}\PY{p}{)}\PY{p}{)}\PY{p}{)}

    \PY{k}{return} \PY{n}{Lambda\PYZus{}inv}\PY{p}{,} \PY{n}{Gamma}

\PY{k}{def} \PY{n+nf}{get\PYZus{}fixed\PYZus{}point\PYZus{}function}\PY{p}{(}\PY{n}{t}\PY{p}{,} \PY{n}{beta2}\PY{p}{,} \PY{n}{beta3}\PY{p}{)}\PY{p}{:}
    \PY{n}{ownership\PYZus{}matrix} \PY{o}{=} \PY{n}{np}\PY{o}{.}\PY{n}{identity}\PY{p}{(}\PY{n}{J}\PY{p}{)}
    \PY{k}{def} \PY{n+nf}{F}\PY{p}{(}\PY{n}{p}\PY{p}{)}\PY{p}{:}
        \PY{n}{Lambda\PYZus{}inv}\PY{p}{,} \PY{n}{Gamma} \PY{o}{=} \PY{n}{get\PYZus{}matrices}\PY{p}{(}\PY{n}{t}\PY{p}{,} \PY{n}{p}\PY{p}{,} \PY{n}{beta2}\PY{p}{,} \PY{n}{beta3}\PY{p}{)}
        \PY{n}{shares} \PY{o}{=} \PY{n}{mkt\PYZus{}share}\PY{p}{(}\PY{n}{t}\PY{p}{,} \PY{n}{p}\PY{p}{,} \PY{n}{beta2}\PY{p}{,} \PY{n}{beta3}\PY{p}{)}
        \PY{n}{zeta} \PY{o}{=} \PY{n}{np}\PY{o}{.}\PY{n}{matmul}\PY{p}{(}\PY{n}{np}\PY{o}{.}\PY{n}{matmul}\PY{p}{(}\PY{n}{Lambda\PYZus{}inv}\PY{p}{,} \PY{n}{ownership\PYZus{}matrix}\PY{o}{*}\PY{n}{Gamma}\PY{p}{)}\PY{p}{,} \PY{p}{(}\PY{n}{p} \PY{o}{\PYZhy{}} \PY{n}{mc}\PY{p}{[}\PY{p}{:}\PY{p}{,}\PY{n}{t}\PY{p}{]}\PY{p}{)}\PY{p}{)} \PY{o}{\PYZhy{}} \PY{n}{np}\PY{o}{.}\PY{n}{matmul}\PY{p}{(}\PY{n}{Lambda\PYZus{}inv}\PY{p}{,} \PY{n}{shares}\PY{p}{)}
        \PY{k}{return} \PY{n}{mc}\PY{p}{[}\PY{p}{:}\PY{p}{,}\PY{n}{t}\PY{p}{]} \PY{o}{+} \PY{n}{zeta}

    \PY{k}{return} \PY{n}{F}
\end{Verbatim}
\end{tcolorbox}

    \begin{tcolorbox}[breakable, size=fbox, boxrule=1pt, pad at break*=1mm,colback=cellbackground, colframe=cellborder]
\prompt{In}{incolor}{15}{\boxspacing}
\begin{Verbatim}[commandchars=\\\{\}]
\PY{c+c1}{\PYZsh{} Simulate equilibrium using the Morrow and Skerlos (2011) method}
\PY{n}{eq\PYZus{}prices\PYZus{}2} \PY{o}{=} \PY{n}{np}\PY{o}{.}\PY{n}{zeros}\PY{p}{(}\PY{p}{(}\PY{n}{J}\PY{p}{,} \PY{n}{T}\PY{p}{)}\PY{p}{)}
\PY{n}{eq\PYZus{}shares\PYZus{}2} \PY{o}{=} \PY{n}{np}\PY{o}{.}\PY{n}{zeros}\PY{p}{(}\PY{p}{(}\PY{n}{J}\PY{p}{,} \PY{n}{T}\PY{p}{)}\PY{p}{)}

\PY{k}{for} \PY{n}{t} \PY{o+ow}{in} \PY{n}{trange}\PY{p}{(}\PY{n}{T}\PY{p}{)}\PY{p}{:}
    \PY{n}{fn} \PY{o}{=} \PY{n}{get\PYZus{}fixed\PYZus{}point\PYZus{}function}\PY{p}{(}\PY{n}{t}\PY{p}{,} \PY{n}{beta2}\PY{p}{[}\PY{p}{:}\PY{p}{,}\PY{n}{t}\PY{p}{]}\PY{p}{,} \PY{n}{beta3}\PY{p}{[}\PY{p}{:}\PY{p}{,}\PY{n}{t}\PY{p}{]}\PY{p}{)}
    \PY{n}{mkt\PYZus{}eq\PYZus{}prices} \PY{o}{=} \PY{n}{fixed\PYZus{}point}\PY{p}{(}\PY{n}{fn}\PY{p}{,} \PY{n}{np}\PY{o}{.}\PY{n}{array}\PY{p}{(}\PY{p}{[}\PY{l+m+mi}{1}\PY{p}{,}\PY{l+m+mi}{1}\PY{p}{,}\PY{l+m+mi}{1}\PY{p}{,}\PY{l+m+mi}{1}\PY{p}{]}\PY{p}{)}\PY{p}{,} \PY{n}{method}\PY{o}{=}\PY{l+s+s2}{\PYZdq{}}\PY{l+s+s2}{iteration}\PY{l+s+s2}{\PYZdq{}}\PY{p}{)}
    \PY{n}{eq\PYZus{}prices\PYZus{}2}\PY{p}{[}\PY{p}{:}\PY{p}{,}\PY{n}{t}\PY{p}{]} \PY{o}{=} \PY{n}{mkt\PYZus{}eq\PYZus{}prices}
    \PY{n}{eq\PYZus{}shares\PYZus{}2}\PY{p}{[}\PY{p}{:}\PY{p}{,} \PY{n}{t}\PY{p}{]} \PY{o}{=} \PY{n}{mkt\PYZus{}share}\PY{p}{(}\PY{n}{t}\PY{p}{,} \PY{n}{mkt\PYZus{}eq\PYZus{}prices}\PY{p}{,} \PY{n}{beta2}\PY{p}{[}\PY{p}{:}\PY{p}{,}\PY{n}{t}\PY{p}{]}\PY{p}{,} \PY{n}{beta3}\PY{p}{[}\PY{p}{:}\PY{p}{,}\PY{n}{t}\PY{p}{]}\PY{p}{)}

\PY{c+c1}{\PYZsh{} the difference between the two methods, check that this is small}
\PY{n}{np}\PY{o}{.}\PY{n}{max}\PY{p}{(}\PY{n}{eq\PYZus{}prices\PYZus{}2} \PY{o}{\PYZhy{}} \PY{n}{eq\PYZus{}prices}\PY{p}{)}\PY{p}{,} \PY{n}{np}\PY{o}{.}\PY{n}{max}\PY{p}{(}\PY{n}{eq\PYZus{}shares\PYZus{}2} \PY{o}{\PYZhy{}} \PY{n}{eq\PYZus{}shares}\PY{p}{)}
\end{Verbatim}
\end{tcolorbox}


    \begin{verbatim}
HBox(children=(IntProgress(value=0, max=600), HTML(value='')))
    \end{verbatim}


    \begin{Verbatim}[commandchars=\\\{\}]

    \end{Verbatim}

            \begin{tcolorbox}[breakable, size=fbox, boxrule=.5pt, pad at break*=1mm, opacityfill=0]
\prompt{Out}{outcolor}{15}{\boxspacing}
\begin{Verbatim}[commandchars=\\\{\}]
(1.373072322508051e-09, 4.207107107134789e-09)
\end{Verbatim}
\end{tcolorbox}

    \begin{tcolorbox}[breakable, size=fbox, boxrule=1pt, pad at break*=1mm,colback=cellbackground, colframe=cellborder]
\prompt{In}{incolor}{16}{\boxspacing}
\begin{Verbatim}[commandchars=\\\{\}]
\PY{c+c1}{\PYZsh{} Precompute the price elasticities and diversion }
\PY{c+c1}{\PYZsh{} What PyBLP does, and what we will do, is replace the diagonal of the diversion ratio matrix with the outside option diversion ratio (instead of \PYZhy{}1)}
\PY{n}{true\PYZus{}price\PYZus{}elasticities} \PY{o}{=} \PY{n}{np}\PY{o}{.}\PY{n}{zeros}\PY{p}{(}\PY{p}{(}\PY{n}{J}\PY{p}{,}\PY{n}{J}\PY{p}{,}\PY{n}{T}\PY{p}{)}\PY{p}{)}
\PY{n}{true\PYZus{}diversion\PYZus{}ratios} \PY{o}{=} \PY{n}{np}\PY{o}{.}\PY{n}{zeros}\PY{p}{(}\PY{p}{(}\PY{n}{J}\PY{p}{,}\PY{n}{J}\PY{p}{,}\PY{n}{T}\PY{p}{)}\PY{p}{)}

\PY{n}{N} \PY{o}{=} \PY{l+m+mi}{100}

\PY{k}{for} \PY{n}{t} \PY{o+ow}{in} \PY{n}{trange}\PY{p}{(}\PY{n}{T}\PY{p}{)}\PY{p}{:}
    \PY{n}{own\PYZus{}price\PYZus{}derivative} \PY{o}{=} \PY{n}{own\PYZus{}mkt\PYZus{}share\PYZus{}derivative}\PY{p}{(}\PY{n}{t}\PY{p}{,} \PY{n}{eq\PYZus{}prices}\PY{p}{[}\PY{p}{:}\PY{p}{,}\PY{n}{t}\PY{p}{]}\PY{p}{,} \PY{n}{beta2}\PY{p}{[}\PY{p}{:}\PY{p}{,}\PY{n}{t}\PY{p}{]}\PY{p}{,} \PY{n}{beta3}\PY{p}{[}\PY{p}{:}\PY{p}{,}\PY{n}{t}\PY{p}{]}\PY{p}{)}
    \PY{n}{derivative\PYZus{}matrix} \PY{o}{=} \PY{n}{full\PYZus{}mkt\PYZus{}share\PYZus{}derivative}\PY{p}{(}\PY{n}{t}\PY{p}{,} \PY{n}{eq\PYZus{}prices}\PY{p}{[}\PY{p}{:}\PY{p}{,}\PY{n}{t}\PY{p}{]}\PY{p}{,} \PY{n}{beta2}\PY{p}{[}\PY{p}{:}\PY{p}{,}\PY{n}{t}\PY{p}{]}\PY{p}{,} \PY{n}{beta3}\PY{p}{[}\PY{p}{:}\PY{p}{,}\PY{n}{t}\PY{p}{]}\PY{p}{)}
    \PY{n}{true\PYZus{}price\PYZus{}elasticities}\PY{p}{[}\PY{p}{:}\PY{p}{,}\PY{p}{:}\PY{p}{,}\PY{n}{t}\PY{p}{]} \PY{o}{=} \PY{n}{eq\PYZus{}prices}\PY{p}{[}\PY{p}{:}\PY{p}{,}\PY{n}{t}\PY{p}{]}\PY{o}{*}\PY{n}{derivative\PYZus{}matrix} \PY{o}{/} \PY{n}{eq\PYZus{}shares}\PY{p}{[}\PY{p}{:}\PY{p}{,}\PY{n}{t}\PY{p}{]}\PY{o}{.}\PY{n}{T}
    \PY{n}{derivative\PYZus{}matrix} \PY{o}{=} \PY{n}{full\PYZus{}mkt\PYZus{}share\PYZus{}derivative}\PY{p}{(}\PY{n}{t}\PY{p}{,} \PY{n}{eq\PYZus{}prices}\PY{p}{[}\PY{p}{:}\PY{p}{,}\PY{n}{t}\PY{p}{]}\PY{p}{,} \PY{n}{beta2}\PY{p}{[}\PY{p}{:}\PY{p}{,}\PY{n}{t}\PY{p}{]}\PY{p}{,} \PY{n}{beta3}\PY{p}{[}\PY{p}{:}\PY{p}{,}\PY{n}{t}\PY{p}{]}\PY{p}{)}
    \PY{n}{outside\PYZus{}derivatives} \PY{o}{=} \PY{n}{outside\PYZus{}mkt\PYZus{}share\PYZus{}derivative}\PY{p}{(}\PY{n}{t}\PY{p}{,} \PY{n}{eq\PYZus{}prices}\PY{p}{[}\PY{p}{:}\PY{p}{,}\PY{n}{t}\PY{p}{]}\PY{p}{,} \PY{n}{beta2}\PY{p}{[}\PY{p}{:}\PY{p}{,}\PY{n}{t}\PY{p}{]}\PY{p}{,} \PY{n}{beta3}\PY{p}{[}\PY{p}{:}\PY{p}{,}\PY{n}{t}\PY{p}{]}\PY{p}{)}
    \PY{k}{for} \PY{n}{j} \PY{o+ow}{in} \PY{n+nb}{range}\PY{p}{(}\PY{n}{J}\PY{p}{)}\PY{p}{:}
        \PY{k}{for} \PY{n}{k} \PY{o+ow}{in} \PY{n+nb}{range}\PY{p}{(}\PY{n}{J}\PY{p}{)}\PY{p}{:}
            \PY{n}{true\PYZus{}diversion\PYZus{}ratios}\PY{p}{[}\PY{n}{j}\PY{p}{,}\PY{n}{k}\PY{p}{,}\PY{n}{t}\PY{p}{]} \PY{o}{=} \PY{o}{\PYZhy{}}\PY{l+m+mi}{1}\PY{o}{*}\PY{n}{derivative\PYZus{}matrix}\PY{p}{[}\PY{n}{k}\PY{p}{,}\PY{n}{j}\PY{p}{]}\PY{o}{/}\PY{n}{derivative\PYZus{}matrix}\PY{p}{[}\PY{n}{j}\PY{p}{,}\PY{n}{j}\PY{p}{]}
    \PY{k}{for} \PY{n}{j} \PY{o+ow}{in} \PY{n+nb}{range}\PY{p}{(}\PY{n}{J}\PY{p}{)}\PY{p}{:}
        \PY{n}{true\PYZus{}diversion\PYZus{}ratios}\PY{p}{[}\PY{n}{j}\PY{p}{,}\PY{n}{j}\PY{p}{,}\PY{n}{t}\PY{p}{]} \PY{o}{=} \PY{o}{\PYZhy{}}\PY{l+m+mi}{1}\PY{o}{*}\PY{n}{outside\PYZus{}derivatives}\PY{p}{[}\PY{n}{j}\PY{p}{]}\PY{o}{/}\PY{n}{derivative\PYZus{}matrix}\PY{p}{[}\PY{n}{j}\PY{p}{,}\PY{n}{j}\PY{p}{]}
\end{Verbatim}
\end{tcolorbox}


    \begin{verbatim}
HBox(children=(IntProgress(value=0, max=600), HTML(value='')))
    \end{verbatim}


    \begin{Verbatim}[commandchars=\\\{\}]

    \end{Verbatim}

    \begin{tcolorbox}[breakable, size=fbox, boxrule=1pt, pad at break*=1mm,colback=cellbackground, colframe=cellborder]
\prompt{In}{incolor}{17}{\boxspacing}
\begin{Verbatim}[commandchars=\\\{\}]
\PY{n}{market\PYZus{}ids} \PY{o}{=} \PY{n}{np}\PY{o}{.}\PY{n}{tile}\PY{p}{(}\PY{n}{np}\PY{o}{.}\PY{n}{arange}\PY{p}{(}\PY{n}{T}\PY{p}{)} \PY{o}{+} \PY{l+m+mi}{1}\PY{p}{,}\PY{p}{(}\PY{n}{J}\PY{p}{,}\PY{l+m+mi}{1}\PY{p}{)}\PY{p}{)}\PY{o}{.}\PY{n}{T}\PY{o}{.}\PY{n}{flatten}\PY{p}{(}\PY{p}{)}
\PY{n}{firm\PYZus{}ids} \PY{o}{=} \PY{n}{np}\PY{o}{.}\PY{n}{tile}\PY{p}{(}\PY{n}{np}\PY{o}{.}\PY{n}{arange}\PY{p}{(}\PY{n}{J}\PY{p}{)} \PY{o}{+} \PY{l+m+mi}{1}\PY{p}{,}\PY{p}{(}\PY{n}{T}\PY{p}{,}\PY{l+m+mi}{1}\PY{p}{)}\PY{p}{)}\PY{o}{.}\PY{n}{flatten}\PY{p}{(}\PY{p}{)}
\PY{n}{satellite} \PY{o}{=} \PY{n}{np}\PY{o}{.}\PY{n}{concatenate}\PY{p}{(}\PY{p}{(}\PY{n}{np}\PY{o}{.}\PY{n}{ones}\PY{p}{(}\PY{p}{(}\PY{l+m+mi}{2}\PY{p}{,}\PY{n}{T}\PY{p}{)}\PY{p}{)}\PY{p}{,} \PY{n}{np}\PY{o}{.}\PY{n}{zeros}\PY{p}{(}\PY{p}{(}\PY{l+m+mi}{2}\PY{p}{,}\PY{n}{T}\PY{p}{)}\PY{p}{)}\PY{p}{)}\PY{p}{)}\PY{o}{.}\PY{n}{T}\PY{o}{.}\PY{n}{flatten}\PY{p}{(}\PY{p}{)}
\PY{n}{wired} \PY{o}{=} \PY{n}{np}\PY{o}{.}\PY{n}{concatenate}\PY{p}{(}\PY{p}{(}\PY{n}{np}\PY{o}{.}\PY{n}{zeros}\PY{p}{(}\PY{p}{(}\PY{l+m+mi}{2}\PY{p}{,}\PY{n}{T}\PY{p}{)}\PY{p}{)}\PY{p}{,} \PY{n}{np}\PY{o}{.}\PY{n}{ones}\PY{p}{(}\PY{p}{(}\PY{l+m+mi}{2}\PY{p}{,}\PY{n}{T}\PY{p}{)}\PY{p}{)}\PY{p}{)}\PY{p}{)}\PY{o}{.}\PY{n}{T}\PY{o}{.}\PY{n}{flatten}\PY{p}{(}\PY{p}{)}
\PY{n}{observed\PYZus{}data} \PY{o}{=} \PY{n}{pd}\PY{o}{.}\PY{n}{DataFrame}\PY{p}{(}\PY{n}{data}\PY{o}{=}\PY{p}{\PYZob{}}
    \PY{l+s+s2}{\PYZdq{}}\PY{l+s+s2}{market\PYZus{}ids}\PY{l+s+s2}{\PYZdq{}}\PY{p}{:} \PY{n}{market\PYZus{}ids}\PY{p}{,}
    \PY{l+s+s2}{\PYZdq{}}\PY{l+s+s2}{firm\PYZus{}ids}\PY{l+s+s2}{\PYZdq{}}\PY{p}{:} \PY{n}{firm\PYZus{}ids}\PY{p}{,}
    \PY{l+s+s2}{\PYZdq{}}\PY{l+s+s2}{shares}\PY{l+s+s2}{\PYZdq{}}\PY{p}{:} \PY{n}{eq\PYZus{}shares}\PY{o}{.}\PY{n}{T}\PY{o}{.}\PY{n}{flatten}\PY{p}{(}\PY{p}{)}\PY{p}{,}
    \PY{l+s+s2}{\PYZdq{}}\PY{l+s+s2}{prices}\PY{l+s+s2}{\PYZdq{}}\PY{p}{:} \PY{n}{eq\PYZus{}prices}\PY{o}{.}\PY{n}{T}\PY{o}{.}\PY{n}{flatten}\PY{p}{(}\PY{p}{)}\PY{p}{,}
    \PY{l+s+s2}{\PYZdq{}}\PY{l+s+s2}{x}\PY{l+s+s2}{\PYZdq{}}\PY{p}{:} \PY{n}{x}\PY{o}{.}\PY{n}{T}\PY{o}{.}\PY{n}{flatten}\PY{p}{(}\PY{p}{)}\PY{p}{,}
    \PY{l+s+s2}{\PYZdq{}}\PY{l+s+s2}{satellite}\PY{l+s+s2}{\PYZdq{}}\PY{p}{:} \PY{n}{satellite}\PY{p}{,}
    \PY{l+s+s2}{\PYZdq{}}\PY{l+s+s2}{wired}\PY{l+s+s2}{\PYZdq{}}\PY{p}{:} \PY{n}{wired}\PY{p}{,}
    \PY{l+s+s2}{\PYZdq{}}\PY{l+s+s2}{w}\PY{l+s+s2}{\PYZdq{}}\PY{p}{:} \PY{n}{w}\PY{o}{.}\PY{n}{T}\PY{o}{.}\PY{n}{flatten}\PY{p}{(}\PY{p}{)}
\PY{p}{\PYZcb{}}\PY{p}{)}
\PY{n}{unobserved\PYZus{}data} \PY{o}{=} \PY{n}{pd}\PY{o}{.}\PY{n}{DataFrame}\PY{p}{(}\PY{n}{data}\PY{o}{=}\PY{p}{\PYZob{}}
    \PY{l+s+s2}{\PYZdq{}}\PY{l+s+s2}{market\PYZus{}ids}\PY{l+s+s2}{\PYZdq{}}\PY{p}{:} \PY{n}{market\PYZus{}ids}\PY{p}{,}
    \PY{l+s+s2}{\PYZdq{}}\PY{l+s+s2}{firm\PYZus{}ids}\PY{l+s+s2}{\PYZdq{}}\PY{p}{:} \PY{n}{firm\PYZus{}ids}\PY{p}{,}
    \PY{l+s+s2}{\PYZdq{}}\PY{l+s+s2}{xi}\PY{l+s+s2}{\PYZdq{}}\PY{p}{:} \PY{n}{xi}\PY{o}{.}\PY{n}{T}\PY{o}{.}\PY{n}{flatten}\PY{p}{(}\PY{p}{)}\PY{p}{,}
    \PY{l+s+s2}{\PYZdq{}}\PY{l+s+s2}{omega}\PY{l+s+s2}{\PYZdq{}}\PY{p}{:} \PY{n}{omega}\PY{o}{.}\PY{n}{T}\PY{o}{.}\PY{n}{flatten}\PY{p}{(}\PY{p}{)}
\PY{p}{\PYZcb{}}\PY{p}{)}
\end{Verbatim}
\end{tcolorbox}

    \begin{tcolorbox}[breakable, size=fbox, boxrule=1pt, pad at break*=1mm,colback=cellbackground, colframe=cellborder]
\prompt{In}{incolor}{18}{\boxspacing}
\begin{Verbatim}[commandchars=\\\{\}]
\PY{c+c1}{\PYZsh{} Instrument Analysis}

\PY{n}{df1} \PY{o}{=} \PY{n}{pd}\PY{o}{.}\PY{n}{DataFrame}\PY{p}{(}\PY{p}{\PYZob{}}
    \PY{l+s+s1}{\PYZsq{}}\PY{l+s+s1}{p1}\PY{l+s+s1}{\PYZsq{}}\PY{p}{:}\PY{n}{eq\PYZus{}prices}\PY{p}{[}\PY{l+m+mi}{0}\PY{p}{,}\PY{p}{:}\PY{p}{]}\PY{p}{,}
    \PY{l+s+s2}{\PYZdq{}}\PY{l+s+s2}{s1}\PY{l+s+s2}{\PYZdq{}}\PY{p}{:}\PY{n}{eq\PYZus{}shares}\PY{p}{[}\PY{l+m+mi}{0}\PY{p}{,}\PY{p}{:}\PY{p}{]}\PY{p}{,}
    \PY{l+s+s1}{\PYZsq{}}\PY{l+s+s1}{p2}\PY{l+s+s1}{\PYZsq{}}\PY{p}{:}\PY{n}{eq\PYZus{}prices}\PY{p}{[}\PY{l+m+mi}{1}\PY{p}{,}\PY{p}{:}\PY{p}{]}\PY{p}{,}
    \PY{l+s+s2}{\PYZdq{}}\PY{l+s+s2}{s2}\PY{l+s+s2}{\PYZdq{}}\PY{p}{:}\PY{n}{eq\PYZus{}shares}\PY{p}{[}\PY{l+m+mi}{1}\PY{p}{,}\PY{p}{:}\PY{p}{]}\PY{p}{,}
    \PY{l+s+s1}{\PYZsq{}}\PY{l+s+s1}{p3}\PY{l+s+s1}{\PYZsq{}}\PY{p}{:}\PY{n}{eq\PYZus{}prices}\PY{p}{[}\PY{l+m+mi}{2}\PY{p}{,}\PY{p}{:}\PY{p}{]}\PY{p}{,}
    \PY{l+s+s2}{\PYZdq{}}\PY{l+s+s2}{s3}\PY{l+s+s2}{\PYZdq{}}\PY{p}{:}\PY{n}{eq\PYZus{}shares}\PY{p}{[}\PY{l+m+mi}{2}\PY{p}{,}\PY{p}{:}\PY{p}{]}\PY{p}{,}
    \PY{l+s+s1}{\PYZsq{}}\PY{l+s+s1}{p4}\PY{l+s+s1}{\PYZsq{}}\PY{p}{:}\PY{n}{eq\PYZus{}prices}\PY{p}{[}\PY{l+m+mi}{3}\PY{p}{,}\PY{p}{:}\PY{p}{]}\PY{p}{,}
    \PY{l+s+s2}{\PYZdq{}}\PY{l+s+s2}{s4}\PY{l+s+s2}{\PYZdq{}}\PY{p}{:}\PY{n}{eq\PYZus{}shares}\PY{p}{[}\PY{l+m+mi}{3}\PY{p}{,}\PY{p}{:}\PY{p}{]}\PY{p}{,}
    \PY{l+s+s1}{\PYZsq{}}\PY{l+s+s1}{x1}\PY{l+s+s1}{\PYZsq{}}\PY{p}{:}\PY{n}{pd}\PY{o}{.}\PY{n}{Series}\PY{p}{(}\PY{n}{x}\PY{p}{[}\PY{l+m+mi}{0}\PY{p}{,}\PY{p}{:}\PY{p}{]}\PY{p}{)}\PY{p}{,}
    \PY{l+s+s1}{\PYZsq{}}\PY{l+s+s1}{w1}\PY{l+s+s1}{\PYZsq{}}\PY{p}{:}\PY{n}{pd}\PY{o}{.}\PY{n}{Series}\PY{p}{(}\PY{n}{w}\PY{p}{[}\PY{l+m+mi}{0}\PY{p}{,}\PY{p}{:}\PY{p}{]}\PY{p}{)}\PY{p}{,}
    \PY{l+s+s1}{\PYZsq{}}\PY{l+s+s1}{x2}\PY{l+s+s1}{\PYZsq{}}\PY{p}{:}\PY{n}{pd}\PY{o}{.}\PY{n}{Series}\PY{p}{(}\PY{n}{x}\PY{p}{[}\PY{l+m+mi}{1}\PY{p}{,}\PY{p}{:}\PY{p}{]}\PY{p}{)}\PY{p}{,}
    \PY{l+s+s1}{\PYZsq{}}\PY{l+s+s1}{w2}\PY{l+s+s1}{\PYZsq{}}\PY{p}{:}\PY{n}{pd}\PY{o}{.}\PY{n}{Series}\PY{p}{(}\PY{n}{w}\PY{p}{[}\PY{l+m+mi}{1}\PY{p}{,}\PY{p}{:}\PY{p}{]}\PY{p}{)}\PY{p}{,}
    \PY{l+s+s1}{\PYZsq{}}\PY{l+s+s1}{x3}\PY{l+s+s1}{\PYZsq{}}\PY{p}{:}\PY{n}{pd}\PY{o}{.}\PY{n}{Series}\PY{p}{(}\PY{n}{x}\PY{p}{[}\PY{l+m+mi}{2}\PY{p}{,}\PY{p}{:}\PY{p}{]}\PY{p}{)}\PY{p}{,}
    \PY{l+s+s1}{\PYZsq{}}\PY{l+s+s1}{w3}\PY{l+s+s1}{\PYZsq{}}\PY{p}{:}\PY{n}{pd}\PY{o}{.}\PY{n}{Series}\PY{p}{(}\PY{n}{w}\PY{p}{[}\PY{l+m+mi}{2}\PY{p}{,}\PY{p}{:}\PY{p}{]}\PY{p}{)}\PY{p}{,}
    \PY{l+s+s1}{\PYZsq{}}\PY{l+s+s1}{x4}\PY{l+s+s1}{\PYZsq{}}\PY{p}{:}\PY{n}{pd}\PY{o}{.}\PY{n}{Series}\PY{p}{(}\PY{n}{x}\PY{p}{[}\PY{l+m+mi}{3}\PY{p}{,}\PY{p}{:}\PY{p}{]}\PY{p}{)}\PY{p}{,}
    \PY{l+s+s1}{\PYZsq{}}\PY{l+s+s1}{w4}\PY{l+s+s1}{\PYZsq{}}\PY{p}{:}\PY{n}{pd}\PY{o}{.}\PY{n}{Series}\PY{p}{(}\PY{n}{w}\PY{p}{[}\PY{l+m+mi}{3}\PY{p}{,}\PY{p}{:}\PY{p}{]}\PY{p}{)}\PY{p}{,}

\PY{p}{\PYZcb{}}\PY{p}{)}

\PY{n}{X} \PY{o}{=} \PY{n}{df1}\PY{p}{[}\PY{p}{[}\PY{l+s+s2}{\PYZdq{}}\PY{l+s+s2}{x1}\PY{l+s+s2}{\PYZdq{}}\PY{p}{,}\PY{l+s+s2}{\PYZdq{}}\PY{l+s+s2}{x2}\PY{l+s+s2}{\PYZdq{}}\PY{p}{,}\PY{l+s+s2}{\PYZdq{}}\PY{l+s+s2}{x3}\PY{l+s+s2}{\PYZdq{}}\PY{p}{,}\PY{l+s+s2}{\PYZdq{}}\PY{l+s+s2}{x4}\PY{l+s+s2}{\PYZdq{}}\PY{p}{,}\PY{l+s+s2}{\PYZdq{}}\PY{l+s+s2}{w1}\PY{l+s+s2}{\PYZdq{}}\PY{p}{,}\PY{l+s+s2}{\PYZdq{}}\PY{l+s+s2}{w2}\PY{l+s+s2}{\PYZdq{}}\PY{p}{,}\PY{l+s+s2}{\PYZdq{}}\PY{l+s+s2}{w3}\PY{l+s+s2}{\PYZdq{}}\PY{p}{,}\PY{l+s+s2}{\PYZdq{}}\PY{l+s+s2}{w4}\PY{l+s+s2}{\PYZdq{}}\PY{p}{]}\PY{p}{]}
\PY{c+c1}{\PYZsh{} regress prices on observables }

\PY{n}{modelp1} \PY{o}{=} \PY{n}{sm}\PY{o}{.}\PY{n}{OLS}\PY{p}{(}\PY{n}{df1}\PY{p}{[}\PY{l+s+s2}{\PYZdq{}}\PY{l+s+s2}{p1}\PY{l+s+s2}{\PYZdq{}}\PY{p}{]}\PY{p}{,}\PY{n}{X}\PY{p}{)}\PY{o}{.}\PY{n}{fit}\PY{p}{(}\PY{p}{)}
\PY{n}{modelp2} \PY{o}{=} \PY{n}{sm}\PY{o}{.}\PY{n}{OLS}\PY{p}{(}\PY{n}{df1}\PY{p}{[}\PY{l+s+s2}{\PYZdq{}}\PY{l+s+s2}{p2}\PY{l+s+s2}{\PYZdq{}}\PY{p}{]}\PY{p}{,}\PY{n}{X}\PY{p}{)}\PY{o}{.}\PY{n}{fit}\PY{p}{(}\PY{p}{)}
\PY{n}{modelp3} \PY{o}{=} \PY{n}{sm}\PY{o}{.}\PY{n}{OLS}\PY{p}{(}\PY{n}{df1}\PY{p}{[}\PY{l+s+s2}{\PYZdq{}}\PY{l+s+s2}{p3}\PY{l+s+s2}{\PYZdq{}}\PY{p}{]}\PY{p}{,}\PY{n}{X}\PY{p}{)}\PY{o}{.}\PY{n}{fit}\PY{p}{(}\PY{p}{)}
\PY{n}{modelp4} \PY{o}{=} \PY{n}{sm}\PY{o}{.}\PY{n}{OLS}\PY{p}{(}\PY{n}{df1}\PY{p}{[}\PY{l+s+s2}{\PYZdq{}}\PY{l+s+s2}{p4}\PY{l+s+s2}{\PYZdq{}}\PY{p}{]}\PY{p}{,}\PY{n}{X}\PY{p}{)}\PY{o}{.}\PY{n}{fit}\PY{p}{(}\PY{p}{)}
\PY{n}{models1} \PY{o}{=} \PY{n}{sm}\PY{o}{.}\PY{n}{OLS}\PY{p}{(}\PY{n}{df1}\PY{p}{[}\PY{l+s+s2}{\PYZdq{}}\PY{l+s+s2}{s1}\PY{l+s+s2}{\PYZdq{}}\PY{p}{]}\PY{p}{,}\PY{n}{X}\PY{p}{)}\PY{o}{.}\PY{n}{fit}\PY{p}{(}\PY{p}{)}
\PY{n}{models2} \PY{o}{=} \PY{n}{sm}\PY{o}{.}\PY{n}{OLS}\PY{p}{(}\PY{n}{df1}\PY{p}{[}\PY{l+s+s2}{\PYZdq{}}\PY{l+s+s2}{s2}\PY{l+s+s2}{\PYZdq{}}\PY{p}{]}\PY{p}{,}\PY{n}{X}\PY{p}{)}\PY{o}{.}\PY{n}{fit}\PY{p}{(}\PY{p}{)}
\PY{n}{models3} \PY{o}{=} \PY{n}{sm}\PY{o}{.}\PY{n}{OLS}\PY{p}{(}\PY{n}{df1}\PY{p}{[}\PY{l+s+s2}{\PYZdq{}}\PY{l+s+s2}{s3}\PY{l+s+s2}{\PYZdq{}}\PY{p}{]}\PY{p}{,}\PY{n}{X}\PY{p}{)}\PY{o}{.}\PY{n}{fit}\PY{p}{(}\PY{p}{)}
\PY{n}{models4} \PY{o}{=} \PY{n}{sm}\PY{o}{.}\PY{n}{OLS}\PY{p}{(}\PY{n}{df1}\PY{p}{[}\PY{l+s+s2}{\PYZdq{}}\PY{l+s+s2}{s4}\PY{l+s+s2}{\PYZdq{}}\PY{p}{]}\PY{p}{,}\PY{n}{X}\PY{p}{)}\PY{o}{.}\PY{n}{fit}\PY{p}{(}\PY{p}{)}
\end{Verbatim}
\end{tcolorbox}

    \begin{tcolorbox}[breakable, size=fbox, boxrule=1pt, pad at break*=1mm,colback=cellbackground, colframe=cellborder]
\prompt{In}{incolor}{19}{\boxspacing}
\begin{Verbatim}[commandchars=\\\{\}]
\PY{n}{modelp1}\PY{o}{.}\PY{n}{rsquared\PYZus{}adj}\PY{p}{,} \PY{n}{modelp2}\PY{o}{.}\PY{n}{rsquared\PYZus{}adj}\PY{p}{,} \PY{n}{modelp3}\PY{o}{.}\PY{n}{rsquared\PYZus{}adj}\PY{p}{,} \PY{n}{modelp4}\PY{o}{.}\PY{n}{rsquared\PYZus{}adj}
\end{Verbatim}
\end{tcolorbox}

            \begin{tcolorbox}[breakable, size=fbox, boxrule=.5pt, pad at break*=1mm, opacityfill=0]
\prompt{Out}{outcolor}{19}{\boxspacing}
\begin{Verbatim}[commandchars=\\\{\}]
(0.9435061522877474,
 0.9480775331245905,
 0.9468682020346536,
 0.9477482549660268)
\end{Verbatim}
\end{tcolorbox}

    \begin{tcolorbox}[breakable, size=fbox, boxrule=1pt, pad at break*=1mm,colback=cellbackground, colframe=cellborder]
\prompt{In}{incolor}{20}{\boxspacing}
\begin{Verbatim}[commandchars=\\\{\}]
\PY{n}{models1}\PY{o}{.}\PY{n}{rsquared\PYZus{}adj}\PY{p}{,} \PY{n}{models2}\PY{o}{.}\PY{n}{rsquared\PYZus{}adj}\PY{p}{,} \PY{n}{models3}\PY{o}{.}\PY{n}{rsquared\PYZus{}adj}\PY{p}{,} \PY{n}{models4}\PY{o}{.}\PY{n}{rsquared\PYZus{}adj}
\end{Verbatim}
\end{tcolorbox}

            \begin{tcolorbox}[breakable, size=fbox, boxrule=.5pt, pad at break*=1mm, opacityfill=0]
\prompt{Out}{outcolor}{20}{\boxspacing}
\begin{Verbatim}[commandchars=\\\{\}]
(0.7644818350345643,
 0.7983742915490801,
 0.7648735858119549,
 0.7737755052449837)
\end{Verbatim}
\end{tcolorbox}

    \hypertarget{part-4}{%
\subsection{Part 4}\label{part-4}}

    \begin{tcolorbox}[breakable, size=fbox, boxrule=1pt, pad at break*=1mm,colback=cellbackground, colframe=cellborder]
\prompt{In}{incolor}{21}{\boxspacing}
\begin{Verbatim}[commandchars=\\\{\}]
\PY{n}{model\PYZus{}data} \PY{o}{=} \PY{n}{observed\PYZus{}data}\PY{o}{.}\PY{n}{copy}\PY{p}{(}\PY{p}{)}
\PY{n}{model\PYZus{}data}\PY{p}{[}\PY{l+s+s2}{\PYZdq{}}\PY{l+s+s2}{x\PYZus{}other}\PY{l+s+s2}{\PYZdq{}}\PY{p}{]} \PY{o}{=} \PY{n}{np}\PY{o}{.}\PY{n}{stack}\PY{p}{(}\PY{p}{[}
    \PY{n}{x}\PY{p}{[}\PY{l+m+mi}{1}\PY{p}{,}\PY{p}{:}\PY{p}{]}\PY{o}{+}\PY{n}{x}\PY{p}{[}\PY{l+m+mi}{2}\PY{p}{,}\PY{p}{:}\PY{p}{]}\PY{o}{+}\PY{n}{x}\PY{p}{[}\PY{l+m+mi}{3}\PY{p}{,}\PY{p}{:}\PY{p}{]}\PY{p}{,}
    \PY{n}{x}\PY{p}{[}\PY{l+m+mi}{0}\PY{p}{,}\PY{p}{:}\PY{p}{]}\PY{o}{+}\PY{n}{x}\PY{p}{[}\PY{l+m+mi}{2}\PY{p}{,}\PY{p}{:}\PY{p}{]}\PY{o}{+}\PY{n}{x}\PY{p}{[}\PY{l+m+mi}{3}\PY{p}{,}\PY{p}{:}\PY{p}{]}\PY{p}{,}
    \PY{n}{x}\PY{p}{[}\PY{l+m+mi}{0}\PY{p}{,}\PY{p}{:}\PY{p}{]}\PY{o}{+}\PY{n}{x}\PY{p}{[}\PY{l+m+mi}{1}\PY{p}{,}\PY{p}{:}\PY{p}{]}\PY{o}{+}\PY{n}{x}\PY{p}{[}\PY{l+m+mi}{3}\PY{p}{,}\PY{p}{:}\PY{p}{]}\PY{p}{,}
    \PY{n}{x}\PY{p}{[}\PY{l+m+mi}{0}\PY{p}{,}\PY{p}{:}\PY{p}{]}\PY{o}{+}\PY{n}{x}\PY{p}{[}\PY{l+m+mi}{1}\PY{p}{,}\PY{p}{:}\PY{p}{]}\PY{o}{+}\PY{n}{x}\PY{p}{[}\PY{l+m+mi}{2}\PY{p}{,}\PY{p}{:}\PY{p}{]}\PY{p}{]}\PY{p}{)}\PY{o}{.}\PY{n}{T}\PY{o}{.}\PY{n}{flatten}\PY{p}{(}\PY{p}{)}
\PY{n}{model\PYZus{}data}\PY{p}{[}\PY{l+s+s2}{\PYZdq{}}\PY{l+s+s2}{w\PYZus{}other}\PY{l+s+s2}{\PYZdq{}}\PY{p}{]} \PY{o}{=} \PY{n}{np}\PY{o}{.}\PY{n}{stack}\PY{p}{(}
    \PY{p}{[}\PY{n}{w}\PY{p}{[}\PY{l+m+mi}{1}\PY{p}{,}\PY{p}{:}\PY{p}{]}\PY{o}{+}\PY{n}{w}\PY{p}{[}\PY{l+m+mi}{2}\PY{p}{,}\PY{p}{:}\PY{p}{]}\PY{o}{+}\PY{n}{w}\PY{p}{[}\PY{l+m+mi}{3}\PY{p}{,}\PY{p}{:}\PY{p}{]}\PY{p}{,}
     \PY{n}{w}\PY{p}{[}\PY{l+m+mi}{0}\PY{p}{,}\PY{p}{:}\PY{p}{]}\PY{o}{+}\PY{n}{w}\PY{p}{[}\PY{l+m+mi}{2}\PY{p}{,}\PY{p}{:}\PY{p}{]}\PY{o}{+}\PY{n}{w}\PY{p}{[}\PY{l+m+mi}{3}\PY{p}{,}\PY{p}{:}\PY{p}{]}\PY{p}{,}
     \PY{n}{w}\PY{p}{[}\PY{l+m+mi}{0}\PY{p}{,}\PY{p}{:}\PY{p}{]}\PY{o}{+}\PY{n}{w}\PY{p}{[}\PY{l+m+mi}{1}\PY{p}{,}\PY{p}{:}\PY{p}{]}\PY{o}{+}\PY{n}{w}\PY{p}{[}\PY{l+m+mi}{3}\PY{p}{,}\PY{p}{:}\PY{p}{]}\PY{p}{,}
     \PY{n}{w}\PY{p}{[}\PY{l+m+mi}{0}\PY{p}{,}\PY{p}{:}\PY{p}{]}\PY{o}{+}\PY{n}{w}\PY{p}{[}\PY{l+m+mi}{1}\PY{p}{,}\PY{p}{:}\PY{p}{]}\PY{o}{+}\PY{n}{w}\PY{p}{[}\PY{l+m+mi}{2}\PY{p}{,}\PY{p}{:}\PY{p}{]}\PY{p}{]}\PY{p}{)}\PY{o}{.}\PY{n}{T}\PY{o}{.}\PY{n}{flatten}\PY{p}{(}\PY{p}{)}
\end{Verbatim}
\end{tcolorbox}

    \begin{tcolorbox}[breakable, size=fbox, boxrule=1pt, pad at break*=1mm,colback=cellbackground, colframe=cellborder]
\prompt{In}{incolor}{22}{\boxspacing}
\begin{Verbatim}[commandchars=\\\{\}]
\PY{c+c1}{\PYZsh{} 4A: Logit}
\PY{n}{outside\PYZus{}shares} \PY{o}{=} \PY{l+m+mi}{1} \PY{o}{\PYZhy{}} \PY{n}{np}\PY{o}{.}\PY{n}{sum}\PY{p}{(}\PY{n}{eq\PYZus{}shares}\PY{p}{,} \PY{n}{axis}\PY{o}{=}\PY{l+m+mi}{0}\PY{p}{,} \PY{n}{keepdims}\PY{o}{=}\PY{k+kc}{True}\PY{p}{)}
\PY{n}{y} \PY{o}{=} \PY{n}{np}\PY{o}{.}\PY{n}{log}\PY{p}{(}\PY{n}{eq\PYZus{}shares}\PY{o}{/}\PY{n}{outside\PYZus{}shares}\PY{p}{)}\PY{o}{.}\PY{n}{T}\PY{o}{.}\PY{n}{flatten}\PY{p}{(}\PY{p}{)}
\PY{n}{X} \PY{o}{=} \PY{n}{model\PYZus{}data}\PY{p}{[}\PY{p}{[}\PY{l+s+s2}{\PYZdq{}}\PY{l+s+s2}{x}\PY{l+s+s2}{\PYZdq{}}\PY{p}{,}\PY{l+s+s2}{\PYZdq{}}\PY{l+s+s2}{satellite}\PY{l+s+s2}{\PYZdq{}}\PY{p}{,}\PY{l+s+s2}{\PYZdq{}}\PY{l+s+s2}{wired}\PY{l+s+s2}{\PYZdq{}}\PY{p}{,}\PY{l+s+s2}{\PYZdq{}}\PY{l+s+s2}{prices}\PY{l+s+s2}{\PYZdq{}}\PY{p}{]}\PY{p}{]}
\PY{n}{results} \PY{o}{=} \PY{n}{sm}\PY{o}{.}\PY{n}{OLS}\PY{p}{(}\PY{n}{y}\PY{p}{,}\PY{n}{X}\PY{p}{)}\PY{o}{.}\PY{n}{fit}\PY{p}{(}\PY{p}{)}
\PY{n}{results}\PY{o}{.}\PY{n}{summary}\PY{p}{(}\PY{p}{)}
\end{Verbatim}
\end{tcolorbox}

            \begin{tcolorbox}[breakable, size=fbox, boxrule=.5pt, pad at break*=1mm, opacityfill=0]
\prompt{Out}{outcolor}{22}{\boxspacing}
\begin{Verbatim}[commandchars=\\\{\}]
<class 'statsmodels.iolib.summary.Summary'>
"""
                            OLS Regression Results
==============================================================================
Dep. Variable:                      y   R-squared:                       0.314
Model:                            OLS   Adj. R-squared:                  0.313
Method:                 Least Squares   F-statistic:                     365.4
Date:                Wed, 13 Oct 2021   Prob (F-statistic):          2.09e-195
Time:                        21:30:31   Log-Likelihood:                -3033.1
No. Observations:                2400   AIC:                             6074.
Df Residuals:                    2396   BIC:                             6097.
Df Model:                           3
Covariance Type:            nonrobust
==============================================================================
                 coef    std err          t      P>|t|      [0.025      0.975]
------------------------------------------------------------------------------
x              0.8375      0.029     28.572      0.000       0.780       0.895
satellite      1.3705      0.122     11.239      0.000       1.131       1.610
wired          1.3589      0.123     11.046      0.000       1.118       1.600
prices        -0.9518      0.044    -21.393      0.000      -1.039      -0.865
==============================================================================
Omnibus:                       41.828   Durbin-Watson:                   2.047
Prob(Omnibus):                  0.000   Jarque-Bera (JB):               48.815
Skew:                          -0.263   Prob(JB):                     2.51e-11
Kurtosis:                       3.460   Cond. No.                         30.0
==============================================================================

Notes:
[1] Standard Errors assume that the covariance matrix of the errors is correctly
specified.
"""
\end{Verbatim}
\end{tcolorbox}

    Note that ignoring the endogeneity of prices results in underestimating
the magnitudes of all the relevant parameters.

    \begin{tcolorbox}[breakable, size=fbox, boxrule=1pt, pad at break*=1mm,colback=cellbackground, colframe=cellborder]
\prompt{In}{incolor}{23}{\boxspacing}
\begin{Verbatim}[commandchars=\\\{\}]
\PY{c+c1}{\PYZsh{}6: IV\PYZhy{}2SLS}
\PY{n}{X\PYZus{}exog} \PY{o}{=} \PY{n}{model\PYZus{}data}\PY{p}{[}\PY{p}{[}\PY{l+s+s2}{\PYZdq{}}\PY{l+s+s2}{x}\PY{l+s+s2}{\PYZdq{}}\PY{p}{,} \PY{l+s+s2}{\PYZdq{}}\PY{l+s+s2}{satellite}\PY{l+s+s2}{\PYZdq{}}\PY{p}{,} \PY{l+s+s2}{\PYZdq{}}\PY{l+s+s2}{wired}\PY{l+s+s2}{\PYZdq{}}\PY{p}{]}\PY{p}{]}\PY{o}{.}\PY{n}{astype}\PY{p}{(}\PY{n+nb}{float}\PY{p}{)}
\PY{n}{X\PYZus{}endog} \PY{o}{=} \PY{n}{model\PYZus{}data}\PY{p}{[}\PY{p}{[}\PY{l+s+s2}{\PYZdq{}}\PY{l+s+s2}{prices}\PY{l+s+s2}{\PYZdq{}}\PY{p}{]}\PY{p}{]}\PY{o}{.}\PY{n}{astype}\PY{p}{(}\PY{n+nb}{float}\PY{p}{)}
\PY{n}{Z} \PY{o}{=} \PY{n}{model\PYZus{}data}\PY{p}{[}\PY{p}{[}\PY{l+s+s2}{\PYZdq{}}\PY{l+s+s2}{w}\PY{l+s+s2}{\PYZdq{}}\PY{p}{,} \PY{l+s+s2}{\PYZdq{}}\PY{l+s+s2}{x\PYZus{}other}\PY{l+s+s2}{\PYZdq{}}\PY{p}{,} \PY{l+s+s2}{\PYZdq{}}\PY{l+s+s2}{w\PYZus{}other}\PY{l+s+s2}{\PYZdq{}}\PY{p}{]}\PY{p}{]}\PY{o}{.}\PY{n}{astype}\PY{p}{(}\PY{n+nb}{float}\PY{p}{)}
\PY{c+c1}{\PYZsh{} we\PYZsq{}ll just instrument for prices with w and x of the own\PYZhy{}product; it seems good enough here}
\PY{n}{iv\PYZus{}model} \PY{o}{=} \PY{n}{IV2SLS}\PY{p}{(}\PY{n}{y}\PY{p}{,} \PY{n}{X\PYZus{}exog}\PY{p}{,} \PY{n}{X\PYZus{}endog}\PY{p}{,} \PY{n}{Z}\PY{p}{)}\PY{o}{.}\PY{n}{fit}\PY{p}{(}\PY{p}{)}
\PY{n}{iv\PYZus{}model}\PY{o}{.}\PY{n}{summary}
\end{Verbatim}
\end{tcolorbox}

            \begin{tcolorbox}[breakable, size=fbox, boxrule=.5pt, pad at break*=1mm, opacityfill=0]
\prompt{Out}{outcolor}{23}{\boxspacing}
\begin{Verbatim}[commandchars=\\\{\}]
<class 'linearmodels.compat.statsmodels.Summary'>
"""
                          IV-2SLS Estimation Summary
==============================================================================
Dep. Variable:              dependent   R-squared:                      0.1841
Estimator:                    IV-2SLS   Adj. R-squared:                 0.1831
No. Observations:                2400   F-statistic:                    2007.7
Date:                Wed, Oct 13 2021   P-value (F-stat)                0.0000
Time:                        21:30:31   Distribution:                  chi2(4)
Cov. Estimator:                robust

                             Parameter Estimates
==============================================================================
            Parameter  Std. Err.     T-stat    P-value    Lower CI    Upper CI
------------------------------------------------------------------------------
x              0.9461     0.0331     28.609     0.0000      0.8813      1.0109
satellite      3.8635     0.1878     20.575     0.0000      3.4955      4.2315
wired          3.8774     0.1880     20.628     0.0000      3.5090      4.2458
prices        -1.8992     0.0700    -27.132     0.0000     -2.0364     -1.7620
==============================================================================

Endogenous: prices
Instruments: w, x\_other, w\_other
Robust Covariance (Heteroskedastic)
Debiased: False
"""
\end{Verbatim}
\end{tcolorbox}

    \begin{tcolorbox}[breakable, size=fbox, boxrule=1pt, pad at break*=1mm,colback=cellbackground, colframe=cellborder]
\prompt{In}{incolor}{24}{\boxspacing}
\begin{Verbatim}[commandchars=\\\{\}]
\PY{c+c1}{\PYZsh{}4.7 nested logit}

\PY{c+c1}{\PYZsh{} construct log of within group share}
\PY{n}{satellite\PYZus{}share}\PY{o}{=} \PY{n}{eq\PYZus{}shares}\PY{p}{[}\PY{l+m+mi}{0}\PY{p}{,}\PY{p}{:}\PY{p}{]} \PY{o}{+} \PY{n}{eq\PYZus{}shares}\PY{p}{[}\PY{l+m+mi}{1}\PY{p}{,}\PY{p}{:}\PY{p}{]}
\PY{n}{wired\PYZus{}share}\PY{o}{=} \PY{n}{eq\PYZus{}shares}\PY{p}{[}\PY{l+m+mi}{2}\PY{p}{,}\PY{p}{:}\PY{p}{]} \PY{o}{+} \PY{n}{eq\PYZus{}shares}\PY{p}{[}\PY{l+m+mi}{3}\PY{p}{,}\PY{p}{:}\PY{p}{]}
\PY{n}{model\PYZus{}data}\PY{p}{[}\PY{l+s+s2}{\PYZdq{}}\PY{l+s+s2}{within\PYZus{}satellite\PYZus{}shares}\PY{l+s+s2}{\PYZdq{}}\PY{p}{]} \PY{o}{=} \PY{n}{model\PYZus{}data}\PY{p}{[}\PY{l+s+s2}{\PYZdq{}}\PY{l+s+s2}{satellite}\PY{l+s+s2}{\PYZdq{}}\PY{p}{]}\PY{o}{*}\PY{n}{np}\PY{o}{.}\PY{n}{log}\PY{p}{(}\PY{n}{eq\PYZus{}shares} \PY{o}{/} \PY{n}{satellite\PYZus{}share}\PY{p}{)}\PY{o}{.}\PY{n}{T}\PY{o}{.}\PY{n}{flatten}\PY{p}{(}\PY{p}{)}
\PY{n}{model\PYZus{}data}\PY{p}{[}\PY{l+s+s2}{\PYZdq{}}\PY{l+s+s2}{within\PYZus{}wired\PYZus{}shares}\PY{l+s+s2}{\PYZdq{}}\PY{p}{]} \PY{o}{=} \PY{n}{model\PYZus{}data}\PY{p}{[}\PY{l+s+s2}{\PYZdq{}}\PY{l+s+s2}{wired}\PY{l+s+s2}{\PYZdq{}}\PY{p}{]}\PY{o}{*}\PY{n}{np}\PY{o}{.}\PY{n}{log}\PY{p}{(}\PY{n}{eq\PYZus{}shares} \PY{o}{/} \PY{n}{wired\PYZus{}share}\PY{p}{)}\PY{o}{.}\PY{n}{T}\PY{o}{.}\PY{n}{flatten}\PY{p}{(}\PY{p}{)}
\PY{n}{model\PYZus{}data}\PY{p}{[}\PY{l+s+s2}{\PYZdq{}}\PY{l+s+s2}{within\PYZus{}group\PYZus{}shares}\PY{l+s+s2}{\PYZdq{}}\PY{p}{]} \PY{o}{=} \PY{n}{model\PYZus{}data}\PY{p}{[}\PY{l+s+s2}{\PYZdq{}}\PY{l+s+s2}{within\PYZus{}wired\PYZus{}shares}\PY{l+s+s2}{\PYZdq{}}\PY{p}{]} \PY{o}{+} \PY{n}{model\PYZus{}data}\PY{p}{[}\PY{l+s+s2}{\PYZdq{}}\PY{l+s+s2}{within\PYZus{}satellite\PYZus{}shares}\PY{l+s+s2}{\PYZdq{}}\PY{p}{]}

\PY{c+c1}{\PYZsh{} now use the other in\PYZhy{}group firm\PYZsq{}s characteristics as instruments}
\PY{n}{model\PYZus{}data}\PY{p}{[}\PY{l+s+s2}{\PYZdq{}}\PY{l+s+s2}{x\PYZus{}other\PYZus{}satellite}\PY{l+s+s2}{\PYZdq{}}\PY{p}{]} \PY{o}{=} \PY{n}{np}\PY{o}{.}\PY{n}{stack}\PY{p}{(}\PY{p}{[}\PY{n}{x}\PY{p}{[}\PY{l+m+mi}{1}\PY{p}{,}\PY{p}{:}\PY{p}{]}\PY{p}{,} \PY{n}{x}\PY{p}{[}\PY{l+m+mi}{0}\PY{p}{,}\PY{p}{:}\PY{p}{]}\PY{p}{,} \PY{n}{x}\PY{p}{[}\PY{l+m+mi}{3}\PY{p}{,}\PY{p}{:}\PY{p}{]}\PY{p}{,} \PY{n}{x}\PY{p}{[}\PY{l+m+mi}{2}\PY{p}{,}\PY{p}{:}\PY{p}{]}\PY{p}{]}\PY{p}{)}\PY{o}{.}\PY{n}{T}\PY{o}{.}\PY{n}{flatten}\PY{p}{(}\PY{p}{)}\PY{o}{*}\PY{n}{model\PYZus{}data}\PY{p}{[}\PY{l+s+s2}{\PYZdq{}}\PY{l+s+s2}{satellite}\PY{l+s+s2}{\PYZdq{}}\PY{p}{]}
\PY{n}{model\PYZus{}data}\PY{p}{[}\PY{l+s+s2}{\PYZdq{}}\PY{l+s+s2}{w\PYZus{}other\PYZus{}satellite}\PY{l+s+s2}{\PYZdq{}}\PY{p}{]} \PY{o}{=} \PY{n}{np}\PY{o}{.}\PY{n}{stack}\PY{p}{(}\PY{p}{[}\PY{n}{w}\PY{p}{[}\PY{l+m+mi}{1}\PY{p}{,}\PY{p}{:}\PY{p}{]}\PY{p}{,} \PY{n}{w}\PY{p}{[}\PY{l+m+mi}{0}\PY{p}{,}\PY{p}{:}\PY{p}{]}\PY{p}{,} \PY{n}{w}\PY{p}{[}\PY{l+m+mi}{3}\PY{p}{,}\PY{p}{:}\PY{p}{]}\PY{p}{,} \PY{n}{w}\PY{p}{[}\PY{l+m+mi}{2}\PY{p}{,}\PY{p}{:}\PY{p}{]}\PY{p}{]}\PY{p}{)}\PY{o}{.}\PY{n}{T}\PY{o}{.}\PY{n}{flatten}\PY{p}{(}\PY{p}{)}\PY{o}{*}\PY{n}{model\PYZus{}data}\PY{p}{[}\PY{l+s+s2}{\PYZdq{}}\PY{l+s+s2}{satellite}\PY{l+s+s2}{\PYZdq{}}\PY{p}{]}
\PY{n}{model\PYZus{}data}\PY{p}{[}\PY{l+s+s2}{\PYZdq{}}\PY{l+s+s2}{x\PYZus{}other\PYZus{}wired}\PY{l+s+s2}{\PYZdq{}}\PY{p}{]} \PY{o}{=} \PY{n}{np}\PY{o}{.}\PY{n}{stack}\PY{p}{(}\PY{p}{[}\PY{n}{x}\PY{p}{[}\PY{l+m+mi}{1}\PY{p}{,}\PY{p}{:}\PY{p}{]}\PY{p}{,} \PY{n}{x}\PY{p}{[}\PY{l+m+mi}{0}\PY{p}{,}\PY{p}{:}\PY{p}{]}\PY{p}{,} \PY{n}{x}\PY{p}{[}\PY{l+m+mi}{3}\PY{p}{,}\PY{p}{:}\PY{p}{]}\PY{p}{,} \PY{n}{x}\PY{p}{[}\PY{l+m+mi}{2}\PY{p}{,}\PY{p}{:}\PY{p}{]}\PY{p}{]}\PY{p}{)}\PY{o}{.}\PY{n}{T}\PY{o}{.}\PY{n}{flatten}\PY{p}{(}\PY{p}{)}\PY{o}{*}\PY{n}{model\PYZus{}data}\PY{p}{[}\PY{l+s+s2}{\PYZdq{}}\PY{l+s+s2}{wired}\PY{l+s+s2}{\PYZdq{}}\PY{p}{]}
\PY{n}{model\PYZus{}data}\PY{p}{[}\PY{l+s+s2}{\PYZdq{}}\PY{l+s+s2}{w\PYZus{}other\PYZus{}wired}\PY{l+s+s2}{\PYZdq{}}\PY{p}{]} \PY{o}{=} \PY{n}{np}\PY{o}{.}\PY{n}{stack}\PY{p}{(}\PY{p}{[}\PY{n}{w}\PY{p}{[}\PY{l+m+mi}{1}\PY{p}{,}\PY{p}{:}\PY{p}{]}\PY{p}{,} \PY{n}{w}\PY{p}{[}\PY{l+m+mi}{0}\PY{p}{,}\PY{p}{:}\PY{p}{]}\PY{p}{,} \PY{n}{w}\PY{p}{[}\PY{l+m+mi}{3}\PY{p}{,}\PY{p}{:}\PY{p}{]}\PY{p}{,} \PY{n}{w}\PY{p}{[}\PY{l+m+mi}{2}\PY{p}{,}\PY{p}{:}\PY{p}{]}\PY{p}{]}\PY{p}{)}\PY{o}{.}\PY{n}{T}\PY{o}{.}\PY{n}{flatten}\PY{p}{(}\PY{p}{)}\PY{o}{*}\PY{n}{model\PYZus{}data}\PY{p}{[}\PY{l+s+s2}{\PYZdq{}}\PY{l+s+s2}{wired}\PY{l+s+s2}{\PYZdq{}}\PY{p}{]}

\PY{n}{X\PYZus{}exog} \PY{o}{=} \PY{n}{model\PYZus{}data}\PY{p}{[}\PY{p}{[}\PY{l+s+s2}{\PYZdq{}}\PY{l+s+s2}{x}\PY{l+s+s2}{\PYZdq{}}\PY{p}{,}\PY{l+s+s2}{\PYZdq{}}\PY{l+s+s2}{satellite}\PY{l+s+s2}{\PYZdq{}}\PY{p}{,}\PY{l+s+s2}{\PYZdq{}}\PY{l+s+s2}{wired}\PY{l+s+s2}{\PYZdq{}}\PY{p}{]}\PY{p}{]}
\PY{n}{X\PYZus{}endog} \PY{o}{=} \PY{n}{model\PYZus{}data}\PY{p}{[}\PY{p}{[}\PY{l+s+s2}{\PYZdq{}}\PY{l+s+s2}{prices}\PY{l+s+s2}{\PYZdq{}}\PY{p}{,} \PY{l+s+s2}{\PYZdq{}}\PY{l+s+s2}{within\PYZus{}satellite\PYZus{}shares}\PY{l+s+s2}{\PYZdq{}}\PY{p}{,} \PY{l+s+s2}{\PYZdq{}}\PY{l+s+s2}{within\PYZus{}wired\PYZus{}shares}\PY{l+s+s2}{\PYZdq{}} \PY{p}{]}\PY{p}{]}
\PY{n}{Z} \PY{o}{=} \PY{n}{model\PYZus{}data}\PY{p}{[}\PY{p}{[}\PY{l+s+s2}{\PYZdq{}}\PY{l+s+s2}{w}\PY{l+s+s2}{\PYZdq{}}\PY{p}{,} \PY{l+s+s2}{\PYZdq{}}\PY{l+s+s2}{x\PYZus{}other}\PY{l+s+s2}{\PYZdq{}}\PY{p}{,} \PY{l+s+s2}{\PYZdq{}}\PY{l+s+s2}{w\PYZus{}other}\PY{l+s+s2}{\PYZdq{}}\PY{p}{,} \PY{l+s+s2}{\PYZdq{}}\PY{l+s+s2}{x\PYZus{}other\PYZus{}satellite}\PY{l+s+s2}{\PYZdq{}}\PY{p}{,} \PY{l+s+s2}{\PYZdq{}}\PY{l+s+s2}{w\PYZus{}other\PYZus{}satellite}\PY{l+s+s2}{\PYZdq{}}\PY{p}{,} \PY{l+s+s2}{\PYZdq{}}\PY{l+s+s2}{x\PYZus{}other\PYZus{}wired}\PY{l+s+s2}{\PYZdq{}}\PY{p}{,} \PY{l+s+s2}{\PYZdq{}}\PY{l+s+s2}{w\PYZus{}other\PYZus{}wired}\PY{l+s+s2}{\PYZdq{}}\PY{p}{]}\PY{p}{]}
\PY{n}{iv\PYZus{}model} \PY{o}{=} \PY{n}{IV2SLS}\PY{p}{(}\PY{n}{y}\PY{p}{,} \PY{n}{X\PYZus{}exog}\PY{p}{,} \PY{n}{X\PYZus{}endog}\PY{p}{,} \PY{n}{Z}\PY{p}{)}\PY{o}{.}\PY{n}{fit}\PY{p}{(}\PY{p}{)}
\PY{n}{iv\PYZus{}model}\PY{o}{.}\PY{n}{summary}
\end{Verbatim}
\end{tcolorbox}

            \begin{tcolorbox}[breakable, size=fbox, boxrule=.5pt, pad at break*=1mm, opacityfill=0]
\prompt{Out}{outcolor}{24}{\boxspacing}
\begin{Verbatim}[commandchars=\\\{\}]
<class 'linearmodels.compat.statsmodels.Summary'>
"""
                          IV-2SLS Estimation Summary
==============================================================================
Dep. Variable:              dependent   R-squared:                      0.3589
Estimator:                    IV-2SLS   Adj. R-squared:                 0.3576
No. Observations:                2400   F-statistic:                    2635.4
Date:                Wed, Oct 13 2021   P-value (F-stat)                0.0000
Time:                        21:30:31   Distribution:                  chi2(6)
Cov. Estimator:                robust

                                    Parameter Estimates
================================================================================
===========
                         Parameter  Std. Err.     T-stat    P-value    Lower CI
Upper CI
--------------------------------------------------------------------------------
-----------
x                           0.8483     0.0377     22.479     0.0000      0.7743
0.9223
satellite                   3.4995     0.1923     18.200     0.0000      3.1226
3.8763
wired                       3.4881     0.1825     19.111     0.0000      3.1303
3.8458
prices                     -1.6649     0.0784    -21.241     0.0000     -1.8185
-1.5112
within\_satellite\_shares     0.2173     0.0770     2.8237     0.0047      0.0665
0.3681
within\_wired\_shares         0.1944     0.0714     2.7237     0.0065      0.0545
0.3342
================================================================================
===========

Endogenous: prices, within\_satellite\_shares, within\_wired\_shares
Instruments: w, x\_other, w\_other, x\_other\_satellite, w\_other\_satellite,
x\_other\_wired, w\_other\_wired
Robust Covariance (Heteroskedastic)
Debiased: False
"""
\end{Verbatim}
\end{tcolorbox}

    \begin{tcolorbox}[breakable, size=fbox, boxrule=1pt, pad at break*=1mm,colback=cellbackground, colframe=cellborder]
\prompt{In}{incolor}{25}{\boxspacing}
\begin{Verbatim}[commandchars=\\\{\}]
\PY{c+c1}{\PYZsh{} define functions for derivatives and shares in the nested logit}

\PY{k}{def} \PY{n+nf}{full\PYZus{}mkt\PYZus{}share\PYZus{}derivative\PYZus{}nested}\PY{p}{(}\PY{n}{t}\PY{p}{,} \PY{n}{p}\PY{p}{,} \PY{n}{pars}\PY{p}{)}\PY{p}{:}

    \PY{n}{XX} \PY{o}{=}  \PY{p}{[}\PY{n}{x}\PY{p}{[}\PY{p}{:}\PY{p}{,}\PY{n}{t}\PY{p}{]}\PY{p}{,} \PY{p}{[}\PY{l+m+mi}{1}\PY{p}{,}\PY{l+m+mi}{1}\PY{p}{,}\PY{l+m+mi}{0}\PY{p}{,}\PY{l+m+mi}{0}\PY{p}{]}\PY{p}{,} \PY{p}{[}\PY{l+m+mi}{0}\PY{p}{,}\PY{l+m+mi}{0}\PY{p}{,}\PY{l+m+mi}{1}\PY{p}{,}\PY{l+m+mi}{1}\PY{p}{]}\PY{p}{,} \PY{n}{p}  \PY{p}{]}

    \PY{n}{v\PYZus{}t} \PY{o}{=} \PY{n}{pars}\PY{p}{[}\PY{l+m+mi}{0}\PY{p}{:}\PY{l+m+mi}{4}\PY{p}{]}  \PY{o}{@} \PY{n}{XX}

    \PY{n}{sigma\PYZus{}1} \PY{o}{=} \PY{n}{pars}\PY{p}{[}\PY{l+m+mi}{4}\PY{p}{]}

    \PY{n}{sigma\PYZus{}2} \PY{o}{=} \PY{n}{pars}\PY{p}{[}\PY{l+m+mi}{5}\PY{p}{]}

    \PY{n}{theta\PYZus{}2} \PY{o}{=} \PY{n}{pars}\PY{p}{[}\PY{l+m+mi}{3}\PY{p}{]}


    \PY{n}{D1} \PY{o}{=} \PY{n}{np}\PY{o}{.}\PY{n}{exp}\PY{p}{(}\PY{n}{v\PYZus{}t}\PY{p}{[}\PY{l+m+mi}{0}\PY{p}{]}\PY{o}{/}\PY{p}{(}\PY{l+m+mi}{1}\PY{o}{\PYZhy{}} \PY{n}{sigma\PYZus{}1}\PY{p}{)}\PY{p}{)} \PY{o}{+} \PY{n}{np}\PY{o}{.}\PY{n}{exp}\PY{p}{(}\PY{n}{v\PYZus{}t}\PY{p}{[}\PY{l+m+mi}{1}\PY{p}{]}\PY{o}{/}\PY{p}{(}\PY{l+m+mi}{1}\PY{o}{\PYZhy{}} \PY{n}{sigma\PYZus{}1}\PY{p}{)}\PY{p}{)}

    \PY{n}{D2} \PY{o}{=} \PY{n}{np}\PY{o}{.}\PY{n}{exp}\PY{p}{(}\PY{n}{v\PYZus{}t}\PY{p}{[}\PY{l+m+mi}{2}\PY{p}{]}\PY{o}{/}\PY{p}{(}\PY{l+m+mi}{1}\PY{o}{\PYZhy{}} \PY{n}{sigma\PYZus{}2}\PY{p}{)}\PY{p}{)} \PY{o}{+} \PY{n}{np}\PY{o}{.}\PY{n}{exp}\PY{p}{(}\PY{n}{v\PYZus{}t}\PY{p}{[}\PY{l+m+mi}{3}\PY{p}{]}\PY{o}{/}\PY{p}{(}\PY{l+m+mi}{1}\PY{o}{\PYZhy{}} \PY{n}{sigma\PYZus{}2}\PY{p}{)}\PY{p}{)}

    \PY{n}{Z} \PY{o}{=} \PY{l+m+mi}{1} \PY{o}{+} \PY{n}{np}\PY{o}{.}\PY{n}{power}\PY{p}{(}\PY{n}{D1}\PY{p}{,} \PY{p}{(}\PY{l+m+mi}{1}\PY{o}{\PYZhy{}} \PY{n}{sigma\PYZus{}1}\PY{p}{)}\PY{p}{)} \PY{o}{+} \PY{n}{np}\PY{o}{.}\PY{n}{power}\PY{p}{(}\PY{n}{D2}\PY{p}{,} \PY{p}{(}\PY{l+m+mi}{1}\PY{o}{\PYZhy{}} \PY{n}{sigma\PYZus{}2}\PY{p}{)}\PY{p}{)}


    \PY{n}{derivatives} \PY{o}{=} \PY{n}{np}\PY{o}{.}\PY{n}{zeros}\PY{p}{(}\PY{p}{(}\PY{n}{J}\PY{p}{,}\PY{n}{J}\PY{p}{)}\PY{p}{)}

    \PY{k}{for} \PY{n}{j} \PY{o+ow}{in} \PY{n+nb}{range}\PY{p}{(}\PY{n}{J}\PY{p}{)}\PY{p}{:}
        \PY{k}{if} \PY{n}{j} \PY{o}{\PYZlt{}} \PY{l+m+mi}{2}\PY{p}{:}
            \PY{n}{derivatives}\PY{p}{[}\PY{n}{j}\PY{p}{,}\PY{n}{j}\PY{p}{]} \PY{o}{=} \PY{p}{(}\PY{n}{theta\PYZus{}2}\PY{o}{/}\PY{p}{(}\PY{l+m+mi}{1} \PY{o}{\PYZhy{}} \PY{n}{sigma\PYZus{}1}\PY{p}{)}\PY{p}{)}\PY{o}{*}\PY{p}{(} \PY{n}{np}\PY{o}{.}\PY{n}{exp}\PY{p}{(}\PY{n}{v\PYZus{}t}\PY{p}{[}\PY{n}{j}\PY{p}{]}\PY{o}{/}\PY{p}{(}\PY{l+m+mi}{1}\PY{o}{\PYZhy{}} \PY{n}{sigma\PYZus{}1}\PY{p}{)}\PY{p}{)}\PY{o}{*}\PY{n}{np}\PY{o}{.}\PY{n}{power}\PY{p}{(}\PY{n}{D1}\PY{p}{,} \PY{n}{sigma\PYZus{}1}\PY{p}{)}\PY{o}{*}\PY{n}{Z} \PY{o}{\PYZhy{}}  \PY{n}{np}\PY{o}{.}\PY{n}{exp}\PY{p}{(}\PY{p}{(}\PY{l+m+mi}{2}\PY{o}{*}\PY{n}{v\PYZus{}t}\PY{p}{[}\PY{n}{j}\PY{p}{]}\PY{p}{)}\PY{o}{/}\PY{p}{(}\PY{l+m+mi}{1}\PY{o}{\PYZhy{}} \PY{n}{sigma\PYZus{}1}\PY{p}{)}\PY{p}{)}\PY{o}{*}\PY{p}{(}   \PY{n}{sigma\PYZus{}1}\PY{o}{*}\PY{n}{np}\PY{o}{.}\PY{n}{power}\PY{p}{(}\PY{n}{D1}\PY{p}{,} \PY{n}{sigma\PYZus{}1} \PY{o}{\PYZhy{}}\PY{l+m+mi}{1}\PY{p}{)}\PY{o}{*}\PY{n}{Z} \PY{o}{+} \PY{p}{(}\PY{l+m+mi}{1}\PY{o}{\PYZhy{}} \PY{n}{sigma\PYZus{}1}\PY{p}{)}\PY{p}{)} \PY{p}{)}  \PY{o}{/} \PY{n}{np}\PY{o}{.}\PY{n}{square}\PY{p}{(}\PY{p}{(}\PY{n}{np}\PY{o}{.}\PY{n}{power}\PY{p}{(}\PY{n}{D1}\PY{p}{,} \PY{n}{sigma\PYZus{}1}\PY{p}{)}\PY{o}{*}\PY{n}{Z}\PY{p}{)}\PY{p}{)}
        \PY{k}{else}\PY{p}{:}
            \PY{n}{derivatives}\PY{p}{[}\PY{n}{j}\PY{p}{,}\PY{n}{j}\PY{p}{]} \PY{o}{=} \PY{p}{(}\PY{n}{theta\PYZus{}2}\PY{o}{/}\PY{p}{(}\PY{l+m+mi}{1} \PY{o}{\PYZhy{}} \PY{n}{sigma\PYZus{}2}\PY{p}{)}\PY{p}{)}\PY{o}{*}\PY{p}{(} \PY{n}{np}\PY{o}{.}\PY{n}{exp}\PY{p}{(}\PY{n}{v\PYZus{}t}\PY{p}{[}\PY{n}{j}\PY{p}{]}\PY{o}{/}\PY{p}{(}\PY{l+m+mi}{1}\PY{o}{\PYZhy{}} \PY{n}{sigma\PYZus{}2}\PY{p}{)}\PY{p}{)}\PY{o}{*}\PY{n}{np}\PY{o}{.}\PY{n}{power}\PY{p}{(}\PY{n}{D2}\PY{p}{,} \PY{n}{sigma\PYZus{}2}\PY{p}{)}\PY{o}{*}\PY{n}{Z} \PY{o}{\PYZhy{}}  \PY{n}{np}\PY{o}{.}\PY{n}{exp}\PY{p}{(}\PY{p}{(}\PY{l+m+mi}{2}\PY{o}{*}\PY{n}{v\PYZus{}t}\PY{p}{[}\PY{n}{j}\PY{p}{]}\PY{p}{)}\PY{o}{/}\PY{p}{(}\PY{l+m+mi}{1}\PY{o}{\PYZhy{}} \PY{n}{sigma\PYZus{}2}\PY{p}{)}\PY{p}{)}\PY{o}{*}\PY{p}{(}   \PY{n}{sigma\PYZus{}2}\PY{o}{*}\PY{n}{np}\PY{o}{.}\PY{n}{power}\PY{p}{(}\PY{n}{D2}\PY{p}{,} \PY{n}{sigma\PYZus{}2} \PY{o}{\PYZhy{}}\PY{l+m+mi}{1}\PY{p}{)}\PY{o}{*}\PY{n}{Z} \PY{o}{+} \PY{p}{(}\PY{l+m+mi}{1}\PY{o}{\PYZhy{}} \PY{n}{sigma\PYZus{}2}\PY{p}{)}\PY{p}{)} \PY{p}{)}  \PY{o}{/} \PY{n}{np}\PY{o}{.}\PY{n}{square}\PY{p}{(}\PY{p}{(}\PY{n}{np}\PY{o}{.}\PY{n}{power}\PY{p}{(}\PY{n}{D2}\PY{p}{,} \PY{n}{sigma\PYZus{}2}\PY{p}{)}\PY{o}{*}\PY{n}{Z}\PY{p}{)}\PY{p}{)}


    \PY{k}{for} \PY{n}{j} \PY{o+ow}{in} \PY{n+nb}{range}\PY{p}{(}\PY{n}{J}\PY{p}{)}\PY{p}{:}
        \PY{k}{for} \PY{n}{k} \PY{o+ow}{in} \PY{n+nb}{range}\PY{p}{(}\PY{n}{J}\PY{p}{)}\PY{p}{:}
            \PY{k}{if} \PY{o+ow}{not} \PY{p}{(}\PY{n}{j} \PY{o}{==} \PY{n}{k}\PY{p}{)}\PY{p}{:}
                \PY{k}{if} \PY{n}{j} \PY{o}{\PYZlt{}} \PY{l+m+mi}{2} \PY{o+ow}{and} \PY{n}{k} \PY{o}{\PYZlt{}} \PY{l+m+mi}{2}\PY{p}{:}
                    \PY{n}{derivatives}\PY{p}{[}\PY{n}{j}\PY{p}{,}\PY{n}{k}\PY{p}{]} \PY{o}{=} \PY{p}{(}\PY{o}{\PYZhy{}}\PY{n}{theta\PYZus{}2}\PY{o}{/}\PY{p}{(}\PY{l+m+mi}{1} \PY{o}{\PYZhy{}} \PY{n}{sigma\PYZus{}1}\PY{p}{)}\PY{p}{)}\PY{o}{*}\PY{n}{np}\PY{o}{.}\PY{n}{exp}\PY{p}{(}\PY{n}{v\PYZus{}t}\PY{p}{[}\PY{n}{j}\PY{p}{]}\PY{o}{/}\PY{p}{(}\PY{l+m+mi}{1}\PY{o}{\PYZhy{}} \PY{n}{sigma\PYZus{}1}\PY{p}{)}\PY{p}{)}\PY{o}{*}\PY{n}{np}\PY{o}{.}\PY{n}{exp}\PY{p}{(}\PY{n}{v\PYZus{}t}\PY{p}{[}\PY{n}{k}\PY{p}{]}\PY{o}{/}\PY{p}{(}\PY{l+m+mi}{1}\PY{o}{\PYZhy{}} \PY{n}{sigma\PYZus{}1}\PY{p}{)}\PY{p}{)}\PY{o}{*}\PY{p}{(} \PY{n}{sigma\PYZus{}1}\PY{o}{*}\PY{n}{np}\PY{o}{.}\PY{n}{power}\PY{p}{(}\PY{n}{D1}\PY{p}{,}\PY{n}{sigma\PYZus{}1}\PY{o}{\PYZhy{}}\PY{l+m+mi}{1}\PY{p}{)}\PY{o}{*}\PY{n}{Z} \PY{o}{+} \PY{p}{(}\PY{l+m+mi}{1}\PY{o}{\PYZhy{}} \PY{n}{sigma\PYZus{}1}\PY{p}{)}   \PY{p}{)}  \PY{o}{/} \PY{n}{np}\PY{o}{.}\PY{n}{square}\PY{p}{(}\PY{p}{(}\PY{n}{np}\PY{o}{.}\PY{n}{power}\PY{p}{(}\PY{n}{D1}\PY{p}{,} \PY{n}{sigma\PYZus{}1}\PY{p}{)}\PY{o}{*}\PY{n}{Z}\PY{p}{)}\PY{p}{)}
                \PY{k}{if} \PY{n}{j} \PY{o}{\PYZgt{}}\PY{o}{=} \PY{l+m+mi}{2} \PY{o+ow}{and} \PY{n}{k} \PY{o}{\PYZgt{}}\PY{o}{=} \PY{l+m+mi}{2}\PY{p}{:}
                    \PY{n}{derivatives}\PY{p}{[}\PY{n}{j}\PY{p}{,}\PY{n}{k}\PY{p}{]} \PY{o}{=} \PY{p}{(}\PY{o}{\PYZhy{}}\PY{n}{theta\PYZus{}2}\PY{o}{/}\PY{p}{(}\PY{l+m+mi}{1} \PY{o}{\PYZhy{}} \PY{n}{sigma\PYZus{}2}\PY{p}{)}\PY{p}{)}\PY{o}{*}\PY{n}{np}\PY{o}{.}\PY{n}{exp}\PY{p}{(}\PY{n}{v\PYZus{}t}\PY{p}{[}\PY{n}{j}\PY{p}{]}\PY{o}{/}\PY{p}{(}\PY{l+m+mi}{1}\PY{o}{\PYZhy{}} \PY{n}{sigma\PYZus{}2}\PY{p}{)}\PY{p}{)}\PY{o}{*}\PY{n}{np}\PY{o}{.}\PY{n}{exp}\PY{p}{(}\PY{n}{v\PYZus{}t}\PY{p}{[}\PY{n}{k}\PY{p}{]}\PY{o}{/}\PY{p}{(}\PY{l+m+mi}{1}\PY{o}{\PYZhy{}} \PY{n}{sigma\PYZus{}2}\PY{p}{)}\PY{p}{)}\PY{o}{*}\PY{p}{(} \PY{n}{sigma\PYZus{}1}\PY{o}{*}\PY{n}{np}\PY{o}{.}\PY{n}{power}\PY{p}{(}\PY{n}{D2}\PY{p}{,}\PY{n}{sigma\PYZus{}2}\PY{o}{\PYZhy{}}\PY{l+m+mi}{1}\PY{p}{)}\PY{o}{*}\PY{n}{Z} \PY{o}{+} \PY{p}{(}\PY{l+m+mi}{1}\PY{o}{\PYZhy{}} \PY{n}{sigma\PYZus{}2}\PY{p}{)}   \PY{p}{)}  \PY{o}{/} \PY{n}{np}\PY{o}{.}\PY{n}{square}\PY{p}{(}\PY{p}{(}\PY{n}{np}\PY{o}{.}\PY{n}{power}\PY{p}{(}\PY{n}{D2}\PY{p}{,} \PY{n}{sigma\PYZus{}2}\PY{p}{)}\PY{o}{*}\PY{n}{Z}\PY{p}{)}\PY{p}{)}
                \PY{k}{if} \PY{n}{j} \PY{o}{\PYZlt{}} \PY{l+m+mi}{2} \PY{o+ow}{and} \PY{n}{k} \PY{o}{\PYZgt{}}\PY{o}{=} \PY{l+m+mi}{2}\PY{p}{:}
                    \PY{n}{derivatives}\PY{p}{[}\PY{n}{j}\PY{p}{,}\PY{n}{k}\PY{p}{]} \PY{o}{=} \PY{p}{(}\PY{o}{\PYZhy{}}\PY{n}{theta\PYZus{}2}\PY{o}{/}\PY{p}{(}\PY{l+m+mi}{1} \PY{o}{\PYZhy{}} \PY{n}{sigma\PYZus{}2}\PY{p}{)}\PY{p}{)}\PY{o}{*}\PY{n}{np}\PY{o}{.}\PY{n}{exp}\PY{p}{(}\PY{n}{v\PYZus{}t}\PY{p}{[}\PY{n}{j}\PY{p}{]}\PY{o}{/}\PY{p}{(}\PY{l+m+mi}{1}\PY{o}{\PYZhy{}} \PY{n}{sigma\PYZus{}1}\PY{p}{)}\PY{p}{)}\PY{o}{*}\PY{n}{np}\PY{o}{.}\PY{n}{exp}\PY{p}{(}\PY{n}{v\PYZus{}t}\PY{p}{[}\PY{n}{k}\PY{p}{]}\PY{o}{/}\PY{p}{(}\PY{l+m+mi}{1}\PY{o}{\PYZhy{}} \PY{n}{sigma\PYZus{}2}\PY{p}{)}\PY{p}{)}\PY{o}{*}\PY{p}{(}\PY{l+m+mi}{1}\PY{o}{\PYZhy{}}\PY{n}{sigma\PYZus{}2}\PY{p}{)}\PY{o}{*}\PY{n}{np}\PY{o}{.}\PY{n}{power}\PY{p}{(}\PY{n}{D2}\PY{p}{,} \PY{o}{\PYZhy{}} \PY{n}{sigma\PYZus{}2}\PY{p}{)}\PY{o}{*}\PY{n}{np}\PY{o}{.}\PY{n}{power}\PY{p}{(}\PY{n}{D1}\PY{p}{,} \PY{n}{sigma\PYZus{}1}\PY{p}{)} \PY{o}{/} \PY{n}{np}\PY{o}{.}\PY{n}{square}\PY{p}{(}\PY{p}{(}\PY{n}{np}\PY{o}{.}\PY{n}{power}\PY{p}{(}\PY{n}{D1}\PY{p}{,} \PY{n}{sigma\PYZus{}1}\PY{p}{)}\PY{o}{*}\PY{n}{Z}\PY{p}{)}\PY{p}{)}
                \PY{k}{if} \PY{n}{j} \PY{o}{\PYZgt{}}\PY{o}{=} \PY{l+m+mi}{2} \PY{o+ow}{and} \PY{n}{k} \PY{o}{\PYZlt{}} \PY{l+m+mi}{2}\PY{p}{:}
                    \PY{n}{derivatives}\PY{p}{[}\PY{n}{j}\PY{p}{,}\PY{n}{k}\PY{p}{]} \PY{o}{=} \PY{p}{(}\PY{o}{\PYZhy{}}\PY{n}{theta\PYZus{}2}\PY{o}{/}\PY{p}{(}\PY{l+m+mi}{1} \PY{o}{\PYZhy{}} \PY{n}{sigma\PYZus{}1}\PY{p}{)}\PY{p}{)}\PY{o}{*}\PY{n}{np}\PY{o}{.}\PY{n}{exp}\PY{p}{(}\PY{n}{v\PYZus{}t}\PY{p}{[}\PY{n}{j}\PY{p}{]}\PY{o}{/}\PY{p}{(}\PY{l+m+mi}{1}\PY{o}{\PYZhy{}} \PY{n}{sigma\PYZus{}2}\PY{p}{)}\PY{p}{)}\PY{o}{*}\PY{n}{np}\PY{o}{.}\PY{n}{exp}\PY{p}{(}\PY{n}{v\PYZus{}t}\PY{p}{[}\PY{n}{k}\PY{p}{]}\PY{o}{/}\PY{p}{(}\PY{l+m+mi}{1}\PY{o}{\PYZhy{}} \PY{n}{sigma\PYZus{}1}\PY{p}{)}\PY{p}{)}\PY{o}{*}\PY{p}{(}\PY{l+m+mi}{1}\PY{o}{\PYZhy{}}\PY{n}{sigma\PYZus{}1}\PY{p}{)}\PY{o}{*}\PY{n}{np}\PY{o}{.}\PY{n}{power}\PY{p}{(}\PY{n}{D1}\PY{p}{,} \PY{o}{\PYZhy{}} \PY{n}{sigma\PYZus{}1}\PY{p}{)}\PY{o}{*}\PY{n}{np}\PY{o}{.}\PY{n}{power}\PY{p}{(}\PY{n}{D2}\PY{p}{,} \PY{n}{sigma\PYZus{}2}\PY{p}{)} \PY{o}{/} \PY{n}{np}\PY{o}{.}\PY{n}{square}\PY{p}{(}\PY{p}{(}\PY{n}{np}\PY{o}{.}\PY{n}{power}\PY{p}{(}\PY{n}{D2}\PY{p}{,} \PY{n}{sigma\PYZus{}2}\PY{p}{)}\PY{o}{*}\PY{n}{Z}\PY{p}{)}\PY{p}{)}


    \PY{n}{estimated\PYZus{}shares} \PY{o}{=} \PY{n}{mkt\PYZus{}share\PYZus{}nested}\PY{p}{(}\PY{n}{t}\PY{p}{,} \PY{n}{p}\PY{p}{,} \PY{n}{pars}\PY{p}{)}
    \PY{k}{return} \PY{n}{derivatives}\PY{p}{,} \PY{o}{\PYZhy{}}\PY{l+m+mi}{1}\PY{o}{*}\PY{n}{theta\PYZus{}2}\PY{o}{*}\PY{n}{estimated\PYZus{}shares}\PY{o}{*}\PY{p}{(}\PY{l+m+mi}{1} \PY{o}{\PYZhy{}} \PY{n}{estimated\PYZus{}shares}\PY{o}{.}\PY{n}{sum}\PY{p}{(}\PY{p}{)}\PY{p}{)}

\PY{k}{def} \PY{n+nf}{mkt\PYZus{}share\PYZus{}nested}\PY{p}{(}\PY{n}{t}\PY{p}{,} \PY{n}{p}\PY{p}{,} \PY{n}{pars}\PY{p}{)}\PY{p}{:}

    \PY{n}{XX} \PY{o}{=}  \PY{p}{[}\PY{n}{x}\PY{p}{[}\PY{p}{:}\PY{p}{,}\PY{n}{t}\PY{p}{]}\PY{p}{,} \PY{p}{[}\PY{l+m+mi}{1}\PY{p}{,}\PY{l+m+mi}{1}\PY{p}{,}\PY{l+m+mi}{0}\PY{p}{,}\PY{l+m+mi}{0}\PY{p}{]}\PY{p}{,} \PY{p}{[}\PY{l+m+mi}{0}\PY{p}{,}\PY{l+m+mi}{0}\PY{p}{,}\PY{l+m+mi}{1}\PY{p}{,}\PY{l+m+mi}{1}\PY{p}{]}\PY{p}{,} \PY{n}{p}  \PY{p}{]}

    \PY{n}{v\PYZus{}t} \PY{o}{=} \PY{n}{pars}\PY{p}{[}\PY{l+m+mi}{0}\PY{p}{:}\PY{l+m+mi}{4}\PY{p}{]} \PY{n+nd}{@XX}

    \PY{n}{sigma\PYZus{}1} \PY{o}{=} \PY{n}{pars}\PY{p}{[}\PY{l+m+mi}{4}\PY{p}{]}

    \PY{n}{sigma\PYZus{}2} \PY{o}{=} \PY{n}{pars}\PY{p}{[}\PY{l+m+mi}{5}\PY{p}{]}

    \PY{n}{theta\PYZus{}2} \PY{o}{=} \PY{n}{pars}\PY{p}{[}\PY{l+m+mi}{3}\PY{p}{]}


    \PY{n}{D1} \PY{o}{=} \PY{n}{np}\PY{o}{.}\PY{n}{exp}\PY{p}{(}\PY{n}{v\PYZus{}t}\PY{p}{[}\PY{l+m+mi}{0}\PY{p}{]}\PY{o}{/}\PY{p}{(}\PY{l+m+mi}{1}\PY{o}{\PYZhy{}} \PY{n}{sigma\PYZus{}1}\PY{p}{)}\PY{p}{)} \PY{o}{+} \PY{n}{np}\PY{o}{.}\PY{n}{exp}\PY{p}{(}\PY{n}{v\PYZus{}t}\PY{p}{[}\PY{l+m+mi}{1}\PY{p}{]}\PY{o}{/}\PY{p}{(}\PY{l+m+mi}{1}\PY{o}{\PYZhy{}} \PY{n}{sigma\PYZus{}1}\PY{p}{)}\PY{p}{)}

    \PY{n}{D2} \PY{o}{=} \PY{n}{np}\PY{o}{.}\PY{n}{exp}\PY{p}{(}\PY{n}{v\PYZus{}t}\PY{p}{[}\PY{l+m+mi}{2}\PY{p}{]}\PY{o}{/}\PY{p}{(}\PY{l+m+mi}{1}\PY{o}{\PYZhy{}} \PY{n}{sigma\PYZus{}2}\PY{p}{)}\PY{p}{)} \PY{o}{+} \PY{n}{np}\PY{o}{.}\PY{n}{exp}\PY{p}{(}\PY{n}{v\PYZus{}t}\PY{p}{[}\PY{l+m+mi}{3}\PY{p}{]}\PY{o}{/}\PY{p}{(}\PY{l+m+mi}{1}\PY{o}{\PYZhy{}} \PY{n}{sigma\PYZus{}2}\PY{p}{)}\PY{p}{)}

    \PY{n}{Z} \PY{o}{=} \PY{l+m+mi}{1} \PY{o}{+} \PY{n}{np}\PY{o}{.}\PY{n}{power}\PY{p}{(}\PY{n}{D1}\PY{p}{,} \PY{p}{(}\PY{l+m+mi}{1}\PY{o}{\PYZhy{}} \PY{n}{sigma\PYZus{}1}\PY{p}{)}\PY{p}{)} \PY{o}{+} \PY{n}{np}\PY{o}{.}\PY{n}{power}\PY{p}{(}\PY{n}{D2}\PY{p}{,} \PY{p}{(}\PY{l+m+mi}{1}\PY{o}{\PYZhy{}} \PY{n}{sigma\PYZus{}2}\PY{p}{)}\PY{p}{)}


    \PY{n}{shares} \PY{o}{=} \PY{n}{np}\PY{o}{.}\PY{n}{zeros}\PY{p}{(}\PY{p}{(}\PY{n}{J}\PY{p}{,}\PY{l+m+mi}{1}\PY{p}{)}\PY{p}{)}

    \PY{k}{for} \PY{n}{j} \PY{o+ow}{in} \PY{n+nb}{range}\PY{p}{(}\PY{n}{J}\PY{p}{)}\PY{p}{:}
        \PY{k}{if} \PY{n}{j} \PY{o}{\PYZlt{}} \PY{l+m+mi}{2}\PY{p}{:}
            \PY{n}{shares}\PY{p}{[}\PY{n}{j}\PY{p}{]} \PY{o}{=} \PY{p}{(}\PY{n}{np}\PY{o}{.}\PY{n}{exp}\PY{p}{(}\PY{n}{v\PYZus{}t}\PY{p}{[}\PY{n}{j}\PY{p}{]}\PY{o}{/}\PY{p}{(}\PY{l+m+mi}{1}\PY{o}{\PYZhy{}}\PY{n}{sigma\PYZus{}1}\PY{p}{)}\PY{p}{)}\PY{p}{)} \PY{o}{/}  \PY{p}{(}\PY{n}{np}\PY{o}{.}\PY{n}{power}\PY{p}{(}\PY{n}{D1}\PY{p}{,} \PY{n}{sigma\PYZus{}1}\PY{p}{)}\PY{o}{*}\PY{n}{Z}\PY{p}{)}
        \PY{k}{else}\PY{p}{:}
            \PY{n}{shares}\PY{p}{[}\PY{n}{j}\PY{p}{]} \PY{o}{=} \PY{p}{(}\PY{n}{np}\PY{o}{.}\PY{n}{exp}\PY{p}{(}\PY{n}{v\PYZus{}t}\PY{p}{[}\PY{n}{j}\PY{p}{]}\PY{o}{/}\PY{p}{(}\PY{l+m+mi}{1}\PY{o}{\PYZhy{}}\PY{n}{sigma\PYZus{}2}\PY{p}{)}\PY{p}{)}\PY{p}{)} \PY{o}{/}  \PY{p}{(}\PY{n}{np}\PY{o}{.}\PY{n}{power}\PY{p}{(}\PY{n}{D2}\PY{p}{,} \PY{n}{sigma\PYZus{}2}\PY{p}{)}\PY{o}{*}\PY{n}{Z}\PY{p}{)}


    \PY{k}{return} \PY{n}{shares}

\end{Verbatim}
\end{tcolorbox}

    \begin{tcolorbox}[breakable, size=fbox, boxrule=1pt, pad at break*=1mm,colback=cellbackground, colframe=cellborder]
\prompt{In}{incolor}{26}{\boxspacing}
\begin{Verbatim}[commandchars=\\\{\}]
\PY{c+c1}{\PYZsh{} Precompute the price elasticities and diversion }
\PY{n}{nested\PYZus{}logit\PYZus{}price\PYZus{}elasticities} \PY{o}{=} \PY{n}{np}\PY{o}{.}\PY{n}{zeros}\PY{p}{(}\PY{p}{(}\PY{n}{J}\PY{p}{,}\PY{n}{J}\PY{p}{,}\PY{n}{T}\PY{p}{)}\PY{p}{)}
\PY{n}{nested\PYZus{}logit\PYZus{}diversion\PYZus{}ratios} \PY{o}{=} \PY{n}{np}\PY{o}{.}\PY{n}{zeros}\PY{p}{(}\PY{p}{(}\PY{n}{J}\PY{p}{,}\PY{n}{J}\PY{p}{,}\PY{n}{T}\PY{p}{)}\PY{p}{)}

\PY{n}{N} \PY{o}{=} \PY{l+m+mi}{100}

\PY{k}{for} \PY{n}{t} \PY{o+ow}{in} \PY{n}{trange}\PY{p}{(}\PY{n}{T}\PY{p}{)}\PY{p}{:}
    \PY{n}{derivative\PYZus{}matrix}\PY{p}{,} \PY{n}{outside\PYZus{}derivative} \PY{o}{=} \PY{n}{full\PYZus{}mkt\PYZus{}share\PYZus{}derivative\PYZus{}nested}\PY{p}{(}\PY{n}{t}\PY{p}{,} \PY{n}{eq\PYZus{}prices}\PY{p}{[}\PY{p}{:}\PY{p}{,}\PY{n}{t}\PY{p}{]}\PY{p}{,} \PY{n}{iv\PYZus{}model}\PY{o}{.}\PY{n}{params}\PY{p}{)}
    \PY{n}{nested\PYZus{}logit\PYZus{}price\PYZus{}elasticities}\PY{p}{[}\PY{p}{:}\PY{p}{,}\PY{p}{:}\PY{p}{,}\PY{n}{t}\PY{p}{]} \PY{o}{=} \PY{n}{eq\PYZus{}prices}\PY{p}{[}\PY{p}{:}\PY{p}{,}\PY{n}{t}\PY{p}{]}\PY{o}{*}\PY{n}{derivative\PYZus{}matrix} \PY{o}{/} \PY{n}{eq\PYZus{}shares}\PY{p}{[}\PY{p}{:}\PY{p}{,}\PY{n}{t}\PY{p}{]}\PY{o}{.}\PY{n}{T}
    \PY{n}{estimated\PYZus{}shares} \PY{o}{=} \PY{n}{mkt\PYZus{}share\PYZus{}nested}\PY{p}{(}\PY{n}{t}\PY{p}{,} \PY{n}{eq\PYZus{}prices}\PY{p}{[}\PY{p}{:}\PY{p}{,}\PY{n}{t}\PY{p}{]}\PY{p}{,} \PY{n}{iv\PYZus{}model}\PY{o}{.}\PY{n}{params}\PY{p}{)}
    \PY{k}{for} \PY{n}{j} \PY{o+ow}{in} \PY{n+nb}{range}\PY{p}{(}\PY{n}{J}\PY{p}{)}\PY{p}{:}
        \PY{k}{for} \PY{n}{k} \PY{o+ow}{in} \PY{n+nb}{range}\PY{p}{(}\PY{n}{J}\PY{p}{)}\PY{p}{:}
            \PY{n}{nested\PYZus{}logit\PYZus{}diversion\PYZus{}ratios}\PY{p}{[}\PY{n}{j}\PY{p}{,}\PY{n}{k}\PY{p}{,}\PY{n}{t}\PY{p}{]} \PY{o}{=} \PY{o}{\PYZhy{}}\PY{l+m+mi}{1}\PY{o}{*}\PY{n}{derivative\PYZus{}matrix}\PY{p}{[}\PY{n}{k}\PY{p}{,}\PY{n}{j}\PY{p}{]}\PY{o}{/}\PY{n}{derivative\PYZus{}matrix}\PY{p}{[}\PY{n}{j}\PY{p}{,}\PY{n}{j}\PY{p}{]}
        \PY{n}{nested\PYZus{}logit\PYZus{}diversion\PYZus{}ratios}\PY{p}{[}\PY{n}{j}\PY{p}{,}\PY{n}{j}\PY{p}{,}\PY{n}{t}\PY{p}{]} \PY{o}{=} \PY{o}{\PYZhy{}}\PY{l+m+mi}{1}\PY{o}{*}\PY{n}{outside\PYZus{}derivative}\PY{p}{[}\PY{n}{j}\PY{p}{]}\PY{o}{/}\PY{n}{derivative\PYZus{}matrix}\PY{p}{[}\PY{n}{j}\PY{p}{,}\PY{n}{j}\PY{p}{]}
\end{Verbatim}
\end{tcolorbox}


    \begin{verbatim}
HBox(children=(IntProgress(value=0, max=600), HTML(value='')))
    \end{verbatim}


    \begin{Verbatim}[commandchars=\\\{\}]

    \end{Verbatim}

    \begin{tcolorbox}[breakable, size=fbox, boxrule=1pt, pad at break*=1mm,colback=cellbackground, colframe=cellborder]
\prompt{In}{incolor}{27}{\boxspacing}
\begin{Verbatim}[commandchars=\\\{\}]
\PY{n}{nested\PYZus{}logit\PYZus{}price\PYZus{}elasticities}\PY{o}{.}\PY{n}{mean}\PY{p}{(}\PY{n}{axis}\PY{o}{=}\PY{l+m+mi}{2}\PY{p}{)}\PY{p}{,} \PY{n}{true\PYZus{}price\PYZus{}elasticities}\PY{o}{.}\PY{n}{mean}\PY{p}{(}\PY{n}{axis}\PY{o}{=}\PY{l+m+mi}{2}\PY{p}{)}
\end{Verbatim}
\end{tcolorbox}

            \begin{tcolorbox}[breakable, size=fbox, boxrule=.5pt, pad at break*=1mm, opacityfill=0]
\prompt{Out}{outcolor}{27}{\boxspacing}
\begin{Verbatim}[commandchars=\\\{\}]
(array([[-6.12190672,  1.99361387,  1.14721285,  1.16871821],
        [ 1.94980523, -6.22209392,  1.23343822,  1.28434919],
        [ 1.18472154,  1.16184628, -6.06022182,  2.04574024],
        [ 1.15410091,  1.21673082,  1.99322672, -6.28818366]]),
 array([[-4.06535006,  1.38543391,  0.80172334,  0.7895892 ],
        [ 1.27934133, -4.16553436,  0.71112989,  0.71512854],
        [ 0.73928313,  0.74163481, -4.17726162,  1.3416553 ],
        [ 0.72070405,  0.7189693 ,  1.30923805, -4.18978309]]))
\end{Verbatim}
\end{tcolorbox}

    \begin{tcolorbox}[breakable, size=fbox, boxrule=1pt, pad at break*=1mm,colback=cellbackground, colframe=cellborder]
\prompt{In}{incolor}{28}{\boxspacing}
\begin{Verbatim}[commandchars=\\\{\}]
\PY{n}{nested\PYZus{}logit\PYZus{}diversion\PYZus{}ratios}\PY{o}{.}\PY{n}{mean}\PY{p}{(}\PY{n}{axis}\PY{o}{=}\PY{l+m+mi}{2}\PY{p}{)}\PY{p}{,} \PY{n}{true\PYZus{}diversion\PYZus{}ratios}\PY{o}{.}\PY{n}{mean}\PY{p}{(}\PY{n}{axis}\PY{o}{=}\PY{l+m+mi}{2}\PY{p}{)}
\end{Verbatim}
\end{tcolorbox}

            \begin{tcolorbox}[breakable, size=fbox, boxrule=.5pt, pad at break*=1mm, opacityfill=0]
\prompt{Out}{outcolor}{28}{\boxspacing}
\begin{Verbatim}[commandchars=\\\{\}]
(array([[0.28479711, 0.33297753, 0.19134439, 0.19088096],
        [0.32541291, 0.28728495, 0.19641401, 0.19088814],
        [0.19547647, 0.20791738, 0.28870153, 0.32231153],
        [0.19719596, 0.20488804, 0.32444768, 0.28792372]]),
 array([[0.33115087, 0.30335128, 0.18522023, 0.18027762],
        [0.32317153, 0.32122579, 0.18063565, 0.17496703],
        [0.19329289, 0.17575241, 0.32765373, 0.30330097],
        [0.19192008, 0.17341037, 0.31037504, 0.32429451]]))
\end{Verbatim}
\end{tcolorbox}

    \hypertarget{part-5}{%
\section{Part 5}\label{part-5}}

    \hypertarget{a-demand-side-estimation-only}{%
\subsection{5.a: Demand-side Estimation
only}\label{a-demand-side-estimation-only}}

    \begin{tcolorbox}[breakable, size=fbox, boxrule=1pt, pad at break*=1mm,colback=cellbackground, colframe=cellborder]
\prompt{In}{incolor}{29}{\boxspacing}
\begin{Verbatim}[commandchars=\\\{\}]
\PY{c+c1}{\PYZsh{} BLP, Demand\PYZhy{}side estimation only}

\PY{n}{demand\PYZus{}problem} \PY{o}{=} \PY{n}{pyblp}\PY{o}{.}\PY{n}{Problem}\PY{p}{(}
    \PY{p}{[}
        \PY{n}{pyblp}\PY{o}{.}\PY{n}{Formulation}\PY{p}{(}\PY{l+s+s2}{\PYZdq{}}\PY{l+s+s2}{0 + prices + x + satellite + wired}\PY{l+s+s2}{\PYZdq{}}\PY{p}{)}\PY{p}{,}
        \PY{n}{pyblp}\PY{o}{.}\PY{n}{Formulation}\PY{p}{(}\PY{l+s+s2}{\PYZdq{}}\PY{l+s+s2}{0 + satellite + wired}\PY{l+s+s2}{\PYZdq{}}\PY{p}{)}
    \PY{p}{]}\PY{p}{,}
    \PY{n}{observed\PYZus{}data}\PY{p}{[}\PY{p}{[}
        \PY{l+s+s2}{\PYZdq{}}\PY{l+s+s2}{market\PYZus{}ids}\PY{l+s+s2}{\PYZdq{}}\PY{p}{,}
        \PY{l+s+s2}{\PYZdq{}}\PY{l+s+s2}{firm\PYZus{}ids}\PY{l+s+s2}{\PYZdq{}}\PY{p}{,}
        \PY{l+s+s2}{\PYZdq{}}\PY{l+s+s2}{shares}\PY{l+s+s2}{\PYZdq{}}\PY{p}{,}
        \PY{l+s+s2}{\PYZdq{}}\PY{l+s+s2}{prices}\PY{l+s+s2}{\PYZdq{}}\PY{p}{,}
        \PY{l+s+s2}{\PYZdq{}}\PY{l+s+s2}{x}\PY{l+s+s2}{\PYZdq{}}\PY{p}{,}
        \PY{l+s+s2}{\PYZdq{}}\PY{l+s+s2}{satellite}\PY{l+s+s2}{\PYZdq{}}\PY{p}{,}
        \PY{l+s+s2}{\PYZdq{}}\PY{l+s+s2}{wired}\PY{l+s+s2}{\PYZdq{}}\PY{p}{]}\PY{p}{]}\PY{p}{,}
    \PY{n}{integration}\PY{o}{=}\PY{n}{pyblp}\PY{o}{.}\PY{n}{Integration}\PY{p}{(}\PY{l+s+s1}{\PYZsq{}}\PY{l+s+s1}{product}\PY{l+s+s1}{\PYZsq{}}\PY{p}{,} \PY{n}{size}\PY{o}{=}\PY{l+m+mi}{9}\PY{p}{)}\PY{p}{,}
\PY{p}{)}
\end{Verbatim}
\end{tcolorbox}

    \begin{Verbatim}[commandchars=\\\{\}]
Initializing the problem {\ldots}
Initialized the problem after 00:00:00.

Dimensions:
=======================================
 T    N     F     I     K1    K2    MD
---  ----  ---  -----  ----  ----  ----
600  2400   4   48600   4     2     3
=======================================

Formulations:
=================================================================
       Column Indices:             0        1        2        3
-----------------------------  ---------  -----  ---------  -----
 X1: Linear Characteristics     prices      x    satellite  wired
X2: Nonlinear Characteristics  satellite  wired
=================================================================
    \end{Verbatim}

    \begin{tcolorbox}[breakable, size=fbox, boxrule=1pt, pad at break*=1mm,colback=cellbackground, colframe=cellborder]
\prompt{In}{incolor}{30}{\boxspacing}
\begin{Verbatim}[commandchars=\\\{\}]
\PY{c+c1}{\PYZsh{} we will assume that the random coefficients on satellite and wired are uncorrellated}

\PY{c+c1}{\PYZsh{} this step is going to spit out a lot of text, most of which is not meaningful yet. }
\PY{c+c1}{\PYZsh{} the first iteration of .solve is only to compute the optimal instruments, and hence these first\PYZhy{}step estimates are not very good}

\PY{n}{demand\PYZus{}problem\PYZus{}w\PYZus{}instruments} \PY{o}{=} \PY{n}{demand\PYZus{}problem}\PY{o}{.}\PY{n}{solve}\PY{p}{(}\PY{n}{sigma}\PY{o}{=}\PY{n}{np}\PY{o}{.}\PY{n}{identity}\PY{p}{(}\PY{l+m+mi}{2}\PY{p}{)}\PY{p}{)}\PY{o}{.}\PY{n}{compute\PYZus{}optimal\PYZus{}instruments}\PY{p}{(}\PY{p}{)}\PY{o}{.}\PY{n}{to\PYZus{}problem}\PY{p}{(}\PY{p}{)}
\end{Verbatim}
\end{tcolorbox}

    \begin{Verbatim}[commandchars=\\\{\}]
Solving the problem {\ldots}

Nonlinear Coefficient Initial Values:
=======================================
 Sigma:      satellite        wired
---------  -------------  -------------
satellite  +1.000000E+00
  wired    +0.000000E+00  +1.000000E+00
=======================================

Nonlinear Coefficient Lower Bounds:
=======================================
 Sigma:      satellite        wired
---------  -------------  -------------
satellite  +0.000000E+00
  wired    +0.000000E+00  +0.000000E+00
=======================================

Nonlinear Coefficient Upper Bounds:
=======================================
 Sigma:      satellite        wired
---------  -------------  -------------
satellite      +INF
  wired    +0.000000E+00      +INF
=======================================

Starting optimization {\ldots}

GMM   Optimization   Objective   Fixed Point  Contraction  Clipped    Objective
Objective      Projected
Step   Iterations   Evaluations  Iterations   Evaluations  Shares       Value
Improvement   Gradient Norm             Theta
----  ------------  -----------  -----------  -----------  -------
-------------  -------------  -------------  ----------------------------
 1         0             1          4478         13776        0
+1.397861E-27                 +1.276800E-13  +1.000000E+00, +1.000000E+00

Optimization completed after 00:00:02.
Computing the Hessian and and updating the weighting matrix {\ldots}
Computed results after 00:00:10.

Problem Results Summary:
================================================================================
===============
GMM     Objective      Projected    Reduced Hessian  Reduced Hessian  Clipped
Weighting Matrix
Step      Value      Gradient Norm  Min Eigenvalue   Max Eigenvalue   Shares
Condition Number
----  -------------  -------------  ---------------  ---------------  -------
----------------
 1    +1.397861E-27  +1.276800E-13   +1.507973E-06    +7.409371E-06      0
+1.043996E+01
================================================================================
===============

Starting optimization {\ldots}

GMM   Optimization   Objective   Fixed Point  Contraction  Clipped    Objective
Objective      Projected
Step   Iterations   Evaluations  Iterations   Evaluations  Shares       Value
Improvement   Gradient Norm             Theta
----  ------------  -----------  -----------  -----------  -------
-------------  -------------  -------------  ----------------------------
 2         0             1            0           600         0
+1.073274E-27                 +3.029540E-13  +1.000000E+00, +1.000000E+00

Optimization completed after 00:00:01.
Computing the Hessian and estimating standard errors {\ldots}
Computed results after 00:00:07.

Problem Results Summary:
================================================================================
==================================
GMM     Objective      Projected    Reduced Hessian  Reduced Hessian  Clipped
Weighting Matrix  Covariance Matrix
Step      Value      Gradient Norm  Min Eigenvalue   Max Eigenvalue   Shares
Condition Number  Condition Number
----  -------------  -------------  ---------------  ---------------  -------
----------------  -----------------
 2    +1.073274E-27  +3.029540E-13   -2.208656E-05    +8.266331E-06      0
+9.227456E+00      +4.366231E+17
================================================================================
==================================

Cumulative Statistics:
===========================================================================
Computation  Optimizer  Optimization   Objective   Fixed Point  Contraction
   Time      Converged   Iterations   Evaluations  Iterations   Evaluations
-----------  ---------  ------------  -----------  -----------  -----------
 00:00:20       Yes          0             4          4478         14376
===========================================================================

Nonlinear Coefficient Estimates (Robust SEs in Parentheses):
===========================================
 Sigma:       satellite          wired
---------  ---------------  ---------------
satellite   +1.000000E+00
           (+4.410576E-03)

  wired     +0.000000E+00    +1.000000E+00
                            (+4.532765E-03)
===========================================

Beta Estimates (Robust SEs in Parentheses):
==================================================================
    prices              x            satellite          wired
---------------  ---------------  ---------------  ---------------
 -4.507325E-01    +8.461436E-01    -8.777418E-02    -1.191342E-01
(+1.126041E-02)  (+3.082421E-02)  (+1.866351E-02)  (+1.861941E-02)
==================================================================
Computing optimal instruments for theta {\ldots}
Computed optimal instruments after 00:00:01.

Optimal Instrument Results Summary:
=======================
Computation  Error Term
   Time        Draws
-----------  ----------
 00:00:01        1
=======================
Re-creating the problem {\ldots}
Re-created the problem after 00:00:00.

Dimensions:
=======================================
 T    N     F     I     K1    K2    MD
---  ----  ---  -----  ----  ----  ----
600  2400   4   48600   4     2     6
=======================================

Formulations:
=================================================================
       Column Indices:             0        1        2        3
-----------------------------  ---------  -----  ---------  -----
 X1: Linear Characteristics     prices      x    satellite  wired
X2: Nonlinear Characteristics  satellite  wired
=================================================================
    \end{Verbatim}

    \begin{tcolorbox}[breakable, size=fbox, boxrule=1pt, pad at break*=1mm,colback=cellbackground, colframe=cellborder]
\prompt{In}{incolor}{31}{\boxspacing}
\begin{Verbatim}[commandchars=\\\{\}]
\PY{c+c1}{\PYZsh{} now we resolve the problem given the optimal instruments}
\PY{n}{demand\PYZus{}problem\PYZus{}results} \PY{o}{=} \PY{n}{demand\PYZus{}problem\PYZus{}w\PYZus{}instruments}\PY{o}{.}\PY{n}{solve}\PY{p}{(}\PY{n}{sigma}\PY{o}{=}\PY{l+m+mf}{0.99}\PY{o}{*}\PY{n}{np}\PY{o}{.}\PY{n}{identity}\PY{p}{(}\PY{l+m+mi}{2}\PY{p}{)}\PY{p}{,}\PY{n}{optimization}\PY{o}{=}\PY{n}{pyblp}\PY{o}{.}\PY{n}{Optimization}\PY{p}{(}\PY{l+s+s1}{\PYZsq{}}\PY{l+s+s1}{l\PYZhy{}bfgs\PYZhy{}b}\PY{l+s+s1}{\PYZsq{}}\PY{p}{,} \PY{p}{\PYZob{}}\PY{l+s+s1}{\PYZsq{}}\PY{l+s+s1}{maxls}\PY{l+s+s1}{\PYZsq{}}\PY{p}{:} \PY{l+m+mi}{30}\PY{p}{\PYZcb{}}\PY{p}{)}\PY{p}{)}
\end{Verbatim}
\end{tcolorbox}

    \begin{Verbatim}[commandchars=\\\{\}]
Solving the problem {\ldots}

Nonlinear Coefficient Initial Values:
=======================================
 Sigma:      satellite        wired
---------  -------------  -------------
satellite  +9.900000E-01
  wired    +0.000000E+00  +9.900000E-01
=======================================

Nonlinear Coefficient Lower Bounds:
=======================================
 Sigma:      satellite        wired
---------  -------------  -------------
satellite  +0.000000E+00
  wired    +0.000000E+00  +0.000000E+00
=======================================

Nonlinear Coefficient Upper Bounds:
=======================================
 Sigma:      satellite        wired
---------  -------------  -------------
satellite      +INF
  wired    +0.000000E+00      +INF
=======================================

Starting optimization {\ldots}

GMM   Optimization   Objective   Fixed Point  Contraction  Clipped    Objective
Objective      Projected
Step   Iterations   Evaluations  Iterations   Evaluations  Shares       Value
Improvement   Gradient Norm             Theta
----  ------------  -----------  -----------  -----------  -------
-------------  -------------  -------------  ----------------------------
 1         0             1          4441         13671        0
-4.871846E+16                 +6.644277E+09  +9.900000E-01, +9.900000E-01
 1         0             2          3111         9787         0
-8.257676E+16  +3.385830E+16  +4.869321E+09  +2.828932E-01, +2.828932E-01
 1         1             3            0           600         0
-4.918336E+16                 +5.388201E-08  +0.000000E+00, +0.000000E+00
 1         1             4          3112         9794         0
-3.109528E+16                 +5.839840E+09  +2.828932E-01, +2.828932E-01
 1         1             5          3111         9787         0
-8.257676E+16                 +4.869321E+09  +2.828932E-01, +2.828932E-01
 1         1             6          3109         9791         0
-4.857029E+16                 +3.903649E+09  +2.828932E-01, +2.828932E-01
 1         1             7          3111         9787         0
-8.257676E+16                 +4.869321E+09  +2.828932E-01, +2.828932E-01

Optimization completed after 00:00:11.
Computing the Hessian and and updating the weighting matrix {\ldots}
Computed results after 00:00:08.

Problem Results Summary:
================================================================================
===============
GMM     Objective      Projected    Reduced Hessian  Reduced Hessian  Clipped
Weighting Matrix
Step      Value      Gradient Norm  Min Eigenvalue   Max Eigenvalue   Shares
Condition Number
----  -------------  -------------  ---------------  ---------------  -------
----------------
 1    -8.257676E+16  +4.869321E+09   -2.349511E+17    -2.874343E+16      0
+8.124302E+16
================================================================================
===============

Starting optimization {\ldots}

GMM   Optimization   Objective   Fixed Point  Contraction  Clipped    Objective
Objective      Projected
Step   Iterations   Evaluations  Iterations   Evaluations  Shares       Value
Improvement   Gradient Norm             Theta
----  ------------  -----------  -----------  -----------  -------
-------------  -------------  -------------  ----------------------------
 2         0             1            0           600         0
+6.537189E-13                 +1.091195E-12  +2.828932E-01, +2.828932E-01

Optimization completed after 00:00:01.
Computing the Hessian and estimating standard errors {\ldots}
Computed results after 00:00:05.

Problem Results Summary:
================================================================================
==================================
GMM     Objective      Projected    Reduced Hessian  Reduced Hessian  Clipped
Weighting Matrix  Covariance Matrix
Step      Value      Gradient Norm  Min Eigenvalue   Max Eigenvalue   Shares
Condition Number  Condition Number
----  -------------  -------------  ---------------  ---------------  -------
----------------  -----------------
 2    +6.537189E-13  +1.091195E-12   -2.784026E-12    +4.671904E-12      0
+2.505258E+17      +1.548412E+17
================================================================================
==================================

Cumulative Statistics:
===========================================================================
Computation  Optimizer  Optimization   Objective   Fixed Point  Contraction
   Time      Converged   Iterations   Evaluations  Iterations   Evaluations
-----------  ---------  ------------  -----------  -----------  -----------
 00:00:25       Yes          2            10          19995        63817
===========================================================================

Nonlinear Coefficient Estimates (Robust SEs in Parentheses):
===========================================
 Sigma:       satellite          wired
---------  ---------------  ---------------
satellite   +2.828932E-01
           (+1.012906E+00)

  wired     +0.000000E+00    +2.828932E-01
                            (+1.039249E+00)
===========================================

Beta Estimates (Robust SEs in Parentheses):
==================================================================
    prices              x            satellite          wired
---------------  ---------------  ---------------  ---------------
 +1.217125E+00    +5.763916E-01    -4.337985E+00    -4.407183E+00
(+2.509104E-02)  (+5.076744E-02)  (+1.455546E-01)  (+1.485866E-01)
==================================================================
    \end{Verbatim}

    These estimates are not bad.

    \hypertarget{a-demand-and-supply-estimation}{%
\subsection{5.a: Demand and Supply
Estimation}\label{a-demand-and-supply-estimation}}

    \begin{tcolorbox}[breakable, size=fbox, boxrule=1pt, pad at break*=1mm,colback=cellbackground, colframe=cellborder]
\prompt{In}{incolor}{32}{\boxspacing}
\begin{Verbatim}[commandchars=\\\{\}]
\PY{n}{full\PYZus{}problem} \PY{o}{=} \PY{n}{pyblp}\PY{o}{.}\PY{n}{Problem}\PY{p}{(}
    \PY{p}{[}
        \PY{n}{pyblp}\PY{o}{.}\PY{n}{Formulation}\PY{p}{(}\PY{l+s+s2}{\PYZdq{}}\PY{l+s+s2}{0 + prices + x + satellite + wired}\PY{l+s+s2}{\PYZdq{}}\PY{p}{)}\PY{p}{,}
        \PY{n}{pyblp}\PY{o}{.}\PY{n}{Formulation}\PY{p}{(}\PY{l+s+s2}{\PYZdq{}}\PY{l+s+s2}{0 + satellite + wired}\PY{l+s+s2}{\PYZdq{}}\PY{p}{)}\PY{p}{,}
        \PY{n}{pyblp}\PY{o}{.}\PY{n}{Formulation}\PY{p}{(}\PY{l+s+s2}{\PYZdq{}}\PY{l+s+s2}{1 + w}\PY{l+s+s2}{\PYZdq{}}\PY{p}{)}
    \PY{p}{]}\PY{p}{,}
    \PY{n}{product\PYZus{}data} \PY{o}{=} \PY{n}{observed\PYZus{}data}\PY{p}{,}
    \PY{n}{integration}\PY{o}{=}\PY{n}{pyblp}\PY{o}{.}\PY{n}{Integration}\PY{p}{(}\PY{l+s+s1}{\PYZsq{}}\PY{l+s+s1}{product}\PY{l+s+s1}{\PYZsq{}}\PY{p}{,} \PY{n}{size}\PY{o}{=}\PY{l+m+mi}{9}\PY{p}{)}\PY{p}{,}
    \PY{n}{costs\PYZus{}type}\PY{o}{=}\PY{l+s+s2}{\PYZdq{}}\PY{l+s+s2}{log}\PY{l+s+s2}{\PYZdq{}}
\PY{p}{)}
\end{Verbatim}
\end{tcolorbox}

    \begin{Verbatim}[commandchars=\\\{\}]
Initializing the problem {\ldots}
Initialized the problem after 00:00:00.

Dimensions:
===================================================
 T    N     F     I     K1    K2    K3    MD    MS
---  ----  ---  -----  ----  ----  ----  ----  ----
600  2400   4   48600   4     2     2     3     2
===================================================

Formulations:
=================================================================
       Column Indices:             0        1        2        3
-----------------------------  ---------  -----  ---------  -----
 X1: Linear Characteristics     prices      x    satellite  wired
X2: Nonlinear Characteristics  satellite  wired
X3: Log Cost Characteristics       1        w
=================================================================
    \end{Verbatim}

    \begin{tcolorbox}[breakable, size=fbox, boxrule=1pt, pad at break*=1mm,colback=cellbackground, colframe=cellborder]
\prompt{In}{incolor}{33}{\boxspacing}
\begin{Verbatim}[commandchars=\\\{\}]
\PY{c+c1}{\PYZsh{} once again, we construct optimal instruments}
\PY{n}{full\PYZus{}problem\PYZus{}w\PYZus{}instruments} \PY{o}{=} \PY{n}{full\PYZus{}problem}\PY{o}{.}\PY{n}{solve}\PY{p}{(}\PY{n}{sigma}\PY{o}{=}\PY{n}{np}\PY{o}{.}\PY{n}{identity}\PY{p}{(}\PY{l+m+mi}{2}\PY{p}{)}\PY{p}{,}\PY{n}{beta}\PY{o}{=}\PY{p}{[}\PY{o}{\PYZhy{}}\PY{l+m+mi}{1}\PY{p}{,}\PY{k+kc}{None}\PY{p}{,}\PY{k+kc}{None}\PY{p}{,}\PY{k+kc}{None}\PY{p}{]}\PY{p}{)}\PY{o}{.}\PY{n}{compute\PYZus{}optimal\PYZus{}instruments}\PY{p}{(}\PY{p}{)}\PY{o}{.}\PY{n}{to\PYZus{}problem}\PY{p}{(}\PY{p}{)}
\end{Verbatim}
\end{tcolorbox}

    \begin{Verbatim}[commandchars=\\\{\}]
Solving the problem {\ldots}

Nonlinear Coefficient Initial Values:
=======================================
 Sigma:      satellite        wired
---------  -------------  -------------
satellite  +1.000000E+00
  wired    +0.000000E+00  +1.000000E+00
=======================================

Beta Initial Values:
==========================================================
   prices            x          satellite        wired
-------------  -------------  -------------  -------------
-1.000000E+00       NAN            NAN            NAN
==========================================================

Nonlinear Coefficient Lower Bounds:
=======================================
 Sigma:      satellite        wired
---------  -------------  -------------
satellite  +0.000000E+00
  wired    +0.000000E+00  +0.000000E+00
=======================================

Beta Lower Bounds:
==========================================================
   prices            x          satellite        wired
-------------  -------------  -------------  -------------
    -INF           -INF           -INF           -INF
==========================================================

Nonlinear Coefficient Upper Bounds:
=======================================
 Sigma:      satellite        wired
---------  -------------  -------------
satellite      +INF
  wired    +0.000000E+00      +INF
=======================================

Beta Upper Bounds:
==========================================================
   prices            x          satellite        wired
-------------  -------------  -------------  -------------
    +INF           +INF           +INF           +INF
==========================================================

Starting optimization {\ldots}

GMM   Optimization   Objective   Fixed Point  Contraction  Clipped    Objective
Objective      Projected
Step   Iterations   Evaluations  Iterations   Evaluations  Shares       Value
Improvement   Gradient Norm                     Theta
----  ------------  -----------  -----------  -----------  -------
-------------  -------------  -------------
-------------------------------------------
 1         0             1          4478         13776        0
+1.420645E-26                 +2.916005E-11  +1.000000E+00, +1.000000E+00,
-1.000000E+00

Optimization completed after 00:00:03.
Computing the Hessian and and updating the weighting matrix {\ldots}
Computed results after 00:00:24.

Problem Results Summary:
================================================================================
===============
GMM     Objective      Projected    Reduced Hessian  Reduced Hessian  Clipped
Weighting Matrix
Step      Value      Gradient Norm  Min Eigenvalue   Max Eigenvalue   Shares
Condition Number
----  -------------  -------------  ---------------  ---------------  -------
----------------
 1    +1.420645E-26  +2.916005E-11   -5.950660E-04    +7.196896E-04      0
+1.338489E+01
================================================================================
===============

Starting optimization {\ldots}

GMM   Optimization   Objective   Fixed Point  Contraction  Clipped    Objective
Objective      Projected
Step   Iterations   Evaluations  Iterations   Evaluations  Shares       Value
Improvement   Gradient Norm                     Theta
----  ------------  -----------  -----------  -----------  -------
-------------  -------------  -------------
-------------------------------------------
 2         0             1            0           600         0
+1.213500E-25                 +1.303812E-11  +1.000000E+00, +1.000000E+00,
-1.000000E+00

Optimization completed after 00:00:02.
Computing the Hessian and estimating standard errors {\ldots}
Computed results after 00:00:18.

Problem Results Summary:
================================================================================
==================================
GMM     Objective      Projected    Reduced Hessian  Reduced Hessian  Clipped
Weighting Matrix  Covariance Matrix
Step      Value      Gradient Norm  Min Eigenvalue   Max Eigenvalue   Shares
Condition Number  Condition Number
----  -------------  -------------  ---------------  ---------------  -------
----------------  -----------------
 2    +1.213500E-25  +1.303812E-11   -6.052454E-03    +5.830682E-04      0
+7.176547E+01      +1.366273E+18
================================================================================
==================================

Cumulative Statistics:
===========================================================================
Computation  Optimizer  Optimization   Objective   Fixed Point  Contraction
   Time      Converged   Iterations   Evaluations  Iterations   Evaluations
-----------  ---------  ------------  -----------  -----------  -----------
 00:00:47       Yes          0             4          4478         14376
===========================================================================

Nonlinear Coefficient Estimates (Robust SEs in Parentheses):
===========================================
 Sigma:       satellite          wired
---------  ---------------  ---------------
satellite   +1.000000E+00
           (+4.250962E-03)

  wired     +0.000000E+00    +1.000000E+00
                            (+4.192555E-03)
===========================================

Beta Estimates (Robust SEs in Parentheses):
==================================================================
    prices              x            satellite          wired
---------------  ---------------  ---------------  ---------------
 -1.000000E+00    +9.090737E-01    +1.357618E+00    +1.341006E+00
(+8.317201E-03)  (+3.059498E-02)  (+2.117336E-02)  (+2.108796E-02)
==================================================================

Gamma Estimates (Robust SEs in Parentheses):
================================
       1                w
---------------  ---------------
 -1.406974E-01    +4.680814E-01
(+1.955098E-02)  (+1.085774E-02)
================================
Computing optimal instruments for theta {\ldots}
Computed optimal instruments after 00:00:04.

Optimal Instrument Results Summary:
=================================================
Computation  Error Term  Fixed Point  Contraction
   Time        Draws     Iterations   Evaluations
-----------  ----------  -----------  -----------
 00:00:04        1          9494         9494
=================================================
Re-creating the problem {\ldots}
Re-created the problem after 00:00:00.

Dimensions:
===================================================
 T    N     F     I     K1    K2    K3    MD    MS
---  ----  ---  -----  ----  ----  ----  ----  ----
600  2400   4   48600   4     2     2     7     8
===================================================

Formulations:
=================================================================
       Column Indices:             0        1        2        3
-----------------------------  ---------  -----  ---------  -----
 X1: Linear Characteristics     prices      x    satellite  wired
X2: Nonlinear Characteristics  satellite  wired
X3: Log Cost Characteristics       1        w
=================================================================
    \end{Verbatim}

    \begin{tcolorbox}[breakable, size=fbox, boxrule=1pt, pad at break*=1mm,colback=cellbackground, colframe=cellborder]
\prompt{In}{incolor}{34}{\boxspacing}
\begin{Verbatim}[commandchars=\\\{\}]
\PY{c+c1}{\PYZsh{} and here are the estimation results}
\PY{n}{full\PYZus{}problem\PYZus{}results} \PY{o}{=} \PY{n}{full\PYZus{}problem\PYZus{}w\PYZus{}instruments}\PY{o}{.}\PY{n}{solve}\PY{p}{(}\PY{n}{sigma}\PY{o}{=}\PY{l+m+mf}{0.9}\PY{o}{*}\PY{n}{np}\PY{o}{.}\PY{n}{identity}\PY{p}{(}\PY{l+m+mi}{2}\PY{p}{)}\PY{p}{,}\PY{n}{beta}\PY{o}{=}\PY{p}{[}\PY{o}{\PYZhy{}}\PY{l+m+mi}{1}\PY{p}{,}\PY{k+kc}{None}\PY{p}{,}\PY{k+kc}{None}\PY{p}{,}\PY{k+kc}{None}\PY{p}{]}\PY{p}{,} \PY{n}{check\PYZus{}optimality}\PY{o}{=}\PY{l+s+s2}{\PYZdq{}}\PY{l+s+s2}{both}\PY{l+s+s2}{\PYZdq{}}\PY{p}{)}
\end{Verbatim}
\end{tcolorbox}

    \begin{Verbatim}[commandchars=\\\{\}]
Solving the problem {\ldots}

Nonlinear Coefficient Initial Values:
=======================================
 Sigma:      satellite        wired
---------  -------------  -------------
satellite  +9.000000E-01
  wired    +0.000000E+00  +9.000000E-01
=======================================

Beta Initial Values:
==========================================================
   prices            x          satellite        wired
-------------  -------------  -------------  -------------
-1.000000E+00       NAN            NAN            NAN
==========================================================

Nonlinear Coefficient Lower Bounds:
=======================================
 Sigma:      satellite        wired
---------  -------------  -------------
satellite  +0.000000E+00
  wired    +0.000000E+00  +0.000000E+00
=======================================

Beta Lower Bounds:
==========================================================
   prices            x          satellite        wired
-------------  -------------  -------------  -------------
    -INF           -INF           -INF           -INF
==========================================================

Nonlinear Coefficient Upper Bounds:
=======================================
 Sigma:      satellite        wired
---------  -------------  -------------
satellite      +INF
  wired    +0.000000E+00      +INF
=======================================

Beta Upper Bounds:
==========================================================
   prices            x          satellite        wired
-------------  -------------  -------------  -------------
    +INF           +INF           +INF           +INF
==========================================================

Starting optimization {\ldots}

GMM   Optimization   Objective   Fixed Point  Contraction  Clipped    Objective
Objective      Projected
Step   Iterations   Evaluations  Iterations   Evaluations  Shares       Value
Improvement   Gradient Norm                     Theta
----  ------------  -----------  -----------  -----------  -------
-------------  -------------  -------------
-------------------------------------------
 1         0             1          4336         13321        0
-2.855162E+02                 +9.659102E+02  +9.000000E-01, +9.000000E-01,
-1.000000E+00
 1         0             2          4336         13295        0
-4.184583E+03  +3.899067E+03  +1.464040E+03  +8.888484E-01, +8.888484E-01,
-1.999876E+00
 1         0             3          4274         13119        0
+2.238252E+03                 +9.920075E+02  +8.442422E-01, +8.442422E-01,
-5.999378E+00
 1         0             4          4329         13295        0
+1.038995E+04                 +1.309991E+03  +8.834613E-01, +8.834613E-01,
-2.482895E+00
 1         0             5          4337         13292        0
-8.407922E+02                 +3.796392E+02  +8.888062E-01, +8.888062E-01,
-2.003663E+00
 1         0             6          4336         13295        0
-4.184583E+03                 +1.464040E+03  +8.888484E-01, +8.888484E-01,
-1.999876E+00
 1         1             7            0           600         0
+4.393448E+08                 +6.001273E+05  +0.000000E+00, +0.000000E+00,
-1.466040E+03
 1         1             8          4329         13284        0
+1.010082E+04                 +1.417132E+03  +8.867007E-01, +8.867007E-01,
-5.537365E+00
 1         1             9          4333         13295        0
-4.623683E+03  +4.390999E+02  +1.030407E+03  +8.887269E-01, +8.887269E-01,
-2.200115E+00
 1         2            10          4442         13663        0
-7.730336E+02                 +1.507782E+03  +8.786028E-01, +1.025720E+00,
-2.753598E+00
 1         2            11          4341         13317        0
+6.407188E+03                 +2.506296E+03  +8.885155E-01, +8.915872E-01,
-2.211671E+00
 1         2            12          4335         13305        0
-1.187810E+04  +7.254420E+03  +3.797859E+02  +8.887268E-01, +8.887274E-01,
-2.200117E+00
 1         3            13          4353         13380        0
-4.620186E+03                 +1.877459E+03  +8.881199E-01, +8.972040E-01,
-2.203337E+00
 1         3            14          4334         13303        0
-4.355088E+03                 +2.911085E+03  +8.887268E-01, +8.887278E-01,
-2.200117E+00
 1         3            15          4336         13303        0
-4.623684E+03                 +1.970713E+03  +8.887268E-01, +8.887274E-01,
-2.200117E+00
 1         3            16          4335         13305        0
-1.187810E+04                 +3.797859E+02  +8.887268E-01, +8.887274E-01,
-2.200117E+00

Optimization completed after 00:00:50.
Computing the Hessian and and updating the weighting matrix {\ldots}
Computed results after 00:00:22.

Problem Results Summary:
================================================================================
===============
GMM     Objective      Projected    Reduced Hessian  Reduced Hessian  Clipped
Weighting Matrix
Step      Value      Gradient Norm  Min Eigenvalue   Max Eigenvalue   Shares
Condition Number
----  -------------  -------------  ---------------  ---------------  -------
----------------
 1    -1.187810E+04  +3.797859E+02   -2.719718E+10    +6.320185E+10      0
+5.553615E+18
================================================================================
===============

Starting optimization {\ldots}

GMM   Optimization   Objective   Fixed Point  Contraction  Clipped    Objective
Objective      Projected
Step   Iterations   Evaluations  Iterations   Evaluations  Shares       Value
Improvement   Gradient Norm                     Theta
----  ------------  -----------  -----------  -----------  -------
-------------  -------------  -------------
-------------------------------------------
 2         0             1            0           600         0
+1.431873E+01                 +8.875149E+01  +8.887268E-01, +8.887274E-01,
-2.200117E+00
 2         0             2          4353         13332        0
+1.171137E+02                 +3.017598E+02  +1.046625E+00, +1.056654E+00,
-1.227045E+00
 2         0             3          4043         12373        0
+3.504647E+00  +1.081408E+01  +5.494750E+00  +9.263405E-01, +9.287298E-01,
-1.968317E+00
 2         1             4          4073         12481        0
+3.367158E+00  +1.374895E-01  +5.270739E+00  +9.381365E-01, +9.431001E-01,
-1.972367E+00
 2         1             5          4196         12890        0
+2.884910E+00  +4.822481E-01  +4.273512E+00  +9.853204E-01, +1.000581E+00,
-1.988567E+00
 2         2             6          4645         14209        0
+2.171413E+00  +7.134969E-01  +1.836328E+00  +1.207614E+00, +1.230418E+00,
-2.037731E+00
 2         3             7          4546         13907        0
+2.115084E+00  +5.632834E-02  +4.114676E-01  +1.166972E+00, +1.170077E+00,
-2.030503E+00
 2         4             8          4599         14044        0
+2.113129E+00  +1.955256E-03  +3.568824E-02  +1.169502E+00, +1.179589E+00,
-2.031245E+00
 2         5             9          4605         14059        0
+2.113105E+00  +2.412179E-05  +7.455956E-03  +1.168982E+00, +1.180300E+00,
-2.031264E+00
 2         6            10          4601         14058        0
+2.113104E+00  +6.821551E-07  +1.822522E-04  +1.168790E+00, +1.180292E+00,
-2.031252E+00
 2         7            11          4605         14056        0
+2.113104E+00  +4.416867E-11  +6.161568E-06  +1.168790E+00, +1.180292E+00,
-2.031252E+00
 2         8            12          4602         14057        0
+2.113104E+00  +1.625367E-13  +1.954205E-08  +1.168790E+00, +1.180292E+00,
-2.031252E+00
 2         9            13          4602         14056        0
+2.113104E+00  +5.240253E-14  +2.852862E-09  +1.168790E+00, +1.180292E+00,
-2.031252E+00

Optimization completed after 00:00:41.
Computing the Hessian and estimating standard errors {\ldots}
Computed results after 00:00:23.

Problem Results Summary:
================================================================================
==================================
GMM     Objective      Projected    Reduced Hessian  Reduced Hessian  Clipped
Weighting Matrix  Covariance Matrix
Step      Value      Gradient Norm  Min Eigenvalue   Max Eigenvalue   Shares
Condition Number  Condition Number
----  -------------  -------------  ---------------  ---------------  -------
----------------  -----------------
 2    +2.113104E+00  +2.852862E-09   +2.453329E+01    +3.924354E+02      0
+2.591693E+17      +2.872436E+04
================================================================================
==================================

Cumulative Statistics:
===========================================================================
Computation  Optimizer  Optimization   Objective   Fixed Point  Contraction
   Time      Converged   Iterations   Evaluations  Iterations   Evaluations
-----------  ---------  ------------  -----------  -----------  -----------
 00:02:16       Yes          14           31         118556       364494
===========================================================================

Nonlinear Coefficient Estimates (Robust SEs in Parentheses):
===========================================
 Sigma:       satellite          wired
---------  ---------------  ---------------
satellite   +1.168790E+00
           (+2.422035E-01)

  wired     +0.000000E+00    +1.180292E+00
                            (+2.322707E-01)
===========================================

Beta Estimates (Robust SEs in Parentheses):
==================================================================
    prices              x            satellite          wired
---------------  ---------------  ---------------  ---------------
 -2.031252E+00    +1.051940E+00    +4.030990E+00    +4.036894E+00
(+8.741266E-02)  (+4.717634E-02)  (+2.140120E-01)  (+2.153075E-01)
==================================================================

Gamma Estimates (Robust SEs in Parentheses):
================================
       1                w
---------------  ---------------
 +4.862138E-01    +2.634093E-01
(+1.981867E-02)  (+1.109183E-02)
================================
    \end{Verbatim}

    These estimates are even better than the previous section. We'll use
these in the coming sections.

    \hypertarget{b-own-price-elasticities-diversion-ratios}{%
\subsection{5.b Own-price Elasticities, Diversion
Ratios}\label{b-own-price-elasticities-diversion-ratios}}

    \begin{tcolorbox}[breakable, size=fbox, boxrule=1pt, pad at break*=1mm,colback=cellbackground, colframe=cellborder]
\prompt{In}{incolor}{35}{\boxspacing}
\begin{Verbatim}[commandchars=\\\{\}]
\PY{n}{estimated\PYZus{}price\PYZus{}elasticities} \PY{o}{=} \PY{n}{full\PYZus{}problem\PYZus{}results}\PY{o}{.}\PY{n}{compute\PYZus{}elasticities}\PY{p}{(}\PY{p}{)}
\end{Verbatim}
\end{tcolorbox}

    \begin{Verbatim}[commandchars=\\\{\}]
Computing elasticities with respect to prices {\ldots}
Finished after 00:00:01.

    \end{Verbatim}

    \begin{tcolorbox}[breakable, size=fbox, boxrule=1pt, pad at break*=1mm,colback=cellbackground, colframe=cellborder]
\prompt{In}{incolor}{36}{\boxspacing}
\begin{Verbatim}[commandchars=\\\{\}]
\PY{n}{estimated\PYZus{}diversion\PYZus{}ratios} \PY{o}{=} \PY{n}{full\PYZus{}problem\PYZus{}results}\PY{o}{.}\PY{n}{compute\PYZus{}diversion\PYZus{}ratios}\PY{p}{(}\PY{p}{)}
\end{Verbatim}
\end{tcolorbox}

    \begin{Verbatim}[commandchars=\\\{\}]
Computing diversion ratios with respect to prices {\ldots}
Finished after 00:00:01.

    \end{Verbatim}

    \begin{tcolorbox}[breakable, size=fbox, boxrule=1pt, pad at break*=1mm,colback=cellbackground, colframe=cellborder]
\prompt{In}{incolor}{37}{\boxspacing}
\begin{Verbatim}[commandchars=\\\{\}]
\PY{n}{estimated\PYZus{}own\PYZus{}price\PYZus{}elasticities} \PY{o}{=} \PY{n}{estimated\PYZus{}price\PYZus{}elasticities}\PY{o}{.}\PY{n}{reshape}\PY{p}{(}\PY{n}{T}\PY{p}{,}\PY{n}{J}\PY{p}{,}\PY{n}{J}\PY{p}{)}\PY{o}{.}\PY{n}{mean}\PY{p}{(}\PY{n}{axis}\PY{o}{=}\PY{l+m+mi}{0}\PY{p}{)}
\end{Verbatim}
\end{tcolorbox}

    \begin{tcolorbox}[breakable, size=fbox, boxrule=1pt, pad at break*=1mm,colback=cellbackground, colframe=cellborder]
\prompt{In}{incolor}{38}{\boxspacing}
\begin{Verbatim}[commandchars=\\\{\}]
\PY{n}{true\PYZus{}price\PYZus{}elasticities}\PY{o}{.}\PY{n}{mean}\PY{p}{(}\PY{n}{axis}\PY{o}{=}\PY{l+m+mi}{2}\PY{p}{)}\PY{p}{,} \PY{n}{estimated\PYZus{}own\PYZus{}price\PYZus{}elasticities}
\end{Verbatim}
\end{tcolorbox}

            \begin{tcolorbox}[breakable, size=fbox, boxrule=.5pt, pad at break*=1mm, opacityfill=0]
\prompt{Out}{outcolor}{38}{\boxspacing}
\begin{Verbatim}[commandchars=\\\{\}]
(array([[-4.06535006,  1.38543391,  0.80172334,  0.7895892 ],
        [ 1.27934133, -4.16553436,  0.71112989,  0.71512854],
        [ 0.73928313,  0.74163481, -4.17726162,  1.3416553 ],
        [ 0.72070405,  0.7189693 ,  1.30923805, -4.18978309]]),
 array([[-4.05026563,  1.36210624,  0.70040876,  0.66790244],
        [ 1.50046032, -4.1558555 ,  0.70040876,  0.66790244],
        [ 0.7378061 ,  0.65818679, -4.16101852,  1.38817984],
        [ 0.7378061 ,  0.65818679,  1.4468293 , -4.17569613]]))
\end{Verbatim}
\end{tcolorbox}

    The estimates are pretty close to the true values

    \begin{tcolorbox}[breakable, size=fbox, boxrule=1pt, pad at break*=1mm,colback=cellbackground, colframe=cellborder]
\prompt{In}{incolor}{39}{\boxspacing}
\begin{Verbatim}[commandchars=\\\{\}]
\PY{n}{estimated\PYZus{}diversion\PYZus{}ratios}\PY{o}{.}\PY{n}{reshape}\PY{p}{(}\PY{p}{(}\PY{n}{T}\PY{p}{,}\PY{n}{J}\PY{p}{,}\PY{n}{J}\PY{p}{)}\PY{p}{)}\PY{o}{.}\PY{n}{mean}\PY{p}{(}\PY{n}{axis}\PY{o}{=}\PY{l+m+mi}{0}\PY{p}{)}
\end{Verbatim}
\end{tcolorbox}

            \begin{tcolorbox}[breakable, size=fbox, boxrule=.5pt, pad at break*=1mm, opacityfill=0]
\prompt{Out}{outcolor}{39}{\boxspacing}
\begin{Verbatim}[commandchars=\\\{\}]
array([[0.32909482, 0.32544548, 0.17501464, 0.17044505],
       [0.3455732 , 0.3188287 , 0.17046779, 0.16513031],
       [0.18282689, 0.16629782, 0.3246221 , 0.32625318],
       [0.18145558, 0.16389933, 0.33358963, 0.32105546]])
\end{Verbatim}
\end{tcolorbox}

    \begin{tcolorbox}[breakable, size=fbox, boxrule=1pt, pad at break*=1mm,colback=cellbackground, colframe=cellborder]
\prompt{In}{incolor}{40}{\boxspacing}
\begin{Verbatim}[commandchars=\\\{\}]
\PY{n}{true\PYZus{}diversion\PYZus{}ratios}\PY{o}{.}\PY{n}{mean}\PY{p}{(}\PY{n}{axis}\PY{o}{=}\PY{l+m+mi}{2}\PY{p}{)}
\end{Verbatim}
\end{tcolorbox}

            \begin{tcolorbox}[breakable, size=fbox, boxrule=.5pt, pad at break*=1mm, opacityfill=0]
\prompt{Out}{outcolor}{40}{\boxspacing}
\begin{Verbatim}[commandchars=\\\{\}]
array([[0.33115087, 0.30335128, 0.18522023, 0.18027762],
       [0.32317153, 0.32122579, 0.18063565, 0.17496703],
       [0.19329289, 0.17575241, 0.32765373, 0.30330097],
       [0.19192008, 0.17341037, 0.31037504, 0.32429451]])
\end{Verbatim}
\end{tcolorbox}

    These look reasonably close as well.

    \hypertarget{part-6}{%
\section{Part 6}\label{part-6}}

    \begin{tcolorbox}[breakable, size=fbox, boxrule=1pt, pad at break*=1mm,colback=cellbackground, colframe=cellborder]
\prompt{In}{incolor}{41}{\boxspacing}
\begin{Verbatim}[commandchars=\\\{\}]
\PY{c+c1}{\PYZsh{} merge firms 1 and 2}
\PY{n}{observed\PYZus{}data}\PY{p}{[}\PY{l+s+s1}{\PYZsq{}}\PY{l+s+s1}{merger\PYZus{}1\PYZus{}ids}\PY{l+s+s1}{\PYZsq{}}\PY{p}{]} \PY{o}{=} \PY{n}{observed\PYZus{}data}\PY{p}{[}\PY{l+s+s1}{\PYZsq{}}\PY{l+s+s1}{firm\PYZus{}ids}\PY{l+s+s1}{\PYZsq{}}\PY{p}{]}\PY{o}{.}\PY{n}{replace}\PY{p}{(}\PY{l+m+mi}{2}\PY{p}{,} \PY{l+m+mi}{1}\PY{p}{)}

\PY{c+c1}{\PYZsh{} merge firms 1 and 3}
\PY{n}{observed\PYZus{}data}\PY{p}{[}\PY{l+s+s1}{\PYZsq{}}\PY{l+s+s1}{merger\PYZus{}2\PYZus{}ids}\PY{l+s+s1}{\PYZsq{}}\PY{p}{]} \PY{o}{=} \PY{n}{observed\PYZus{}data}\PY{p}{[}\PY{l+s+s1}{\PYZsq{}}\PY{l+s+s1}{firm\PYZus{}ids}\PY{l+s+s1}{\PYZsq{}}\PY{p}{]}\PY{o}{.}\PY{n}{replace}\PY{p}{(}\PY{l+m+mi}{3}\PY{p}{,} \PY{l+m+mi}{1}\PY{p}{)}
\end{Verbatim}
\end{tcolorbox}

    \begin{tcolorbox}[breakable, size=fbox, boxrule=1pt, pad at break*=1mm,colback=cellbackground, colframe=cellborder]
\prompt{In}{incolor}{42}{\boxspacing}
\begin{Verbatim}[commandchars=\\\{\}]
\PY{n}{marginal\PYZus{}costs} \PY{o}{=} \PY{n}{full\PYZus{}problem\PYZus{}results}\PY{o}{.}\PY{n}{compute\PYZus{}costs}\PY{p}{(}\PY{p}{)}

\PY{n}{merger\PYZus{}1\PYZus{}prices} \PY{o}{=} \PY{n}{full\PYZus{}problem\PYZus{}results}\PY{o}{.}\PY{n}{compute\PYZus{}prices}\PY{p}{(}
    \PY{n}{firm\PYZus{}ids}\PY{o}{=}\PY{n}{observed\PYZus{}data}\PY{p}{[}\PY{l+s+s1}{\PYZsq{}}\PY{l+s+s1}{merger\PYZus{}1\PYZus{}ids}\PY{l+s+s1}{\PYZsq{}}\PY{p}{]}\PY{p}{,}
    \PY{n}{costs}\PY{o}{=}\PY{n}{marginal\PYZus{}costs}
\PY{p}{)}

\PY{n}{merger\PYZus{}2\PYZus{}prices} \PY{o}{=} \PY{n}{full\PYZus{}problem\PYZus{}results}\PY{o}{.}\PY{n}{compute\PYZus{}prices}\PY{p}{(}
    \PY{n}{firm\PYZus{}ids}\PY{o}{=}\PY{n}{observed\PYZus{}data}\PY{p}{[}\PY{l+s+s1}{\PYZsq{}}\PY{l+s+s1}{merger\PYZus{}2\PYZus{}ids}\PY{l+s+s1}{\PYZsq{}}\PY{p}{]}\PY{p}{,}
    \PY{n}{costs}\PY{o}{=}\PY{n}{marginal\PYZus{}costs}
\PY{p}{)}
\end{Verbatim}
\end{tcolorbox}

    \begin{Verbatim}[commandchars=\\\{\}]
Computing marginal costs {\ldots}
Finished after 00:00:01.

Solving for equilibrium prices {\ldots}
Finished after 00:00:03.

Solving for equilibrium prices {\ldots}
Finished after 00:00:03.

    \end{Verbatim}

    \begin{tcolorbox}[breakable, size=fbox, boxrule=1pt, pad at break*=1mm,colback=cellbackground, colframe=cellborder]
\prompt{In}{incolor}{43}{\boxspacing}
\begin{Verbatim}[commandchars=\\\{\}]
\PY{n}{np}\PY{o}{.}\PY{n}{mean}\PY{p}{(}\PY{n}{eq\PYZus{}prices}\PY{p}{,} \PY{n}{axis}\PY{o}{=}\PY{l+m+mi}{1}\PY{p}{)}
\end{Verbatim}
\end{tcolorbox}

            \begin{tcolorbox}[breakable, size=fbox, boxrule=.5pt, pad at break*=1mm, opacityfill=0]
\prompt{Out}{outcolor}{43}{\boxspacing}
\begin{Verbatim}[commandchars=\\\{\}]
array([2.73266213, 2.71653207, 2.76078363, 2.73913598])
\end{Verbatim}
\end{tcolorbox}

    \begin{tcolorbox}[breakable, size=fbox, boxrule=1pt, pad at break*=1mm,colback=cellbackground, colframe=cellborder]
\prompt{In}{incolor}{44}{\boxspacing}
\begin{Verbatim}[commandchars=\\\{\}]
\PY{c+c1}{\PYZsh{} relative price changes, merging 1 and 2}
\PY{n}{np}\PY{o}{.}\PY{n}{mean}\PY{p}{(}\PY{n}{merger\PYZus{}1\PYZus{}prices}\PY{o}{.}\PY{n}{reshape}\PY{p}{(}\PY{p}{(}\PY{n}{T}\PY{p}{,}\PY{n}{J}\PY{p}{)}\PY{p}{)}\PY{p}{,}\PY{n}{axis}\PY{o}{=}\PY{l+m+mi}{0}\PY{p}{)}
\end{Verbatim}
\end{tcolorbox}

            \begin{tcolorbox}[breakable, size=fbox, boxrule=.5pt, pad at break*=1mm, opacityfill=0]
\prompt{Out}{outcolor}{44}{\boxspacing}
\begin{Verbatim}[commandchars=\\\{\}]
array([2.97945002, 2.99353235, 2.77124927, 2.74875349])
\end{Verbatim}
\end{tcolorbox}

    \begin{tcolorbox}[breakable, size=fbox, boxrule=1pt, pad at break*=1mm,colback=cellbackground, colframe=cellborder]
\prompt{In}{incolor}{45}{\boxspacing}
\begin{Verbatim}[commandchars=\\\{\}]
\PY{c+c1}{\PYZsh{} relative price changes, merging 1 and 3}
\PY{n}{np}\PY{o}{.}\PY{n}{mean}\PY{p}{(}\PY{n}{merger\PYZus{}2\PYZus{}prices}\PY{o}{.}\PY{n}{reshape}\PY{p}{(}\PY{p}{(}\PY{n}{T}\PY{p}{,}\PY{n}{J}\PY{p}{)}\PY{p}{)}\PY{p}{,}\PY{n}{axis}\PY{o}{=}\PY{l+m+mi}{0}\PY{p}{)}
\end{Verbatim}
\end{tcolorbox}

            \begin{tcolorbox}[breakable, size=fbox, boxrule=.5pt, pad at break*=1mm, opacityfill=0]
\prompt{Out}{outcolor}{45}{\boxspacing}
\begin{Verbatim}[commandchars=\\\{\}]
array([2.84694606, 2.72847439, 2.8831668 , 2.75133819])
\end{Verbatim}
\end{tcolorbox}

    \begin{tcolorbox}[breakable, size=fbox, boxrule=1pt, pad at break*=1mm,colback=cellbackground, colframe=cellborder]
\prompt{In}{incolor}{46}{\boxspacing}
\begin{Verbatim}[commandchars=\\\{\}]
\PY{n}{reduction\PYZus{}factors} \PY{o}{=} \PY{n}{np}\PY{o}{.}\PY{n}{concatenate}\PY{p}{(}\PY{p}{[}\PY{l+m+mf}{0.85}\PY{o}{*}\PY{n}{np}\PY{o}{.}\PY{n}{ones}\PY{p}{(}\PY{p}{[}\PY{n}{T}\PY{p}{,}\PY{l+m+mi}{2}\PY{p}{]}\PY{p}{)}\PY{p}{,}\PY{n}{np}\PY{o}{.}\PY{n}{ones}\PY{p}{(}\PY{p}{[}\PY{n}{T}\PY{p}{,}\PY{l+m+mi}{2}\PY{p}{]}\PY{p}{)}\PY{p}{]}\PY{p}{,}\PY{n}{axis}\PY{o}{=}\PY{l+m+mi}{1}\PY{p}{)}\PY{o}{.}\PY{n}{reshape}\PY{p}{(}\PY{p}{(}\PY{n}{T}\PY{o}{*}\PY{n}{J}\PY{p}{,}\PY{l+m+mi}{1}\PY{p}{)}\PY{p}{)}
\PY{n}{reduced\PYZus{}costs} \PY{o}{=} \PY{n}{marginal\PYZus{}costs} \PY{o}{*} \PY{n}{reduction\PYZus{}factors}


\PY{n}{merger\PYZus{}1\PYZus{}prices\PYZus{}w\PYZus{}cost\PYZus{}reduction} \PY{o}{=} \PY{n}{full\PYZus{}problem\PYZus{}results}\PY{o}{.}\PY{n}{compute\PYZus{}prices}\PY{p}{(}
    \PY{n}{firm\PYZus{}ids}\PY{o}{=}\PY{n}{observed\PYZus{}data}\PY{p}{[}\PY{l+s+s1}{\PYZsq{}}\PY{l+s+s1}{merger\PYZus{}1\PYZus{}ids}\PY{l+s+s1}{\PYZsq{}}\PY{p}{]}\PY{p}{,}
    \PY{n}{costs}\PY{o}{=}\PY{n}{reduced\PYZus{}costs}
\PY{p}{)}
\end{Verbatim}
\end{tcolorbox}

    \begin{Verbatim}[commandchars=\\\{\}]
Solving for equilibrium prices {\ldots}
Finished after 00:00:03.

    \end{Verbatim}

    \begin{tcolorbox}[breakable, size=fbox, boxrule=1pt, pad at break*=1mm,colback=cellbackground, colframe=cellborder]
\prompt{In}{incolor}{47}{\boxspacing}
\begin{Verbatim}[commandchars=\\\{\}]
\PY{c+c1}{\PYZsh{} post\PYZhy{}merger relative price changes, 1 and 2 with marginal cost reduction}
\PY{n}{np}\PY{o}{.}\PY{n}{mean}\PY{p}{(}\PY{n}{merger\PYZus{}1\PYZus{}prices\PYZus{}w\PYZus{}cost\PYZus{}reduction}\PY{o}{.}\PY{n}{reshape}\PY{p}{(}\PY{p}{(}\PY{n}{T}\PY{p}{,}\PY{n}{J}\PY{p}{)}\PY{p}{)}\PY{p}{,}\PY{n}{axis}\PY{o}{=}\PY{l+m+mi}{0}\PY{p}{)}
\end{Verbatim}
\end{tcolorbox}

            \begin{tcolorbox}[breakable, size=fbox, boxrule=.5pt, pad at break*=1mm, opacityfill=0]
\prompt{Out}{outcolor}{47}{\boxspacing}
\begin{Verbatim}[commandchars=\\\{\}]
array([2.78201464, 2.79398463, 2.76110894, 2.73900857])
\end{Verbatim}
\end{tcolorbox}

    \begin{tcolorbox}[breakable, size=fbox, boxrule=1pt, pad at break*=1mm,colback=cellbackground, colframe=cellborder]
\prompt{In}{incolor}{48}{\boxspacing}
\begin{Verbatim}[commandchars=\\\{\}]
\PY{n}{pre\PYZus{}merger\PYZus{}surpluses} \PY{o}{=} \PY{n}{full\PYZus{}problem\PYZus{}results}\PY{o}{.}\PY{n}{compute\PYZus{}consumer\PYZus{}surpluses}\PY{p}{(}\PY{p}{)}
\PY{n}{post\PYZus{}merger\PYZus{}surpluses} \PY{o}{=} \PY{n}{full\PYZus{}problem\PYZus{}results}\PY{o}{.}\PY{n}{compute\PYZus{}consumer\PYZus{}surpluses}\PY{p}{(}\PY{n}{prices}\PY{o}{=}\PY{n}{merger\PYZus{}1\PYZus{}prices\PYZus{}w\PYZus{}cost\PYZus{}reduction}\PY{p}{)}
\end{Verbatim}
\end{tcolorbox}

    \begin{Verbatim}[commandchars=\\\{\}]
Computing consumer surpluses with the equation that assumes away nonlinear
income effects {\ldots}
Finished after 00:00:01.

Computing consumer surpluses with the equation that assumes away nonlinear
income effects {\ldots}
Finished after 00:00:01.

    \end{Verbatim}

    \begin{tcolorbox}[breakable, size=fbox, boxrule=1pt, pad at break*=1mm,colback=cellbackground, colframe=cellborder]
\prompt{In}{incolor}{49}{\boxspacing}
\begin{Verbatim}[commandchars=\\\{\}]
\PY{c+c1}{\PYZsh{} assuming measure of consumers in each market is 1, the net surpluses are just the sums}
\PY{c+c1}{\PYZsh{} this is the net effect on consumer welfare}
\PY{n}{np}\PY{o}{.}\PY{n}{sum}\PY{p}{(}\PY{n}{post\PYZus{}merger\PYZus{}surpluses} \PY{o}{\PYZhy{}} \PY{n}{pre\PYZus{}merger\PYZus{}surpluses}\PY{p}{)}
\end{Verbatim}
\end{tcolorbox}

            \begin{tcolorbox}[breakable, size=fbox, boxrule=.5pt, pad at break*=1mm, opacityfill=0]
\prompt{Out}{outcolor}{49}{\boxspacing}
\begin{Verbatim}[commandchars=\\\{\}]
-6.5718246112984176
\end{Verbatim}
\end{tcolorbox}

    \begin{tcolorbox}[breakable, size=fbox, boxrule=1pt, pad at break*=1mm,colback=cellbackground, colframe=cellborder]
\prompt{In}{incolor}{50}{\boxspacing}
\begin{Verbatim}[commandchars=\\\{\}]
\PY{n}{post\PYZus{}merger\PYZus{}shares} \PY{o}{=} \PY{n}{full\PYZus{}problem\PYZus{}results}\PY{o}{.}\PY{n}{compute\PYZus{}shares}\PY{p}{(}\PY{n}{merger\PYZus{}1\PYZus{}prices\PYZus{}w\PYZus{}cost\PYZus{}reduction}\PY{p}{)}
\PY{n}{pre\PYZus{}merger\PYZus{}profits} \PY{o}{=} \PY{n}{full\PYZus{}problem\PYZus{}results}\PY{o}{.}\PY{n}{compute\PYZus{}profits}\PY{p}{(}\PY{p}{)}
\PY{n}{post\PYZus{}merger\PYZus{}profits} \PY{o}{=} \PY{n}{full\PYZus{}problem\PYZus{}results}\PY{o}{.}\PY{n}{compute\PYZus{}profits}\PY{p}{(}\PY{n}{merger\PYZus{}1\PYZus{}prices\PYZus{}w\PYZus{}cost\PYZus{}reduction}\PY{p}{,} \PY{n}{post\PYZus{}merger\PYZus{}shares}\PY{p}{,} \PY{n}{reduced\PYZus{}costs}\PY{p}{)}
\end{Verbatim}
\end{tcolorbox}

    \begin{Verbatim}[commandchars=\\\{\}]
Computing shares {\ldots}
Finished after 00:00:00.

Computing profits {\ldots}
Finished after 00:00:01.

Computing profits {\ldots}
Finished after 00:00:01.

    \end{Verbatim}

    \begin{tcolorbox}[breakable, size=fbox, boxrule=1pt, pad at break*=1mm,colback=cellbackground, colframe=cellborder]
\prompt{In}{incolor}{51}{\boxspacing}
\begin{Verbatim}[commandchars=\\\{\}]
\PY{c+c1}{\PYZsh{} once again assuming measure 1 of consumers in each market}
\PY{c+c1}{\PYZsh{} net change in profits}
\PY{n}{np}\PY{o}{.}\PY{n}{sum}\PY{p}{(}\PY{n}{post\PYZus{}merger\PYZus{}profits} \PY{o}{\PYZhy{}} \PY{n}{pre\PYZus{}merger\PYZus{}profits}\PY{p}{)}
\end{Verbatim}
\end{tcolorbox}

            \begin{tcolorbox}[breakable, size=fbox, boxrule=.5pt, pad at break*=1mm, opacityfill=0]
\prompt{Out}{outcolor}{51}{\boxspacing}
\begin{Verbatim}[commandchars=\\\{\}]
69.19100664228262
\end{Verbatim}
\end{tcolorbox}

    \begin{tcolorbox}[breakable, size=fbox, boxrule=1pt, pad at break*=1mm,colback=cellbackground, colframe=cellborder]
\prompt{In}{incolor}{52}{\boxspacing}
\begin{Verbatim}[commandchars=\\\{\}]
\PY{c+c1}{\PYZsh{} welfare change}
\PY{n}{np}\PY{o}{.}\PY{n}{sum}\PY{p}{(}\PY{n}{post\PYZus{}merger\PYZus{}surpluses} \PY{o}{\PYZhy{}} \PY{n}{pre\PYZus{}merger\PYZus{}surpluses}\PY{p}{)} \PY{o}{+} \PY{n}{np}\PY{o}{.}\PY{n}{sum}\PY{p}{(}\PY{n}{post\PYZus{}merger\PYZus{}profits} \PY{o}{\PYZhy{}} \PY{n}{pre\PYZus{}merger\PYZus{}profits}\PY{p}{)}
\end{Verbatim}
\end{tcolorbox}

            \begin{tcolorbox}[breakable, size=fbox, boxrule=.5pt, pad at break*=1mm, opacityfill=0]
\prompt{Out}{outcolor}{52}{\boxspacing}
\begin{Verbatim}[commandchars=\\\{\}]
62.6191820309842
\end{Verbatim}
\end{tcolorbox}

\end{document}
	% line of code telling latex that your document is ending. If you leave this out, you'll get an error
