\documentclass{article}
	% basic article document class
	% use percent signs to make comments to yourself -- they will not show up.

\usepackage{amsmath}
\usepackage{amssymb}
	% packages that allow mathematical formatting

\usepackage{graphicx}
	% package that allows you to include graphics

\usepackage{setspace}
	% package that allows you to change spacing

\onehalfspacing
	% text become 1.5 spaced

\usepackage{fullpage}
	% package that specifies normal margins


\begin{document}
	% line of code telling latex that your document is beginning


\title{Summary: Berry 1994}

\author{Nicholas Wu}

\date{Fall 2021}
	% Note: when you omit this command, the current dateis automatically included

\maketitle
	% tells latex to follow your header (e.g., title, author) commands.

In supply-and-demand markets with homogeneous goods, one can use traditional instrumental variables to estimate demand parameters to handle the presence of unobserved demand factors. However, in supply-and-demand markets with product differentiation and a discrete-choice model, market shares are a nonlinear function of prices and unobserved product characteristics, and hence one cannot use a straightforward application of traditional instrumental variables to resolve the price endogeneity. The objective of Berry's \textit{Estimating Discrete-Choice Models of Product Differentiation} is to estimate such discrete-choice supply-and-demand models with product differentiation by avoiding the nonlinear instrumental variables issue.

The key idea and contribution of the paper centers around the implementation of a random coefficients specification for utility; that is, utility of a consumer is assumed to be a linear function of quantity, price, and hidden characteristics, with coefficients to be estimated. Then simply by construction, the mean utility level of a product across all consumers is linear in quantity, price, and unobserved characteristics. Therefore, if one can determine the mean utility level of a product (under some normalization) from the market shares, then the price endogeneity can be resolved with instrumental variables.

Berry specifically proves that under some weak conditions, the expression for market shares as a function of mean utility levels can be inverted; for an observation of the market shares, one can derive a unique vector of mean utility levels for each product (normalizing the outside-option to 0). This provides the other piece; since mean utility levels can be determined from market shares, and mean utility levels are linear in price and quantity, price endogeneity can be resolved by selecting the appropriate instrument(s) for prices.

This method of inverting the market-share function to recover mean utility levels has advantages over other potential approaches for dealing with price-correlated unobserved demand characteristics. Since the price-setting model implies that the unobserved product characteristic cannot be exogenous conditional on prices and quantities, one cannot integrate out the contribution of the unobserved characteristic. Additionally, the unobserved product characteristics are not identified separately from the coefficients on prices and firm-specific unobservables; hence, it is not possible to estimate the unobserved product characteristics as a fixed effect. Another approach, solving for the reduced form, is generally difficult to calculate; further, it is difficult to establish equilibrium uniqueness in the reduced form model.

Berry also provides Monte-Carlo simulations to show how the approach introduced in this paper can correct for biases due to unobserved product characteristics. In concluding, the paper notes a number of future directions, including relaxing the assumption that product characteristics are econometrically exogenous, using less restrictive consumer and firm preferences, and accounting for measurement error. 
\end{document}
	% line of code telling latex that your document is ending. If you leave this out, you'll get an error
